\section[变分法]{变分法} \label{sec:06.03} % 
% \makebox[5em][s]{} % 短题目拉间距

考虑任一物理体系的束缚态,以$\hat{H}$表示体系的总能量算符($\hat{H}$与时间无关).任取一个归一化的波函数$\varPsi$(与时间无关),计算$\hat{H}$的平均值,并记为$E$,即
\begin{empheq}{align}
	&\int\varPsi^{*}\varPsi d\tau=\langle \varPsi|\varPsi \rangle =1		\label{eq63.1}\\
	E&=\int\varPsi^{*}\hat{H}\varPsi d\tau=\langle \varPsi|\hat{H}|\varPsi \rangle	\label{eq63.2}
\end{empheq}
然后令波函数作任意微小变化(变分)
\begin{empheq}{equation}\label{eq63.3}
	\varPsi\rightarrow\phi=\varPsi+\delta \varPsi
\end{empheq}
对于函数$\phi$重新计算$\hat{H}$的平均值,记为$E+\delta E$
\begin{empheq}{equation}\label{eq63.4}
	E+\delta E=\frac{\int\phi^{*}\hat{H}\phi d\tau}{\int\phi^{*}\phi d\tau}
\end{empheq}
(注意,$\phi$可能不再是归一化的)将\eqref{eq63.3}式代入\eqref{eq63.4}式,略去二级小量($\delta\varPsi$的二次项),得到如下结果:
\eqlong
\begin{empheq}{align*}
	\int\phi^{*}\phi d\tau &=1+\bigg(\int\varPsi\delta\varPsi^{*}d\tau+c.c \bigg)	\\
	\int\phi^{*}\hat{H}\phi d\tau&=\int(\varPsi^{*}+\delta\varPsi^{*})\hat{H}(\varPsi+\delta\varPsi)d\tau	\\
	&=E+\int\delta\varPsi^{*}\hat{H}\varPsi d\tau+\int\varPsi^{*}\hat{H}\delta\varPsi d\tau	\\
	&=E+\bigg(\int\delta\varPsi^{*}\hat{H}\varPsi d\tau+c.c \bigg)	\\
	&=E+\delta E+E\bigg(\int\varPsi\delta\varPsi^{*}d\tau+c.c\bigg)
\end{empheq}
即
\begin{empheq}{equation}\label{eq63.5}
	\delta E=\int\delta\varPsi^{*}\hat{H}\varPsi d\tau+c.c.-E\bigg(\int\delta\varPsi^{*}\varPsi d\tau+c.c \bigg)
\end{empheq}\eqnormal
考虑到$\delta\varPsi$的任意性,由\eqref{eq63.5}式容易看出以下结论:

(i) 如果$\varPsi$满足方程$\hat{H}\varPsi=E\varPsi$,则$\delta E=0$.

(ii) 如果对任意$\delta\varPsi$要求$\delta E=0$,则$\varPsi$必须满足
\eqshort
\begin{empheq}{equation}\label{eq63.6}
	\hat{H}\varPsi=E\varPsi
\end{empheq}\eqnormal
总之,“$\hat{H}$的平均值取变分极值$(\delta E=0)$”与“$\varPsi$为$\hat{H}$的本征函数”是等价的,这称为变分原理.

这个变分原理曾被薛定谔用来建立波动力学.本节向读者介绍一种寻找束缚态波函数与能级的近似方法,俗称变分法,它的理论基础就是上述变分原理.

设体系的总能量算符$\hat{H}$已经给定,但能量本征方程无法精确求解.设能级(按照由低到高的次序)为$E_{0},E_{1},E_{2},\cdots$相应的束缚态波函数(归一化的)为$\varPsi_{0}$,$\varPsi_{1}$,$\varPsi_{2}$,$\cdots$.任取一个归一化的波函数$\varPsi$,按照态叠加原理,$\varPsi$在原则上可以表示成能量本征函数的线性叠加:
\begin{empheq}{equation}\label{eq63.7}
	\varPsi=\sum_{n}C_{n}\varPsi_{n},\quad \sum_{n}C_{n}^{*}C_{n}=1
\end{empheq}
则在$\varPsi$态中$\hat{H}$的平均值(记为$E$)为
\begin{empheq}{equation}\label{eq63.8}
	E=\int\varPsi^{*}\hat{H}\varPsi d\tau=\sum_{n}C_{n}^{*}C_{n}E_{n}
\end{empheq}
由于$E_{n}\geqslant E_{0}$,显然$E\geqslant E_{0}$,$E_{0}$为基态能级.以上分析提示我们可以这样来寻找基态$\varPsi_{0}$,$E_{0}$.选取一种估计接近于基态的函数(归一化的)形式$\varPsi(\lambda,x)$,$x$代表坐标变量,$\lambda$为待定参数(一个或多个).对$\varPsi(\lambda,x)$计算$\hat{H}$的平均值,记为$E(\lambda)$,即
\begin{empheq}{equation}\label{eq63.9}
	E(\lambda)=\int\varPsi^{*}(\lambda,x)\hat{H}\varPsi(\lambda,x)d\tau
\end{empheq}
在不改变$\varPsi$的函数形式的条件下,根据变分极值条件
\begin{empheq}{equation}\label{eq63.10}
	\delta E=0,\quad \text{即}\frac{\partial E(\lambda)}{\partial\lambda}=0
\end{empheq}
求出$\lambda$的最佳值$\lambda_{0}$,相应的$\varPsi(\lambda_{0},x)$及$E(\lambda_{0}$即可作为基态波函数$\varPsi_{0}$及基态能级$E_{0}$的近似.如所得结果不够满意,可以另选一种$\varPsi$的函数形式,重复上述计算.在选择试探波函数$\varPsi(\lambda,x)$的函数形式时,当然应充分考虑基态的各方面性质,并可将其他近似方法(例如微扰论)求得的结果作为参考.一般说来,用这种参数变分法通常总能求得较好的结果,当然,计算工作量有时较大,需要电子计算机的帮助.

同样的方法也可用来寻找激发态$\varPsi_{n}$,$E_{n}$的近似,条件是所有能量小于$E_{n}$的能量本征态($\varPsi_{0}$,$\varPsi_{1}$,$\cdots$,$\varPsi_{n-1}$)均已求出,这时$\varPsi_{n}$的试探函数$\varPsi(\lambda,x)$应选取成与这些波函数正交,以保证在展开式\eqref{eq63.7}中没有这些状态的成分,则
\eqindent{7}
\begin{empheq}{align*}
	\varPsi&(\lambda,x)=C_{n}\varPsi_{n}+C_{n+1}\varPsi_{n+1}+\cdots	\\
	E(\lambda)&=C_{n}^{*}C_{n}E_{n}+C_{n+1}^{*}C_{n+1}E_{n+1}+\cdots\geqslant E_{n}
\end{empheq}\eqnormal
再用变分极值条件\eqref{eq63.10}式求出$\lambda$的最佳值$\lambda_{0}$,即可得到$\varPsi_{n}$,$E_{n}$的近似结果.

\example 粒子(质量$m$)在无限深势阱$(-a<x<a)$中运动.试用多项式近似结合变分法求基态能级的近似值,并与精确解比较.

\solution 精确解见$\S$\ref{sec:02.04},基态为(用本节的符号)
\begin{empheq}{equation}\label{eq63.11}
	\varPsi_{0}(x)=\sqrt{\frac{1}{a}}\cos\frac{\pi x}{2a},\quad E_{0}=\frac{\pi^{2}\hbar^{2}}{8ma^{2}}
\end{empheq}
$\varPsi_{0}$为偶宇称,并满足边界条件$\varPsi(a)=\varPsi(-a)=0$.

如用多项式作为$\varPsi_{0}$的近似表示,考虑到此为偶宇称,应取
\begin{empheq}{equation}\label{eq63.12}
	\varPsi=C_{0}+C_{2}\bigg(\frac{x}{a}\bigg)^{2}+C_{4}\bigg(\frac{x}{a}\bigg)^{4}+\cdots
\end{empheq}
如只取前两项,为了满足边界条件,必须取$C_{2}=-C_{0}$,而$C_{0}$可由归一化条件
\eqshort
\begin{empheq}{equation*}
	\int_{-a}^{a}\varPsi^{*}\varPsi dx=1
\end{empheq}
定出,结果为
\begin{empheq}{equation}\label{eq63.13}
	\varPsi=\sqrt{\frac{15}{16a}}\bigg(1-\frac{x^{2}}{a^{2}}\bigg)
\end{empheq}\eqnormal
这样已经不再有安插变分参数的余地.由\eqref{eq63.13}式求得能量平均值为
\eqlong
\begin{empheq}{equation}\label{eq63.14}
	E=\int_{-a}^{a}\varPsi^{*}\bigg(-\frac{\hbar^{2}}{2m}\frac{d^{2}}{dx^{2}}\bigg)\varPsi dx=\frac{5}{4}\frac{\hbar^{2}}{ma^{2}}
\end{empheq}\eqnormal
这和$E_{0}$的精确值已经相当接近,事实上
\begin{empheq}{equation*}
	E/E_{0}=10/\pi^{2}\approx\num{1.0132}
\end{empheq}

如取\eqref{eq63.12}式中前三项作为基态试探波函数,为了满足边界条件,应取$C_{4}=-(C_{0}+C_{2})$.如以$C_{0}$作为归一化系数,则试探波函数可以表示成
\eqlong
\begin{empheq}{equation}\label{eq63.15}
	\varPsi(\lambda,x)=C_{0}\bigg[1+\lambda\bigg(\frac{x}{a}\bigg)^{2}-(1+\lambda)\bigg(\frac{x}{a}\bigg)^{4}\bigg]
\end{empheq}\eqnormal
由归一化条件求出
\begin{empheq}{equation}\label{eq63.16}
	\frac{16}{315}(\lambda^{2}+8\lambda+28)aC_{0}^{2}=1
\end{empheq}
而能量平均值为
\eqlong
\begin{empheq}{align}\label{eq63.17}
	E(\lambda)&=-\frac{\hbar^{2}}{2m}\int_{-a}^{a}\varPsi(\lambda,x)\frac{d^{2}}{dx^{2}}\varPsi(\lambda,x)dx	\nonumber\\
	&=\frac{3}{4}\frac{11\lambda^{2}+36\lambda+60}{\lambda^{2}+8\lambda+28}\frac{\hbar^{2}}{ma^{2}}
\end{empheq}\eqnormal
根据变分极值条件\eqref{eq63.10}式,求得$\lambda$的最佳值为
\begin{empheq}{equation*}
	\lambda_{0}=-\num{1.22075},\quad \lambda_{0}^{\prime}=-\num{8.31775}
\end{empheq}
代入\eqref{eq63.17}式,得到
\eqlong
\begin{empheq}{equation}\label{eq63.18}
	E(\lambda_{0})=\num{1.23372}\frac{\hbar^{2}}{ma^{2}},\quad E(\lambda_{0}^{\prime})=\num{12.7663}\frac{\hbar^{2}}{ma^{2}}
\end{empheq}\eqnormal
$E(\lambda_{0})$和基态能级的精确值非常接近,事实上
\begin{empheq}{equation*}
	E_{0}=\frac{\pi^{2}}{8}\frac{\hbar^{2}}{ma^{2}}\approx \num{1.23370}\frac{\hbar^{2}}{ma^{2}}
\end{empheq}
\pskip
$E(\lambda_{0}^{\prime})$则接近于第三个能级$\frac{9\pi^{2}\hbar^{2}}{8ma^{2}}$.

将$\lambda_{0}=-\num{1.22075}$代入\eqref{eq63.15}式,所得波函数$\varPsi(\lambda_{0},x)$的波形非常接近于真正的基态$\varPsi_{0}$,计算表明
\begin{empheq}{equation*}
	\int_{-a}^{a}\varPsi_{0}\varPsi(\lambda_{0},x)dx=1-\alpha,\quad \alpha<10^{-5}
\end{empheq}
亦即$\varPsi(\lambda_{0},x)$中基态为主要成分,各种激发态成分之和仅为
\begin{empheq}{equation*}
	1-(1-\alpha)^{2}=2\alpha-\alpha^{2}\sim 10^{-5}.
\end{empheq}









