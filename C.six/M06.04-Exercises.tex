\begin{exercises}
	
\exercise 某物理体系只有两个能级.$H_{0}$在表象中$H$的矩阵表示为
\begin{empheq}{equation*}
	H=H_{0}+H^{\prime}=\begin{bmatrix}
		E_{1}^{(0)}+a & b \\
		b E_{2}^{(0)}+a	\\
	\end{bmatrix}
\end{empheq}
$a,b$为实数,并且远小于$(E_{2}^{(0)}-E_{1}^{(0)})$.试求能级的精确值.再按照微扰论公式写出能级(二级近似).比较两种结果.

\exercise 粒子在势阱
\begin{empheq}{equation*}
	{V(x)=}
	\begin{dcases}
		V_{0}x/a,\quad 0<x<a	\\
		\infty,\qquad x\leqslant0,x\geqslant a
	\end{dcases}
\end{empheq}
中运动,$V_{0}\leqslant\dfrac{\hbar^{2}}{m^{2}a}$,试用微扰论计算其能谱(一级近似).

[提示:利用无限深平底势阱(2-5题)的结果.]

\exercise 粒子在无限深势阱$(-a<x<a)$中运动.如受到微扰$H^{\prime}=\gamma\delta(x)$作用,求各能级产生的变化(一级修正).微扰论成立的条件是什么?

\exercise 三维自由转子的能量算符为$H_{0}=\dfrac{\boldsymbol{L}^{2}}{2\boldsymbol{I}}$,$\boldsymbol{I}$为转动惯量.试确定其能级及简并度.设此转子受到微扰$H^{\prime}=\lambda\cos\theta$作用,求基态能级的变化(二级近似).

\exercise 一维谐振子,受到微扰$H^{\prime}=\dfrac{\lambda p}{m}$作用,试按照微扰论公式求能谱(准确到$\lambda^{2}$),与精确值(3-30题)比较.

\exercise 三维各向同性谐振子,受到微扰
\begin{empheq}{equation*}
	H^{\prime}=\lambda xyz+\bigg(\frac{\lambda^{2}}{\hbar\omega}\bigg)x^{2}y^{2}z^{2}
\end{empheq}
作用,求基态能级,准确到$\lambda^{2}$.

[提示:采用直角坐标系,$\varPsi_{n_{1}n_{2}n_{3}}=\varPsi_{n1}(x)+\varPsi_{n2}(y)+\varPsi_{n3}(z)$].

\exercise 在类氢离子(核电荷$Ze$)的能谱计算中,通常视原子核为点电荷.这样求出的能级记成$E_{n}^{(0)}$.实际上原子核更接近于电荷均匀分布的小球,核半径$R=r_{0}Z^{1/3},r_{0}=\num{1.635}\times10^{-13}\si{cm}$.试求这种核电荷分布效应对于离子中电子基态能级的影响(一级微扰修正).

$\bigg[$提示:电子所受库仑作用势为
\begin{empheq}{equation*}
	{V(r)}
	\begin{dcases}
		-\frac{Z\e^{2}}{R}\bigg(\frac{3}{2}-\frac{r^{2}}{2R^{2}}\bigg),\quad r<R	\\
		-\frac{Z\e^{2}}{r},\qquad\quad \qquad\quad  r>R
	\end{dcases}
\end{empheq}
以$H^{\prime}=V(r)-\bigg(-\dfrac{Z\e^{2}}{r}\bigg)$作为微扰.注意$R\ll\dfrac{a_{0}}{Z}$.
$\bigg]$
\exercise 在静电场中,如点电荷获得的静电势能是$V(\boldsymbol{r})$,则将点电荷换成电荷均匀分布的小球(半径$r_{0}$)时,静电势能是
\begin{empheq}{equation*}
	U(\boldsymbol{r})=V(\boldsymbol{r})+\frac{1}{6}r_{0}^{2}\nabla^{2}V(\boldsymbol{r})+\cdots
\end{empheq}
其中$\boldsymbol{r}$为球心位置对于氢原子,(原子核即质子,视为点电荷)视电子为点电核时,库仑势能为$V(\boldsymbol{r})=-\dfrac{\ell^{2}}{r}$.如视电子为电荷均匀分布的小球,$r_{0}=\dfrac{\e^{2}}{m_{e}C^{2}}$(经典电子半径),库仑势能改为上述$U(\boldsymbol{r})$,求1s和2p能级的一级微扰修正.[相当于兰姆移位(Lamb shift)]

[提示:$H^{\prime}=V-V,\nabla^{2}\dfrac{1}{r}=-4\pi\delta(\boldsymbol{r})$]

\exercise 有一个在磁场中的三维转子,能量算符为
\begin{empheq}{equation*}
	H=k\boldsymbol{L}^{2}+\omega L_{z}+\lambda L_{x},\quad k,\omega\gg\lambda
\end{empheq}

(a) 视$\lambda$项为微扰,求能级的零级近似、一级修正、二级修正.

(b) 求能级的精确值,并与微扰论结果比较.

[提示:找一个方向$\boldsymbol{n}$,使$\omega L_{z}+\lambda L_{x}=\omega^{\prime}L_{n}$,而$\boldsymbol{L}^{2},L_{n}$有共同本征态,$L_{n}$本征值为$m\hbar$.]

\exercise (a) 粒子在二维无限深方势阱$(0<x<a,0<y<a$中运动,写出能级与能量本征函数.

(b) 加上微扰$H^{\prime}=\lambda xy$,求最低两个能级的一级微扰修正.

\exercise 苯分子的“自由电子模型”认为电子是在一个环形势场中运动,并受到具有$\ce{C_{6}}$对称性的微扰作用.即

(a) 不计及微扰作用时,可以认为电子是在半径为$R$的环上自由运动.写出能量本征值与本征函数,作为零级近似.(参看2-6题)

(b) 微扰可以表示成$H^{\prime}=V(\varphi)=-V_{0}\cos 6\varphi$,试研究它对各能级的影响(一级修正),特别要找出发生分裂的能级.

\exercise 对于类氢离子的第二个能级,讨论其在微扰$H^{\prime}=xyf(r)$作用下的分裂情况(一级效应).$f(r)$有良好的积分收敛性,无其他特殊性质.

\exercise 某体系的能量算符$H_{0}$只有两个本征值$E_{1}^{(0)}$,$E_{2}^{(0)}$,前者二重简并,后者不简并.受微扰$H^{\prime}$作用后,能量算符的矩阵表示($H_{0}$表象)为
\begin{empheq}{equation*}
	H=H_{0}+H^{\prime}=\begin{bmatrix}
		E_{1}^{(0)} & 0 & a	\\
		0 & E_{1}^{(0)} & b	\\
		a & b & E_{2}^{(0)}	\\
	\end{bmatrix}
\end{empheq}
试用微扰论求能级(二级近似).再用矩阵方法求能级的精确公式,然后作近似展开$(a,b\ll E_{2}^{(0)}\sim E_{1}^{(0)})$,与微扰论结果比较.

\exercise (a) 如粒子(质量$\mu$)的波函数(未归一化)为
\begin{empheq}{equation*}
	\varPsi(\lambda,r)=e^{-\lambda r}\quad\text{或}\quad e^{-\lambda^{2}r^{2}/2}
\end{empheq}
求动能平均值.

(b) 以这两种函数分别作为三维各向同性谐振子的基态试探函数($\lambda$为变分参数),求基态能级近似值.

\exercise 粒子(质量$\mu$)在势场$V(x)=kx^{4}(k>0)$中运动,用变分法求基态能级近似值.试探波函数(未归一化)取为

(a) $\varPsi(\lambda,x)=e^{-\lambda|x|}$

(b) $\varPsi(\lambda,x)=e^{-\lambda^{2}r^{2}/2}$.解释(a)项结果较差的原因.

\exercise 粒子在无限深势阱$(-a<x<a)$中运动,试用多项式近似及变分法求第一激发能级的近似值,与精确值比较.

$\bigg[$提示:波函数为奇宇称,取$\varPsi=N\biggl\{\dfrac{x}{a}+\lambda\bigg(\dfrac{x}{a}\bigg)^{3}-(1+\lambda)\bigg(\dfrac{x}{a}\bigg)^{5}\biggr\}.$	$\bigg]$

\end{exercises}