\section[薛定谔方程]{薛定谔方程} \label{sec:02.01} % 
% \makebox[5em][s]{} % 短题目拉间距

众所周知,在真空中自由传播的电磁波满足波动方程
\begin{equation}\label{eq21.1}
	\nabla^{2}\varPsi-\frac{1}{c^{2}}\frac{\partial^{2}}{\partial t^{2}}\varPsi=0
\end{equation}
其中$\nabla$和$\nabla^{2}$为梯度算符和拉普拉斯算符:
\begin{equation}\label{eq21.2}
	\nabla=e_{1}\frac{\partial}{\partial x}+e_{2}\frac{\partial}{\partial y}+e_{3}\frac{\partial}{\partial z}
\end{equation}
\begin{equation}\label{eq21.3}
	\nabla^{2}=\nabla\cdot\nabla=\frac{\partial^{2}}{\partial x^{2}}+\frac{\partial^{2}}{\partial y^{2}}+\frac{\partial^{2}}{\partial z^{2}}
\end{equation}\eqshort
($x,y,z$)为空间直角坐标,$t$为时间,$\varPsi$是波函数.考虑\eqref{eq21.1}式的单色平面波特解
\begin{equation}\label{eq21.4}
	\varPsi_{k}=Ae^{i(k,r-\omega t)}
\end{equation}
$A$是振幅,$\omega$为角频率($\omega=2\pi\nu$,$\nu$为角频率),$\boldsymbol{k}$为波矢量,$\boldsymbol{k}$的方向即平面波的传播方向,$k=|\boldsymbol{k}|=\frac{2\pi}{\lambda}$,$\lambda$为波长.对于电磁波,$\nu$和$\lambda$间或$\omega$和$k$间有如下关系:
\begin{equation}\label{eq21.5}
	v\lambda=c,\quad \omega=ck
\end{equation}
电磁波就是光波,也就是光子.在没有光子概念之前,人们对波函数和波动方程的理解是,波函数$\varPsi_{h}$描写了单色平面光波的性质(振幅,频率,波长,传播方向,等等),波动方程全面描述了光波的运动(传播)规律.有了光子概念后,根据爱因斯坦-德布罗意关系式:
\begin{equation}\label{eq21.6}
	E=\hbar\omega,\quad \boldsymbol{p}=\hbar \boldsymbol{k}
\end{equation}
易见波函数$\varPsi_{h}$同时也描述了光子的力学性质(能量E,动量$\boldsymbol{p}$).光子的相对论力学基本公式是
\begin{equation}\label{eq21.7}
	E^{2}=c^{2}\boldsymbol{p}^{2}
\end{equation}\eqindent{5}
这个关系也已经体现在波动方程\eqref{eq21.1}中,具体来说对于波函数$\varPsi_{h}$,显然有下列关系:
\begin{equation}\label{eq21.8}
	\begin{aligned}
		&i\hbar\frac{\partial}{\partial t}\varPsi_{k}=\hbar\omega\varPsi_{h}=E\varPsi_{k} \\
		&\bigg(i\hbar\frac{\partial}{\partial t} \bigg)^{2}\varPsi_{k}=\hbar^{2}\omega^{2}\varPsi_{k}=E^{2}\varPsi_{k} \\
		&-i\hbar\nabla\varPsi_{k}=\hbar k\varPsi_{k}=\boldsymbol{p}\varPsi_{k} \\
		&\bigg(-i\hbar\nabla \bigg)\cdot\bigg(-i\hbar\nabla \bigg)\varPsi_{k}=
		-i\hbar^{2}\nabla^{2}\varPsi_{k}=\hbar^{2}k^{2}\varPsi_{k}=\boldsymbol{p}^{2}\varPsi_{k}
	\end{aligned}
\end{equation}\eqnormal
因此,将\eqref{eq21.4}式代入\eqref{eq21.1}式,即得
\begin{equation*}
	\bigg(\frac{\omega^{2}}{c^{2}}-k^{2} \bigg)\varPsi_{k}=
	\frac{1}{\hbar^{2}c^{2}}\bigg(E^{2}-c^{2}\boldsymbol{p}^{2}\bigg)\varPsi_{k}=0
\end{equation*}\eqindent{5}
这正是\eqref{eq21.7}式的具体体现.可见,就作用于波函数的效果来说,算符$i\hbar\frac{\partial}{\partial t}$代表光子的能量,$-\hbar\nabla$代表光子的动量.用符号“$\quad\hat{}\quad$”表示算符,我们可以写
\begin{equation}\label{eq21.9}
	\boxed{
	\begin{aligned}
	&\hat{E}=i\hbar \frac{\partial}{\partial t},\quad &&\hat{\boldsymbol{p}}=-i\hbar\nabla,\quad &&\hat{\boldsymbol{p}}^{2}=-\hbar^{2}\nabla^{2}		\\
	&\hat{\boldsymbol{p}}_{x}=-i\hbar\frac{\partial}{\partial x},\quad &&\hat{\boldsymbol{p}}_{y}=-i\hbar\frac{\partial}{\partial y},\quad 
	&&\hat{\boldsymbol{p}}_{z}=-i\hbar\frac{\partial}{\partial z}
	\end{aligned}
	}
\end{equation}\eqshort
在上述理解的基础上,如以光子的相对论力学公式\eqref{eq21.7}作为出发点,只要将各力学量换成相应的算符,并作用于波函数$\varPsi$,就得到光子的波动方程\eqref{eq21.1}.当然,对于光(电磁波)来说,用这种方法来建立波动方程实无必要,因为早在发现光子的力学关系\eqref{eq21.7}以前,已经先由经典电磁理论得到了波动方程[注意\eqref{eq21.1}式不含$\hbar$].但如将这种方法推广到电子,情况就不同了.

低速运动的自由电子,能量-动量关系为
\begin{equation}\label{eq21.10}
	E=\frac{\boldsymbol{p}^{2}}{2m}\quad(m\text{即}m_{e})
\end{equation}
因此,描述电子波动性的波函数应该满足波动方程:
\begin{equation*}
	\hat{E}\varPsi=\frac{1}{2m}\boldsymbol{p}^{2}\varPsi
\end{equation*}\eqnormal
亦即
\begin{equation}\label{eq21.11}
	i\hbar\frac{\partial}{\partial t}\varPsi=-\frac{\hbar^{2}}{2m}\nabla^{2}\varPsi
\end{equation}
电子在保守力场中运动时,基本力学关系为
\begin{equation}\label{eq21.12}
	E=T+V=\frac{\boldsymbol{p}^{2}}{2m}+V(x,y,z)
\end{equation}
$T=\frac{\boldsymbol{p}^{2}}{2m}$为动能,$V$为势能.相应的波动方程为
\begin{equation}\label{eq21.13}
	\hat{E}\varPsi=\frac{1}{2m}\boldsymbol{p}^{2}\varPsi+V\varPsi
\end{equation}
亦即
\begin{equation}\label{eq21.14}
	i\hbar\frac{\partial}{\partial t}\varPsi=-\frac{\hbar^{2}}{2m}\nabla^{2}\varPsi+V(x,y,z)\varPsi
\end{equation}\eqshort
这称为薛定谔方程,它是非相对论量子力学的基本方程.

薛定谔方程是德布罗意物质波思想在理论上的发展.薛定谔认识到,研究波动过程,首先要建立一个波动方程.他考察了波动光学和几何光学的联系,对比分析了几何光学和经典分析力学在数学表现形式上的相似性,终于找到了\eqref{eq21.14}式作为“波动力学”的基础.

薛定谔方程\eqref{eq21.14}是量子力学的一项基本假设,它的正确性已随着80年来量子力学在各方面的应用而得到确认.上述“建立”薛定谔方程的方法当然并不是这个方程的逻辑证明或严格导出.首先,从光子推广到电子,这本身就是一种假设性的做法.其次,在由\eqref{eq21.12}式得出\eqref{eq21.14}式的过程中,$E$和$\boldsymbol{p}$均按\eqref{eq21.9}式换成了相应的算符,而势能$V$却未作任何代换就将它直接乘在波函数上,这样做并无任何先验的理由,带有假设性.

在经典力学中,动能$T$与势能$V$之和称为哈密顿量(Hamiltonian),记作$H$.当势能$V$与时间无关时,$H$也就是总能$E$.\eqref{eq21.14}式常写成
\begin{equation}\label{eq21.15}
	i\hbar\frac{\partial}{\partial t}\varPsi=\hat{H}\varPsi
\end{equation}\eqnormal
$\hat{H}$是代表哈密顿量的算符:
\begin{equation}\label{eq21.16}
	\hat{H}=\hat{T}+V=-\frac{\hbar^{2}}{2m}\nabla^{2}+V(x,y,z)
\end{equation}\eqshort
带电粒子在磁场中运动时,其波动方程仍可写成\eqref{eq21.15}式的形式,但哈密顿算符$\hat{H}$不再由\eqref{eq21.14}式表示.这个问题将在$\S$\ref{sec:07.04}中讨论

量子力学的一项基本假设(也是一项基本概念)是,微观粒子(如电子)的运动状态可以用一个“波函数”来描述.这就是说,“波动性”一词的含义不应狭义地仅从经典物理的波动意义上去理解,而应广义地理解成泛指粒子运动状态的全部性质,波函数$\varPsi$就是这些性质的全面数学描述.波函数的时空变化规律由薛定谔方程\eqref{eq21.15}表示,它在量子力学中的地位相当于经典力学中的牛顿运动方程.波函数也称态函数.

由于薛定谔方程是时间$t$的一阶微分方程,如果知道了初始时刻$t_{0}$时的波函数$\varPsi(\boldsymbol{p},t)$,原则上可以求出各个$t>t_{0}$时刻的波函数$\varPsi(\boldsymbol{p},t)$.亦即,知道了$t_{0}$时刻粒子的运动状态,利用薛定谔方程就可以求出$t>t_{0}$时刻粒子的运动状态.这是微观粒子因果性的表现方式.

现在来研究\eqref{eq21.15}式的一种分离变量函数形式的特解:
\begin{equation}\label{eq21.17}
	\boldsymbol{\varPsi}(\boldsymbol{r},t)=\varPsi(\boldsymbol{r}f(t))
\end{equation}\eqnormal
代入\eqref{eq21.15}式,得到
\begin{equation*}
	i\hbar\frac{\partial f(t)}{\partial t}\varPsi(r)=f(t)\hat{H}\varPsi(r)
\end{equation*}
以$\varPsi(r)f(t)$除上式,得到
\begin{equation*}
	i\hbar\frac{df}{dt}\big/ f(t)=[\hat{H}\varPsi(r)]/ \varPsi(r)
\end{equation*}
左端是$t$的函数,右端是$\boldsymbol{r}$的函数,只有它们等于同一个常数(记为$\hbar\omega$),上式才能成立,因此
\begin{subnumcases}{}
	i\hbar\frac{df}{dt}=\hbar\omega f(t) \label{eq21.18a}\\
	\hat{H}\varPsi(r)=\hbar\omega\varPsi(r) 	\label{eq21.18b}
\end{subnumcases}\label{eq21.18}

由\eqref{eq21.18a}式容易解出
\begin{equation*}
	f(t)=e^{-i\omega t}=e^{-iEt/h},\quad E=\hbar\omega
\end{equation*}
参数$E=\hbar\omega$的意义显然就是能量.\eqref{eq21.18b}式亦即
\begin{equation}\label{eq21.19}
	\hat{H}\varPsi(r)=E\varPsi(r)
\end{equation}
与某个$E$值相应的$\varPsi(r)$可以记作$\varPsi_{E}(r)$,而特解\eqref{eq21.17}则记作
\begin{equation}\label{eq21.20}
	\varPsi_{E}(\boldsymbol{r},t)=\varPsi_{E}(\boldsymbol{r})e^{-iEt/\hbar}
\end{equation}\eqlong
这是能量具有确定值的状态的波函数,它随时间变化的函数形式已经完全确定,即$e^{-iEt/h}$,它的空间部分$\varPsi_{E}(\boldsymbol{r})$需由\eqref{eq21.19}式解出.$\varPsi_{E}(\boldsymbol{r},t)$称为定态波函数,\eqref{eq21.19}式称为定态薛定谔方程.而\eqref{eq21.14}或\eqref{eq21.15}式则称为含时间的薛定谔方程.

在量子力学中,能量$E$的可能取值由求解定态薛定谔方程来决定.在有些问题中,仅当$E$取某些特定的值时\eqref{eq21.19}式才存在合乎物理要求的解,这些特定的$E$值就是量子化能级.在有些问题中,$E$的取值可在一定范围(例如$E>0$) 内取任意值,\eqref{eq21.19}式均有合乎物理要求的解,这种能量值就是连续能谱.

含时间薛定谔方程\eqref{eq21.15}是线性微分方程,它的通解可以表示为全体特解的线性叠加,即
\begin{equation}\label{eq21.21}
	\varPsi(\boldsymbol{r},t)=\sum_{n}C_{n}\varPsi_{E_{n}}(\boldsymbol{r},t)
	=\sum_{n}C_{n}\varPsi_{E_{n}}(\boldsymbol{r})e^{-iE_{n}t/\hbar}
\end{equation}\eqshort
$C_{n}$为待定系数.因此,求解定态薛定谔方程\eqref{eq21.19}是解含时间薛定谔方程\eqref{eq21.15}的基础.

\example 克莱因-戈登(Klein-Gordon)方程

按照相对论力学,质量为$m_{0}$的粒子自由运动时,能量-动量关系为
\begin{equation}\label{eq21.22}
	E^{2}=c^{2}\boldsymbol{p}^{2}+m_{0}^{2}c^{4}
\end{equation}\eqnormal
因此,相对论波动方程为
\begin{equation}\label{eq21.23}
	(c^{2}\hat{\boldsymbol{p}}^{2}+m_{0}^{2}c^{4}-\hat{E}^{2} )\varPsi=0
\end{equation}
其中算符$\hat{E},\hat{\boldsymbol{p}}$由\eqref{eq21.9}式表示.将\eqref{eq21.23}式写成明显的微分方程,就是
\begin{equation}\label{eq21.24}
	\nabla^{2}\varPsi-\frac{1}{c^{2}}\frac{\partial^{2}}{\partial t^{2}}\varPsi-\frac{m_{0}^{2}c^{2}}{\hbar^{2}}\varPsi=0
\end{equation}
称为克莱因-戈登方程,它比电磁波的波动方程\eqref{eq21.1}多了一个质量项.这个方程不能保证粒子数守恒,它可以描述自旋为0的粒子(玻色子,例如$\pi^{0}$介子)的运动.

克莱因-戈登方程方程也存在单色平面波特解,波函数仍可表示成\eqref{eq21.4}式,它描述粒子的自由运动,相应于下列力学量取值:
\begin{equation}\label{eq21.25}
	\boldsymbol{p}=\hbar\boldsymbol{k},\quad E=\hbar\omega=(\hbar^{2}c^{2}k^{2})^{\frac{1}{2}}
\end{equation}
如低速运动,$\hbar k \ll m_{0}c$,则$E$的近似值为
\begin{equation}\label{eq21.26}
	\begin{aligned}
		E &=m_{0}c^{2} \bigg(1+\frac{\hbar^{2}k^{2}}{m_{0}c^{2}} \bigg)^{\frac{1}{2}}	\\
		  &\approx m_{0}c^{2} \bigg(1+\frac{\hbar^{2}k^{2}}{2m_{0}c^{2}} \bigg) \\
		  &=m_{0}c^{2}+\frac{p^{2}}{2m_{0}}
	\end{aligned}
\end{equation}