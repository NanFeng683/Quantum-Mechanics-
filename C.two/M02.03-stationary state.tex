\section[定态]{\makebox[5em][s]{定态}} \label{sec:02.03} % 
% \makebox[5em][s]{} % 短题目拉间距

本节讨论定态波函数的一般数学性质,而以一维束缚定态作为讨论的重点.

在$\S$\ref{sec:02.01}中已经讲过,具有确定能量值的状态称为定态,定态波函数的时空变量可以分离,一般形式为
\begin{equation}\label{eq23.1}
	\varPsi_{E}(\boldsymbol{r},t)=\varPsi_{E}(\boldsymbol{r})e^{-iEt/\hbar}
\end{equation}
其中$\varPsi_{E}(\boldsymbol{r})$满足定态薛定谔方程:
\begin{equation}\label{eq23.2}
	\hat{H}\varPsi(\boldsymbol{r})=E\varPsi(r)
\end{equation}
本节只讨论哈密顿量$H$等于动能加势能的情形,即
\begin{equation}\label{eq23.3}
	\hat{H}=\hat{T}+V=-\frac{\hbar^{2}}{2m}\nabla^{2}+V(r)
\end{equation}
这时定态薛定谔方程为
\begin{equation}\label{eq23.4}
	-\frac{\hbar^{2}}{2m}\nabla^{2}\varPsi+V(r)\varPsi=E\varPsi
\end{equation}
对于一维问题,
\begin{equation}\label{eq23.5}
	\hat{H}=\hat{T}+V=-\frac{\hbar^{2}}{2m}\frac{d^{2}}{dx^{2}}+V(x)
\end{equation}
定态薛定谔方程为
\begin{equation}\label{eq23.6}
	-\frac{\hbar^{2}}{2m}\frac{d^{2}}{dx^{2}}\varPsi+V(x)\varPsi=E\varPsi
\end{equation}\eqshort

当粒子处于定态时,将\eqref{eq23.1}式代入\eqref{eq22.8}、\eqref{eq22.9}式,易见概率密度$\rho$和概率流密度$\boldsymbol{j}$均与$t$无关.所以定态是稳定状态,其物理性质不随时间变化.

如果定态波函数能够满足归一化条件:
\begin{equation}\label{eq23.7}
	\int_{\text{全}}\varPsi^{*}\varPsi d\tau=1
\end{equation}\eqnormal
则在无限远处$\varPsi$必然迅速趋于0,粒子实际上是在有限范围内运动,这种状态称为束缚态.以后凡谈到束缚态,一律理解成其波函数已经满足归一化条件.

对于束缚定态,以$\varPsi^{*}$左乘\eqref{eq23.4}式各项,并对全空间积分,得到
\begin{equation}\label{eq23.8}
	E=\int_{\text{全}}V\varPsi^{*}\varPsi d\tau-\frac{\hbar^{2}}{2m}\int_{\text{全}}\varPsi^{*}\nabla^{2}\varPsi d\tau
\end{equation}
由于粒子可以在空间各不同地点出现,势能$V(r)$的值随粒子的位置$r$而变,上式右端第1项显然就是势能的平均值,记为$\bar{V}$,
\begin{equation}\label{eq23.9}
	\bar{V}=\int_{\text{全}}V(r)\varPsi^{*}(r)\varPsi(r)d\tau
\end{equation}
\eqref{eq23.8}式右端第二项当然就是动能的平均值,记为$\bar{T}$,即
\setlength{\mathindent}{4em}
\begin{equation}\label{eq23.10}
	\begin{aligned} 
		\bar{T} 
		&=\int_{\text{全}}\varPsi^{*}\frac{\hat{\boldsymbol{p}}^{2}}{2m}\varPsi d\tau
		 =-\frac{\hbar^{2}}{2m}\int_{\text{全}}\varPsi^{*}\nabla^{2}\varPsi d\tau	\\
		&=\frac{\hbar^{2}}{2m}\int_{\text{全}}(\nabla\varPsi^{*})\cdot(\nabla\varPsi)d\tau
		 =\frac{\hbar^{2}}{2m}\int_{\text{全}}|\nabla\varPsi|^{2} d\tau >0
	\end{aligned}
\end{equation}\eqshort
计算中利用了无限远处$\varPsi\rightarrow 0$的条件以及高斯定理.\eqref{eq23.8}式即
\begin{equation*}\label{eq23.8'}
	E=\bar{T}+\bar{V} \tag{$2.3.8^{\prime}$}
\end{equation*}\eqllong
由于动能平均值总是大于0,所以$E$总是大于势能平均值.

在$\S$\ref{sec:03.04}中将给出其他物理狱的平均值计算公式.

由于必须满足归一化条件,$\varPsi(r)$一般只能取有限值.但如存在$V(r)\rightarrow-\infty$的孤立奇点,在该点$\varPsi$可能趋于$\infty$.在$V(r)\rightarrow \infty$处,$\varPsi$必须趋于0,以保证$V$为有限值($V<E$).

对于一维方程\eqref{eq23.6},任取一点$x_{0}$及其无限小邻区[$x_{0}-\varepsilon,x_{0}+\varepsilon$],在此邻区内将方程积分,得
\begin{empheq}{equation*}
	\begin{aligned}
		\frac{2m}{\hbar^{2}}\int_{x_{0}-\varepsilon}^{x_{0}+\varepsilon}[V(x)-E]\varPsi(x)dx
		&=\int_{x_{0}-\varepsilon}^{x_{0}+\varepsilon} \varPsi^{\prime\prime}(x)dx \\
		&=\varPsi^{\prime}(x_{0}+\varepsilon)-\varPsi^{\prime}(x_{0}-\varepsilon)
	\end{aligned}
\end{empheq}\eqnormal
$\varPsi^{\prime\prime}$和$\varPsi^{\prime}$即$\frac{d^{2}\varPsi}{dx^{2}}$和$\frac{d\varPsi}{dx}$.如在$x_{0}$及其邻区内$V(x)$取有限值,则当$\varepsilon\rightarrow 0$时上式左端显然趋于0,因此$\varPsi^{\prime}(x_{0}+\varepsilon)=\varPsi^{\prime}(x_{0}-\varepsilon)$,亦即在$x_{0}$附近$\varPsi^{\prime}$连续.但如果$x\rightarrow x_{0}$时$V\rightarrow \pm\infty$,则当$\varepsilon\rightarrow 0$时上式左端就不可能趋于0,因此$\varPsi^{\prime}(x_{0}-\varepsilon)$和$\varPsi^{\prime}(x_{0}+\varepsilon)$可能不相等,亦即在$x_{0}$处$\varPsi^{\prime}$可能不连续,应作具体考虑.由于
\begin{equation*}
	\int_{x_{0}-\varepsilon}^{x_{0}+\varepsilon}\varPsi^{\prime}(x)dx
	=\varPsi(x_{0}+\varepsilon)-\varPsi(x_{0}-\varepsilon)
\end{equation*}
如在$x_{0}$及其邻区内$\varPsi^{\prime}(x)$取有限值[包括$\varPsi^{\prime}(x_{0}+\varepsilon)\neq\varPsi^{\prime}(x_{0}-\varepsilon)$的情况],当时上式左端显然趋于0,因此在$x_{0}$处$\varPsi$连续.

总起来说,关于$\varPsi(x)$及$\varPsi^{\prime}(x)$的连续性,有如下结论:

在$V(x)$取有限值的区域内,$\varPsi$及$\varPsi^{\prime}$均为连续函数,并取有限值;$V(x)\rightarrow\infty$处,$\varPsi(x)\rightarrow 0$,$\varPsi^{\prime}$有可能不连续;$V(x)\rightarrow-\infty$处,$\varPsi$有可能趋于$\infty$,$\varPsi^{\prime}$有可能不连续.

下面证明几个关于定态薛定谔方程的定理,它们对于具体问题的求解有指导意义.

(1) 如果$\varPsi=u+iv$($u,v$为实函数是\eqref{eq23.4}式或\eqref{eq23.6}式对应于某个能量特征值$E$(实数)的解,则$\varPsi$的实部$u$和虚部$v$都是方程的解(对应同样的能量值).

\prove 以三维方程为例,将$\varPsi=u+iv$($u,v$代入\eqref{eq23.6}式,得到
\begin{equation*}
	-\frac{\hbar^{2}}{2m}\nabla^{2}(u+iv)+V\cdot(u+iv)=E(u+iv)
\end{equation*}
两端分别取实部和虚部,即得
\begin{empheq}{equation*}
	\begin{aligned} 
		&& -\frac{\hbar^{2}}{2m}\nabla^{2}u+Vu=Eu \\
		&& -\frac{\hbar^{2}}{2m}\nabla^{2}v+Vv=Ev
	\end{aligned}
\end{empheq}
这正是所要证明的结果.根据这个定理,必要时我们可以全部选取实函数作为定态薛定谔方程的特征解.

(2) 对于一维方程\eqref{eq23.6},如$\varPsi_{1}$和$\varPsi_{2}$也是某个能量特征值$E$的两个线性独立解,则
\begin{equation}\label{eq23.11}
	\varPsi_{2}(x)\varPsi_{2}^{\prime}-\varPsi_{2}\varPsi_{1}^{\prime}(x)=C\text{(常数)}
\end{equation}
\prove $varPsi_{2}$和$\varPsi_{1}$分别满足\eqref{eq23.6}式,即
\begin{empheq}{equation*}
	\begin{aligned} 
		&& \varPsi_{1}^{\prime\prime}=\frac{2m}{\hbar^{2}}(V-E)\varPsi_{1} \\
		&& \varPsi_{2}^{\prime\prime}=\frac{2m}{\hbar^{2}}(V-E)\varPsi_{2}
	\end{aligned}
\end{empheq}
用$\varPsi_{2}$分别乘第一式、第二式,相减,即得
\begin{equation*}
	\varPsi_{2}\varPsi_{2}^{\prime\prime}-\varPsi_{2}\varPsi_{1}^{\prime\prime}
	=\frac{d}{dx}(\varPsi_{1}\varPsi_{2}^{\prime}-\varPsi_{2}\varPsi_{1}^{\prime})=0
\end{equation*}
再积分,即得\eqref{eq23.11}式,$C$为积分常数.

(3) 对于一维方程\eqref{eq23.6},与任何一个能盘特征值相应的线性独立解最多有两个,即每个能级最多有两个简并态.

\prove 用反证法,设相应于能量特征值$E$存在3个线性独立解$\varPsi_{1},\varPsi_{2},\varPsi_{3}$.根据定理\eqref{eq23.2},有
\begin{equation}
	\begin{aligned} \notag
		&& \varPsi_{1}\varPsi_{2}^{\prime}-=\varPsi_{2}\varPsi_{1}^{\prime}=C_{1} \\
		&& \varPsi_{1}\varPsi_{3}^{\prime}-=\varPsi_{3}\varPsi_{1}^{\prime}=C_{2}
	\end{aligned}
\end{equation}
用$C_{1}$乘第一式,$C_{2}$乘第二式,相减,得到
\begin{equation*}
	\varPsi_{1}(C_{2}\varPsi_{2}^{\prime}-C_{1}\varPsi_{3}^{\prime})
	-(C_{2}\varPsi_{2}-C_{1}\varPsi_{3})\varPsi_{1}^{\prime}=0
\end{equation*}\eqshort
令$\varphi=C_{3}\varPsi_{2}-C_{1}\varPsi_{3}$,上式即
\begin{equation*}
	\varPsi_{1}\varphi^{\prime}-\varphi\varPsi_{1}^{\prime}=0
\end{equation*}\eqnormal
等价于
\begin{equation*}
	\varPsi_{1}^{2}\bigg(\frac{\varphi}{\varPsi_{1}}\bigg)^{\prime}=0,\quad\text{或}\quad
	\varphi^{2}\bigg(\frac{\varPsi_{1}}{\varphi}\bigg)^{\prime}=0
\end{equation*}
因此,
\begin{equation*}
	\bigg(\frac{\varphi}{\varPsi_{1}}\bigg)^{\prime}=0,\quad \varphi=C\varPsi_{1}\quad \text{C为积分常数}
\end{equation*}\eqshort
亦即
\begin{equation*}
	C\varPsi_{1}=C_{2}\varPsi_{2}-C_{3}\varPsi_{2}
\end{equation*}
如在有限区域内$V(x)$取有限值,则$\varPsi_{1},\varPsi_{2},\varPsi_{3}$均为连续函数,因此上式在全空间成立,亦即$\varPsi_{1},\varPsi_{2},\varPsi_{3}$是线性相关的.这个结论和开始时三者线性独立的假设矛盾.故知每个能级的线性独立解最多只有两个.证明完毕.

(4) 对于一维束缚态,所有能级都是非简并的,波函数为实函数.

\prove 仍用反证法,设$\varPsi_{1},\varPsi_{2}$为能级$E$的两个线性独立解,按照定理\eqref{eq23.2},\eqref{eq23.11}式成立. 令$x\rightarrow\infty$,由于束缚态波函数满足归一化条件,$x\rightarrow\infty$处$\varPsi\rightarrow 0$,所以\eqref{eq23.10}式中$C=0$,于是
\begin{equation*}
	\varPsi_{1}\varPsi_{2}^{\prime}-\varPsi_{2}\varPsi_{1}^{\prime}=0
\end{equation*}\eqnormal
亦即
\begin{equation*}
	\varPsi_{1}^{2}\bigg(\frac{\varPsi_{2}}{\varPsi_{1}}\bigg)^{\prime}=0,\quad\text{或}\quad
	\varPsi_{2}^{2}\bigg(\frac{\varPsi_{1}}{\varPsi_{2}}\bigg)^{\prime}
\end{equation*}
以下仿照定理\eqref{eq23.3}的证明过程,可得$\varPsi_{2}=C^{\prime}\varPsi_{2}$,与二者线性独立的假设矛盾. 故知每个束缚能级只有一个线性独立的波函数.再利用定理\eqref{eq23.1},可知这个波函数必为实函数(当然,归一化系数中仍可包含一个任意的相因子$e^{i\alpha}$,$\alpha$为实数.)

物理现象常有各种对称性,当$\boldsymbol{r}$换成$-\boldsymbol{r}$,($x\rightarrow-x,y\rightarrow-y,z\rightarrow-z$)如果
\begin{equation}
	\begin{aligned} \notag
	&\varPsi(-\boldsymbol{r})=\varPsi(\boldsymbol{r}),&\text{则称}\varPsi\text{为偶宇称};
		\\
	&\varPsi(-\boldsymbol{r})=-\varPsi(\boldsymbol{r}),&\text{则称}\varPsi\text{为奇宇称}.
	\end{aligned}
\end{equation}
对于算符,也同样称谓.例如$\hat{\boldsymbol{p}}_{x}=-i\hbar\frac{\partial}{\partial x}$,为奇宇称;$\hat{T}=-\frac{\hbar^{2}}{2m}\nabla^{2}$,为偶宇称. 如势能$V$为偶宇称,即
\begin{equation*}
	V(\boldsymbol{r})=V(-\boldsymbol{r}),\quad\text{或}\quad V(x)=V(-x)
\end{equation*}
则哈密顿算符$\hat{H}$为偶宇称.否则,$\hat{H}$就没有确定的宇称性.

(5) 对于一维束缚定态,如果$V(x)$为偶宇称,则每一个$\varPsi_{E}(x)$都有明确的宇称性.

\prove 作为能量本征态,$\varPsi_{E}(x)$满足\eqref{eq23.6}式,
\begin{equation*}
	-\frac{\hbar^{2}}{2m}\frac{d^{2}}{dx^{2}}\varPsi_{E}(x)+V(x)\varPsi_{E}(x)=E\varPsi_{E}(x)
\end{equation*}
上式中$x$换成$-x$,成为[注意$V(x)=V(-x)$]
\begin{equation*}
	-\frac{\hbar^{2}}{2m}\frac{d^{2}}{dx^{2}}\varPsi_{E}(-x)+V(x)\varPsi_{E}(-x)=E\varPsi_{E}(-x)
\end{equation*}\eqshort
因此$\varPsi_{E}(-x)$也是\eqref{eq23.6}式的解,对应于同一个能量值$E$.但按照定理\eqref{eq23.4},束缚态能级不简并,因此$\varPsi_{E}(x)$和$\varPsi_{E}(-x)$代表同一个束缚态,二者只能相差一个常系数,即
\begin{equation*}
	\varPsi_{E}(-x)=C\varPsi_{E}(x)
\end{equation*}\eqnormal
令$x\rightarrow-x$,由上式可得
\begin{equation*}
	\varPsi_{E}(x)=C\varPsi_{E}(-x)=C^{2}\varPsi_{E}(x)
\end{equation*}\eqshort
因此
\begin{equation*}
	C^{2}=1,\quad C= \pm 1
\end{equation*}\eqnormal

如$C=1$,$\varPsi_{E}(-x)=\varPsi_{E}(x)$,为偶宇称.

如$C=-1$,$\varPsi_{E}(-x)=-\varPsi_{E}(x)$,为奇宇称

由此易知,当$V(x)$为偶宇称,如两个束缚态波函数具有不同的宇称,则它们必定分别属于不同的能级.

对于非束缚态的一维定态(游离态),在$V(x)$为偶宇称的条件下,由于能级可能是2重简并的[定理\eqref{eq23.3}],所以虽然$\varPsi_{E}(x)$和$\varPsi_{E}(-x)$都满足\eqref{eq23.6}式,它们仍然可能是线性独立的,这时可以将它们组合成两个具有相反宇称的波函数:
\eqlong
\begin{equation} \label{eq23.12}
	\begin{aligned} 
		&& \varPsi_{E+}(x)=\varPsi_{E}(x)+\varPsi_{E}(-x), \text{偶宇称}	\\
		&& \varPsi_{E-}(x)=\varPsi_{E}(x)-\varPsi_{E}(-x), \text{奇宇称}
	\end{aligned}
\end{equation}\eqshort
类似的结论也适用于3维定态(包括束缚态和游离态),读者试自证明之.

有明确宇称性的波函数便于运算(例如可以简化积分计算),因此当$V(r)$为偶宇称时,总是尽量选择具有宇称性的函数作为薛定谔方程的独立解.

\example 粒子的一维自由运动.

\solution 薛定谔方程为
\begin{equation*}
	i\hbar\frac{\partial}{\partial t}\varPsi
	=-\frac{\hbar^{2}}{2m}\frac{\partial^{2}}{\partial x^{2}}\varPsi
\end{equation*}
定态特解可以表示成
\begin{equation*}
	\varPsi_{E}(x,t)\approx \varPsi(x)e^{-iEt/h}
\end{equation*}
$\varPsi(x)$满足定态薛定谔
\begin{equation*}
	-\frac{\hbar^{2}}{2m}\frac{\partial^{2}}{\partial x^{2}}\varPsi=E\varPsi,\quad E\geqslant0
\end{equation*}
令
\begin{equation}\label{eq23.13}
	E=\frac{\hbar^{2}k^{2}}{2m}
\end{equation}
方程可以简化成
\begin{equation*}
	\varPsi^{\prime\prime}+k^{2}\varPsi=0
\end{equation*}\eqnormal
其特解为
\begin{equation}\label{eq23.14}
	\varPsi_{k}(x)=-\frac{1}{\sqrt{2\pi\hbar}}e^{ikx},\quad -\infty<k<\infty
\end{equation}\eqlong
$\bigg($系数$-\frac{1}{\sqrt{2\pi\hbar}}$的选取理由将在$\S$\ref{sec:03.05}中说明$\bigg)$$k$可取一切实数值,$\varPsi_{k}(x)$均为全空间的连续有限函数,所以能量$E$的取值可以连续变化,能谱是连续的,定态波函数可以表示成
\begin{equation}\label{eq23.15}
	\varPsi_{k}(x,t)=\varPsi_{k}(x)e^{-iEt/h}=-\frac{1}{\sqrt{2\pi\hbar}}e^{i(kx-Et/h)}
\end{equation}
除$E=0(k=0)$外,全部能级都是2重简并的,相应于$k=\pm\frac{\sqrt{2mE}}{\hbar}$,给定$E$后,$k$有正、负两种取值,相应于两种动量值$(p=\hbar k)$.

为了突出能级的2重简并,也可将2个简并态波函数记成
\begin{equation*}\label{eq23.14'}
	\varPsi_{k}(x)=\frac{1}{\sqrt{2\pi\hbar}}e^{ikx},\varPsi_{-k}(x)=\frac{1}{\sqrt{2\pi\hbar}}e^{-ikx} \tag{$2.3.14^{\prime}$}
\end{equation*}\eqnormal
其中$k$取正值,即
\begin{equation*}\label{eq23.13'}
	k=\frac{\sqrt{2mE}}{\hbar},\quad 0\leqslant k <\infty \tag{$2.3.13^{\prime}$}
\end{equation*}\eqlong
对于$\varPsi_{k}$,粒子动量为$(p=\hbar k)$;对于$\varPsi_{-k}$,动量为$(p=-\hbar k)$.

波函数\eqref{eq23.14'}式没有宇称性,但可以将它们组合成具有宇称性的函数:[注意$\varPsi_{-k}(x)=\varPsi_{k}(-x)$]
\begin{equation} \label{eq23.16}
	\begin{aligned} 
	&& \varPsi_{E+}(x)=\varPsi_{k}(x)+\varPsi_{-k}(-x)=\frac{2}{\sqrt{2\pi\hbar}}\cos kx	\\
	&& \varPsi_{E-}(x)=\varPsi_{k}(x)-\varPsi_{-k}(-x)=\frac{2i}{\sqrt{2\pi\hbar}}\sin kx
	\end{aligned}
\end{equation}\eqnormal
本例可以作为上述普遍定理的具体例证.读者可以利用\eqref{eq23.14'}或\eqref{eq23.16}式验证定理\eqref{eq23.2}的正确性.











