\begin{exercises}

\exercise 两个质量为$m$的全同粒子,在一维弹性力场中运动,$V_{1}=\dfrac{kx_{1}^{2}}{2},V_{2}=\dfrac{kx_{2}^{2}}{2}$,两粒子间的相互作用可以忽略.

(a) 写出体系的总能量算符,分别用$(x_{1},x_{2})$表示和$(X,x)$表示.($X$是质心坐标,$x$是相对坐标)讨论质心运动和相对运动的性质.

(b) 写出单粒子运动的基态与第一激发态波函数.($\varPsi_{0}$与$\varPsi_{1}$,略去归一化常数)如一个粒子处于基态,一个粒子处于第一激发态,写出对称的和反对称的体系轨道波函数,先用$(x_{1},x_{2})$表示,再用$(X,x)$表示,解释其性质.
\pskip
\exercise 两个质量为$m$的非全同粒子,受到外场作用,还有相互作用,作用势为
\eqllong
\begin{empheq}{equation*}
	V_{1}=\frac{\alpha}{2}x_{1}^{2},\quad V_{2}=\frac{\alpha}{2}x_{2}^{2},\quad V_{12}=\frac{\beta}{2}(x_{1}^{2}-x_{2}^{2})\quad (\alpha,\beta>0)
\end{empheq}\eqnormal

求体系的能谱.
\pskip
\exercise 某体系由3个自旋为0的全同粒子组成,限定单粒子状态只能是$\varPsi_{\alpha},\varPsi_{\beta},\varPsi_{\gamma}$,试写出体系的所有可能状态的波函数.

\exercise 两个自旋为0的全同粒子在无限深势阱$(0<x_{i}<a,i=1,2)$中运动,忽略粒子间相互作用.写出这个二粒子体系的基态与第一激发态波函数,并对基态计算$\bar{x},\Delta x,\bar{X},\Delta X$.

[提示:先计算$\bar{x_{1}},\bar{x_{2}},\bar{x_{1}^{2}},\bar{x_{2}^{2}}$.]
\pskip
\exercise 两个质量相同的粒子在无限深势阱$(0<x_{i}<a,i=1,2)$中运动.粒子间相互作用势$V_{12}=aV_{0}\delta(x_{1}-x_{2}),V_{0}\ll\dfrac{\hbar^{2}}{ma^{2}}$,因此可以视之为微扰.求这二粒子体系最低的3个能级.(不考虑粒子是否全同.)
\pskip
\exercise 类氦离子由原子核(电荷$Ze,Z>2$)及两个电子构成.如略去核运动,二电子体系总能量算符可以写成
\begin{empheq}{equation*}
	H=-\frac{\hbar^{2}}{2m_{e}}(\nabla_{1}^{2}+\nabla_{2}^{2})-Z\e^{2}\left(\frac{1}{r_{1}}+\frac{1}{r_{2}}\right)+\frac{\e^{2}}{r_{12}}
\end{empheq}\eqllong

(a) 视$\dfrac{\e^{2}}{r_{12}}$为微扰,求基态能级的粗略近似.

(b) 用变分法求基态能级近似值.
\pskip
\exercise 有-种简化的“一维类氦离子”模型,原子核-电子和电子-电子作用势均用“接触作用”表示.略去核运动后,电子体系总能量算符表示成
\begin{empheq}{equation*}
	H=-\frac{\hbar^{2}}{2m_{e}}\left(\frac{\partial^{2}}{\partial x_{1}^{2}}+\frac{\partial^{2}}{\partial x_{2}^{2}}\right)-Z\e^{2}[\delta(x_{1})+\delta(x_{2})]+\e^{2}\delta(x_{1}-x_{2})
\end{empheq}\eqlong

如距离以$\dfrac{a_{0}}{Z}$为单位,能量以$\dfrac{Z^{2}\e^{2}}{a_{0}}$为单位,$H$可以简化成
\begin{empheq}{equation*}
	H=-\frac{1}{2}\left(\frac{\partial^{2}}{\partial x_{1}^{2}}+\frac{\partial^{2}}{\partial x_{2}^{2}}\right)-\delta(x_{1})-\delta(x_{2})+\frac{1}{Z}\delta(x_{1}-x_{2})
\end{empheq}

(a) 视电子-电子作用势为微扰,求体系束缚态能级(只有一个).

(b) 用变分法求体系能级.

你将发现本题结果与10-6题结果惊人地接近.
\pskip
\exercise 某个三电子体系,单电子轨道态为$\varPsi_{a},\varPsi_{b},\varPsi_{c}$.当体系总自旋量子数$S=\dfrac{3}{2}$时,写出体系的轨道波函数,求相互作用能$\left(\dfrac{\e^{2}}{r_{12}}+\dfrac{\e^{2}}{r_{23}}+\dfrac{\e^{2}}{r_{31}}\text{的平均值}\right)$,将其表示成“库仑能”和“交换能”.
\pskip
\exercise 由两个全同粒子组成的体系,如单粒子自旋量子数为S,$[\boldsymbol{S}^{2}=S(S+1)\hbar^{2}]$体系总自旋态有多少种独立的交换对称态、反对称态?
\pskip
\exercise (a) 如原子中的电子换成某种电荷为$(-e)$,自旋量子数为$\dfrac{3}{2}$的粒子,求前三种惰性气体的原子序数($Z$,核中质子数)

(b) 如原子中的电子换成某种电荷为$(-\dfrac{e}{3})$,自旋量子数为$\dfrac{1}{2}$的费密子,求前三种惰性气体的原子序数.
\pskip
\exercise 两电子体系,定义“自旋交换算符”$P_{12},P_{12}\chi(S_{1z},S_{2z})=\chi(S_{2z},S_{1z})$,证明:$P_{12}=\dfrac{1}{2}(1+\boldsymbol{\sigma}_{1}\cdot\boldsymbol{\sigma}_{2})=\boldsymbol{S}^{2}-1$.($\boldsymbol{S}$为总自旋,取$\hbar=1$)
\pskip
\exercise 设粒子1,2自旋量子数均为$\dfrac{1}{2}$,以$\boldsymbol{S}=\boldsymbol{S}_{1}+\boldsymbol{S}_{2}$表示总自旋算符,取$\hbar=1$.证明以下算符关系:

\begin{empheq}{alignat*=2}
	[\boldsymbol{S}_{1}\cdot\boldsymbol{S}_{2},\boldsymbol{S}_{1}]=i\boldsymbol{S}_{1}\times\boldsymbol{S}_{2}, &\quad [\boldsymbol{S}_{1}\cdot\boldsymbol{S}_{2},\boldsymbol{S}_{2}]=-i\boldsymbol{S}_{1}\times\boldsymbol{S}_{2}	\\
	[\boldsymbol{S}_{1}\cdot\boldsymbol{S}_{2},\boldsymbol{S}]=0, &\quad  \boldsymbol{S}\boldsymbol{S}^{2}=\boldsymbol{S}^{2}\boldsymbol{S}=2\boldsymbol{S}	\\
\end{empheq}\eqnormal

\exercise 两个自旋量子数为$\dfrac{1}{2}$的非全同粒子,位置固定,相互作用能(算符)为$H=\dfrac{\omega}{\hbar}\boldsymbol{S}_{1}\cdot\boldsymbol{S}_{2}$时粒子1自旋$(\boldsymbol{S}_{1})$沿正$z$轴方向极化,粒子2自旋$(\boldsymbol{S}_{2})$极化方向正好相反.对于任意$t>0$时刻,

(a) 求体系总自旋波函数$\chi(t)$,用单粒子自旋态$\alpha(1),\beta(1),\alpha(2),\beta(2)$表示出来,[提示:将初始总自旋态展开成$H$的本征态的叠加.]

(b) 求$\boldsymbol{S}_{1}$极化方向与初始指向相同的概率.

(c) 求$\boldsymbol{S}_{1}$和$\boldsymbol{S}_{2}$极化方向均指向正$z$轴的概率.

(d) 总自旋量子数$S=1,0$的各自概率.
\pskip
\exercise 如原子的最外层是两个单粒子能级为$E_{nl}$的电子,作为二电子体系,讨论其总$L,S,J$的可能取值组合,证明$L+S=$偶数.

[提示:注意波函数的交换对称性,同时参看$\S$\ref{sec:10.02}例题.]
\pskip
\exercise 已知氘核(d)自旋为$\hbar$,宇称为偶;中子(n)自旋为$\dfrac{\hbar}{2}$;$\pi^{-}$介子自旋为0.试根据反应
\begin{empheq}{equation*}
	\pi^{-}\text{(静止)}+\ce{d}\rightarrow \ce{n}+\ce{n}
\end{empheq}

决定$\pi^{-}$介子的内禀宇称.

[提示:反应前后角动量守恒,宇称守恒.反应后为费密子体系.]
\pskip
\exercise 考虑一个多电子原子或分子,总能量包括各个电子及原子核的动能以及粒子间的库仑势能.(忽略自旋和相对论效应)体系的基态能级记为$E_{0}$,基态波函数记为$\varPsi_{0}$,预备用变分法找$E_{0}$,$\varPsi_{0}$的近似.设己找到$\varPsi_{0}$的粗略近似$\phi(\boldsymbol{r}_{1},\boldsymbol{r}_{2},\cdots,\boldsymbol{r}_{N})$,并已求出体系动能平均值$W$,势能平均值$U<0$,再取试探波函数为$\varPsi(\lambda)=\lambda^{\dfrac{3N}{2}}\phi(\lambda\boldsymbol{r}_{1},\cdots,\lambda\boldsymbol{r}_{N})$.证明:

(a) $\lambda$最佳值为$\lambda_{0}=-\dfrac{U}{2W}$,$E_{0}$近似值$E(\lambda_{0})=-\dfrac{U^{2}}{4W}$.

(b) 如$\phi(\boldsymbol{r}_{1},\cdots,\boldsymbol{r}_{N})$刚好就是$\varPsi_{0}$,则$U=-2W,E_{0}=W+U$(即位力定理结论)试将本题结论用于氦原子,利用微扰论结果直接得出变分法结果.

\end{exercises}