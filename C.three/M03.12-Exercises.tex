\begin{exercises}

\exercise 粒子在一维无限深势阱[$\S$\ref{sec:02.04}(1.)]中运动,已知初始波函数,$\varPsi(x,0)=\frac{1}{\sqrt{2}}[\varPsi_{1}(x)+\varPsi_{2}(x)]$,求$\varPsi(x,t)$,$\bar{E}$,$\overline{E^{2}}$,$\bar{x}(t)$.

\exercise 粒子在一维无限深势阱($0<x<a$,见2-5题)中运动,初始波函数为$\varPsi(x,0)=cx(a-x)$,$c$为归一化常数,请确定$c$,并计算各能量本征值$(E_{n})$的测量概率以及$\bar{E}$,$\Delta E$.

\exercise 已知平面转子(见2-6题)初始波函数为$\varPsi(\varphi,0)=A\sin^{2}\varphi$,$A$为归一化常数,请确定$A$,并求$\varPsi(\varphi,t)$以及$\bar{E}$,$\Delta E$.

\exercise 粒子作一维自由运动,$-\infty<x<\infty$,已知初始波函数为$\varPsi(x,0)=A(\sin^{2}k_{0}x+\cos k_{0}x)$,求$\varPsi(x,t)$,动量$(p_{x})$及动能的取值及概率以及平均值.

\exercise 一维谐振子,初始波函数为$\varPsi(x,0)=\frac{\sqrt{3}}{2}\varPsi_{0}(x)+\frac{1}{2}\varPsi_{1}(x)$,求:$\varPsi(x,t)$,$\bar{E}(t)$,$\bar{x}(t)$,$\bar{p}_{x}(t)$.

\exercise 谐振子处于基态,求动量分布概率,并据此计算动能平均值以及$\Delta p$,与2-14题的结果比较.

\exercise 一维谐振子$\bigg(V=\frac{kx^{2}}{2}\bigg)$处于基态.如势场突然变成$V_{2}(x)=kx^{2}$,即弹性力增大一倍.求$V_{2}$场的能级以及粒子处于新势场中基态的概率.

\exercise 粒子处于一维无限深势阱$(|x|<a)$中的基态.如势阱突然扩展至$|x|<2a$,(阱宽加倍)设扩展时粒子的波函数不变.求扩展后粒子能量的可能测值以及粒子处于新势阱基态的概率.

\exercise 证明:对于任何一维束缚态,$\overline{xp_{x}}=\frac{i\hbar}{2}$,$\overline{p_{x}x}=-\frac{i\hbar}{2}$,并举一个实例验证之.

\exercise 粒子作一维运动,如归一化波函数$\varPsi(x)=u(x)e^{i\hbar x}$,$u(x)$及$k$均为实数,求动量平
均值.

\exercise 利用基本对易式$[x,p_{x}]=i\hbar$等,计算$[p_{x},L_{y}]$,$[p_{x},L_{z}]$,并写出另外几个下标轮换$(x\rightarrow y\rightarrow z\rightarrow x)$公式.

\exercise 证明算符关系:
\begin{empheq}{align*}
	\boldsymbol{r}\times\boldsymbol{L}+\boldsymbol{L}\times\boldsymbol{r}&=2i\hbar\boldsymbol{r}	\\
	\boldsymbol{p}\times\boldsymbol{L}+\boldsymbol{L}\times\boldsymbol{p}&=2i\hbar\boldsymbol{p}
\end{empheq}

\exercise 设线性算符$F$不是厄密的,即$F\neq F^{+}$.试证明$F$可以分解成$F=A+iB$,$A,B$均为厄密算符.求$A,B$.

\exercise 设$U$为么正算符,$(UU^{+}=U^{*}U=1)$证明$U$必可分解成$U=A+iB$,$A,B$为对易$(AB=BA)$厄密算符,而且$A^{2}+B^{2}=1$.进一步再证明$U$可以表示成$U=e^{iD}$的形式,$D$为厄密算符.
[提示:$A,B$的共同本符函数是完备系,本征值$a_{n},b_{n}$.满足关系$a_{n}^{2}+b_{n}^{2}=1$.]

\exercise 算符$A$(无量纲)的指数函数定义成
\begin{empheq}{equation*}
	e^{A}=\sum_{n=0}^{\infty}\frac{A^{n}}{n!}=1+A+\frac{A^{2}}{2!}+\cdots+\frac{A^{n}}{n!}+\cdots
\end{empheq}

(a) 如$A,B$对易,证明$e^{A}e^{B}=e^{A+B}$

(b) 如$A,B$不对易,令$[A,B]=AB-BA=C_{1},[A,C_{1}]=C_{2},\cdots,[A,C_{n}]=C_{n+1}$,证明公式
\begin{empheq}{equation*}
	e^{A}Be^{-A}=B+\sum_{n=1}^{\infty}\frac{C_{n}}{n!}
\end{empheq}

\exercise 设$p,q$为无量纲算符,$[q,p]=i$,$f(q)$是$q$的解析函数.(可展开为正幕级数)证明下列公式:

\begin{tabular}{ll}
	(a) $[q^{n},p]=inq^{n-1}$ & (b) $[f,p]=i\frac{df}{dq}$ \\ 
	(c) $[e^{i\lambda q},p]=-\lambda e^{-i\lambda q}$($\lambda$为参数) & (d) $[q,p^{n}]=inp^{n-1}$ \\
	(e)	$[q,e^{i\lambda q}]=-\lambda e^{i\lambda q}$ & \\
\end{tabular}

\exercise 设算符$A,B$不对易,$[A,B]=C$,而$C$与$A$及$B$均可对易,即$[C,A]=0,[C,B]=0$.计算$[A,B^{n}]$,$[A,e^{B}]$,$[A,f(B)]$,$f(B)$为$B$的解析函数(可展开为正幕级数).

\exercise 同上题,证明Glauber公式:
\begin{empheq}{equation*}
	e^{A+B}=e^{A}e^{B}e^{-\frac{C}{2}}=e^{B}e^{A}e^{\frac{C}{2}}
\end{empheq}

[提示: 令$f(\lambda)=e^{\lambda A}e^{\lambda B}$,$\lambda$为参数,证明
\begin{empheq}{equation*}
	\frac{df(\lambda)}{d\lambda}=f(\lambda)(A+B+\lambda C)
\end{empheq}
再对$\lambda$积分$\cdots\cdots$最后取$\lambda=1$.]

\exercise 设力学量算符$A$满足的最简单的代数方程式是
\begin{empheq}{equation*}
	f(A)=A^{n}+C_{n-1}A^{n-1}+\cdots+C_{1}A+C_{0}=0
\end{empheq}
各$C_{i}$为常数.试证明:$A$有$n$个本征值,它们都是方程$f(x)=0$的根.

\exercise 设力学量$k$的算符可以表示成两个不对易算符$L$及$M$之积,$K=LM$,而$L,M$的对易式为$[L,M]=1$.$K$的本征函数、本征值记为$\varPsi_{n}$,$\lambda_{n}(n=1,2,\cdots)$试证明:

\noindent 函数$L\varPsi_{n},M\varPsi_{n}$如存在,则它们也是$K$的本征函数,本征值分别为$(\lambda_{n}-1),(\lambda_{n}+1)$.

\exercise 求能使$x-p_{x}$不确定度关系下限成立$\bigg(\Delta x\Delta p_{x}=\frac{\hbar}{2}\bigg)$的一维波函数.

\exercise 证明:对于$H=\frac{\boldsymbol{p}^{2}}{2m}+V(\boldsymbol{r})$的情形,在任何束缚定态下,$\boldsymbol{p}$各分量的平均值为0.

\exercise 粒子在一维势场$V(x)=k|x|^{\nu}(k,\nu>0)$中运动,利用不确定度关系估算基态能级及$\Delta x$的量级.

\exercise 核子(质子,中子)间的核力(强作用)通过吞吐$\pi$介子而传递,$m_{\pi}\approx 270m_{e}$.试利用不确定度关系估算核力力程.

\exercise 质量$m$,电荷$q$的粒子在均匀电场(场强$\mathscr{E}$,沿正$x$轴方向)中运动,能量算符为$H=\frac{p_{x}^{2}}{2m}=-q\mathscr{E}x$.已知$t=0$时$\bar{x}=0,\bar{p}_{x}=p_{0}$,试利用海森堡方程计算$\bar{x}(t),\bar{p}_{x}(t)$.

\exercise 带电粒子在均匀磁场(沿$z$轴方向)中运动时,哈密顿算符可以近似表示成$H=\frac{\boldsymbol{p}^{2}}{2m}-\omega\boldsymbol{L}$,$\omega=\frac{qB}{2mc}$,$q$为粒子的电荷,$B$为场强.已知$t=0$时$\bar{p}_{x}=p_{0}$,$\bar{p}_{y}=\bar{p}_{z}$,试求$t>0$时$\bar{p}_{x}(t),\bar{p}_{y}(t),\bar{p}_{z}(t)$.又,本题有哪些重要的守恒力学量?

\exercise 对于一维谐振子,求海森堡图像中算符$\hat{x}(t)$,$\hat{p}_{x}(t)$,将它们用$\hat{x},\hat{p}_{x},t$表示出来.

\exercise 粒子在势场$V(\boldsymbol{r})$中运动,设$V$与粒子的质量无关.证明:如粒子的质量增大,则束缚态能级下降.

[提示:利用海尔曼定理.]

\exercise 质量为$m$的粒子在对数函数型中心力场$V(\boldsymbol{r})=V_{0}\ln\bigg(\frac{r}{r_{0}})$中运动,$V_{0},r_{0}>0$并与质量$m$无关.试利用海尔曼定理和位力定理,证明:(a) 各个束缚态的能量虽然不同,但动能平均值相同. (b) 如改变粒子的质量,任何两个能级的间距不受影响.

\exercise 粒子作一维运动,哈密顿算符为$H=\frac{p^{2}}{2m}+\frac{\lambda}{m}p+\frac{k}{2}x^{2}$($p$即$p_{x},\lambda,k>0$)求能谱.

[提示:利用海尔曼定理和谐振子的结果.]



\end{exercises}
