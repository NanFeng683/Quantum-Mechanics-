\section[状态和力学量随时间的变化]{状态和力学量随时间的变化} \label{sec:03.09} % 
% \makebox[5em][s]{} % 短题目拉间距

{\heiti 1. 波函数随时间的变化}

波函数$\varPsi$随时间的变化取决于薛定谔方程
\begin{empheq}{equation}\label{eq39.1}
	i\hbar\frac{\partial}{\partial t}\varPsi(\boldsymbol{r},t)=\hat{H}\varPsi(\boldsymbol{r},t)
\end{empheq}
$\hat{H}$为哈密顿算符.如$\hat{H}$与时间$t$无关,$\hat{H}$就是总能量算符.在以下的讨论中,均设$\hat{H}$与$t$无关.设$\hat{H}$的本征函数,本征值为$|\varPsi_{n}(\boldsymbol{r},E_{n})|$,满足本征方程
\begin{empheq}{equation}\label{eq39.2}
	\hat{H}\varPsi_{n}=E_{n}\varPsi_{n},\quad n=1,2,\cdots
\end{empheq}
\eqref{eq39.1}式是时间$t$的一阶微分方程,只需给定初始时刻$(t_{0}=0)$的波函数$\varPsi(\boldsymbol{r},0)$,就可以解出所有时刻的波函数$\varPsi(\boldsymbol{r},t)$.具体说来,先将$\varPsi(\boldsymbol{r},0)$表示成各$\varPsi_{n}(\boldsymbol{r})$的线性叠加,
\begin{empheq}{equation}\label{eq39.3}
	\varPsi(\boldsymbol{r},0)=\sum_{n}C_{n}\varPsi_{n}(\boldsymbol{r})
\end{empheq}
其中
\begin{empheq}{equation}\label{eq39.4}
	C_{n}=\int\varPsi_{n}^{*}(\boldsymbol{r})\varPsi(\boldsymbol{r},0)d\tau
\end{empheq}
以\eqref{eq39.3}式作为初始条件,容易求出\eqref{eq39.1}式的解为[参看\eqref{eq21.21}式]
\begin{empheq}{equation}\label{eq39.5}
	\varPsi(\boldsymbol{r},0)=\sum_{n}C_{n}e^{E_{n}t/(i\hbar)}\varPsi_{n}(\boldsymbol{r})
\end{empheq}
利用本征方程\eqref{eq39.2}以及\eqref{eq39.3}式,可将\eqref{eq39.5}式化成
\begin{empheq}{align}\label{eq39.6}
	\varPsi(\boldsymbol{r},t) &=\sum_{n}C_{n}e^{\hat{H}t/(i\hbar)}\varPsi_{n}(\boldsymbol{r})	\nonumber\\
	&=e^{\hat{H}t/(i\hbar)}\varPsi(\boldsymbol{r},0)=\hat{U}(t)\varPsi(\boldsymbol{r},0)
\end{empheq}
其中
\begin{empheq}{equation}\label{eq39.7}
	\hat{U}(t)=e^{\hat{H}t/(i\hbar)}\equiv\sum_{\nu=0}^{\infty}\frac{1}{\nu!}\bigg(\frac{\hat{H}t}{i\hbar}\bigg)^{\nu}
\end{empheq}
称为演化算符.\eqref{eq39.6}式表明,演化算符$\hat{U}(t)$作用于初始时刻的波函数,就得到$t$时刻的波函数.注意$\hat{U}(0)=1$.$\hat{U}(t)$不是厄密算符,其共轭为
\begin{empheq}{equation}\label{eq39.8}
	\hat{U}^{*}(t)=e^{-\hat{H}t/(i\hbar)}
\end{empheq}
因此
\begin{empheq}{equation}\label{eq39.9}
	\hat{U}^{*}(t)\hat{U}(t)=\hat{U}(t)\hat{U}^{*}(t)=1
\end{empheq}
亦即$\hat{U}(t)$是一种么正变换.注意$\hat{U}(t)$和$\hat{U}^{*}(t)$均与$\hat{H}$对易,其变化率为
\begin{empheq}{equation}\label{eq39.10}
	\begin{aligned}
		\frac{\partial\hat{U}(t)}{\partial t}=\frac{1}{i\hbar}\hat{H}\hat{U}(t)=\frac{1}{i\hbar}\hat{U}(t)\hat{H}	\\
		\frac{\partial^{*}\hat{U}(t)}{\partial t}=\frac{1}{i\hbar}\hat{H}\hat{U}^{*}(t)=\frac{1}{i\hbar}\hat{U}^{*}(t)\hat{H}
	\end{aligned}
\end{empheq}

{\heiti 2. 力学量平均值随时间的变化}

量子力学中力学量的变化率不能仿照经典力学的方式来定义,在经典力学中,处于某种运动状态的质点,任何力学量$F(t)$在任何时刻$t$均有明确的数值,而且随$t$作连续变化,所以可以定义$F$的时间变化率为
\begin{empheq}{equation*}
	\frac{dF(t)}{dt}=\lim_{\Delta\rightarrow 0}\frac{F(t+\Delta t)-F(t)}{\Delta t}
\end{empheq}
量子力学则不同.我们来讨论在任意运动态$\varPsi(\boldsymbol{r},t)$下某个力学量$A(\boldsymbol{r,p})$的取值问题.首先,一般来说,在任意时刻$t$,力学量$A$的取值不是唯一的,可以取多种本征值(各有某种几率).其次,即使在某个特定时刻$t_{0}$,$\varPsi(\boldsymbol{r},t_{0})$刚好是$\hat{A}$的本征态,因而这时$A$取单一的本征值,但在随后的时刻,由于波函数按照薛定谔方程随时间变化,$\varPsi(\boldsymbol{r},t_{0}+\Delta t)$一般就不再是$\hat{A}$的本征态,因此在时刻$t_{0}+\Delta t$,$A$将取多种本征值这样,当然就谈不上力学量$A$的取值随时间作连续变化.所以经典力学量的变化率定义在量子力学中完全不适用.

在量子力学中,力学量$A(\boldsymbol{r,p})$在运动态$\varPsi(\boldsymbol{r},t)$下的平均值为
\begin{empheq}{equation}\label{eq39.11}
	\bar{A}(t)=\int\varPsi^{*}(\boldsymbol{r},t)\hat{A}\varPsi(\boldsymbol{r},t)d\tau
\end{empheq}
由于$\varPsi$随$t$作连续变化,所以$\bar{A}$亦随$t$作连续变化.力学量$A$的变化率算符$\frac{d\hat{A}}{dt}$按照下述方式定义:$\frac{d\hat{A}}{dt}$在运动态$\varPsi(\boldsymbol{r},t)$下的平均值等于$\\bar{A}(t)$的变化率,亦即
\eqindent{1}
\begin{empheq}{equation}\label{eq39.12}
	\frac{\overline{dA}}{dt}\equiv\int\varPsi^{*}(\boldsymbol{r},t)\frac{d\hat{A}}{dt}\varPsi(\boldsymbol{r},t)d\tau=\frac{d}{dt}\int\varPsi^{*}(\boldsymbol{r},t)\hat{A}\varPsi(\boldsymbol{r},t)d\tau=\frac{d}{dt}\overline{A}(t)
\end{empheq}\eqnormal
利用演化算符$\hat{U}(t)$,\eqref{eq39.11}式可以化成
\eqindent{6}
\begin{empheq}{align}\label{eq39.13}
	\overline{A}(t) &=\int\{\hat{U}(t)\varPsi(\boldsymbol{r},0)\}^{*}\hat{A}\hat{U}(t)\varPsi(\boldsymbol{r},0)d\tau\nonumber	\\
	&=\int\varPsi^{*}(\boldsymbol{r},0)\hat{U}^{+}(t)\hat{A}\hat{U}(t)\varPsi(\boldsymbol{r},0)d\tau
\end{empheq}\eqnormal
利用\eqref{eq39.10}式,不难求出
\eqindent{4}
\begin{empheq}{align}\label{eq39.14}
	\frac{d\overline{A}(t)}{dt}
	&=\int\varPsi^{*}(\boldsymbol{r},0)
	\biggl\{\frac{\partial}{\partial t}\hat{U}^{*}(t)\hat{A}\hat{U}(t)\biggr\}
	\varPsi(\boldsymbol{r},0)d\tau	\nonumber\\
	&=\frac{1}{i\hbar}\int\varPsi^{*}(\boldsymbol{r},0)\hat{U}^{+}(t)(\hat{A}\hat{H}-\hat{H}\hat{A})\hat{U}(t)\varPsi(\boldsymbol{r},0)d\tau	\nonumber\\
	&=\frac{1}{i\hbar}\int\{\hat{U}(t)\varPsi(\boldsymbol{r},0)\}^{*}(\hat{A}\hat{H}-\hat{H}\hat{A})\hat{U}(t)\varPsi(\boldsymbol{r},0)d\tau	\nonumber\\
	&=\frac{1}{i\hbar}\int\varPsi^{*}(\boldsymbol{r},t)(\hat{A}\hat{H}-\hat{H}\hat{A})\hat{U}(t)\varPsi(\boldsymbol{r},t)d\tau	\nonumber\\
	&=\frac{1}{i\hbar}\overline{(AH-HA)}=\frac{1}{i\hbar}\overline{[\hat{A},\hat{H}]}
\end{empheq}\eqnormal
亦即
\begin{empheq}{equation}\label{eq39.15}
	\boxed{\frac{d\hat{A}}{dt}=\frac{1}{i\hbar}[\hat{A},\hat{H}]}
\end{empheq}
在\eqref{eq39.14}式的推导过程中,并没有用到$\hat{A}$是厄密算符$(\hat{A}^{*}=\hat{A})$这个条件.所以,对于非厄密算符$(\hat{A}^{*}\neq\hat{A})$,仍可将\eqref{eq39.11}式作为其平均值的定义,\eqref{eq39.12}式作为其变化率算符的定义,结果仍得到\eqref{eq39.14}、\eqref{eq39.15}式.

如果$\hat{A},\hat{H}$对易,即$[\hat{A},\hat{H}]$,则$\frac{d\hat{A}}{dt}=0,\frac{d\bar{A}}{dt}=0$,力学量$A$称为守恒量.当$\hat{H}$不显含$t$时,$H$就是守恒量,即能量守恒.

\eqref{eq39.15}式也称力学量算符的运动方程.下面就
\begin{empheq}{equation}\label{eq39.16}
	\hat{H}=\frac{\hat{\boldsymbol{p}}^{2}}{2m}+V(\boldsymbol{r})
\end{empheq}
的情形,具体求出基本力学量算符的运动方程.先求位置$r$的运动方程,
\begin{empheq}{equation}\label{eq39.17}
	\frac{d\hat{x}}{dt}=\frac{1}{i\hbar}[\hat{x},\hat{H}]=\frac{1}{i\hbar 2m}[\hat{x},\hat{\boldsymbol{p}^{2}}]=\frac{\hat{p_{x}}}{m}
\end{empheq}
类似地,求得
\begin{empheq}{equation*}\label{eq39.17'}
	\frac{d\hat{y}}{dt}=\frac{\hat{p_{y}}}{m},\quad\frac{d\hat{z}}{dt}=\frac{\hat{p_{z}}}{m}	\tag{$3.9.17^{\prime}$}
\end{empheq}
写成矢量形式,就是
\begin{empheq}{equation*}\label{eq39.17''}
	\frac{d\hat{\boldsymbol{r}}}{dt}=\frac{\hat{\boldsymbol{r}}}{m}	\tag{$3.9.17^{\prime\prime}$}
\end{empheq}
这公式相当于经典力学中速度的定义,再看动量,
\begin{empheq}{equation}\label{eq39.18}
	\frac{d\hat{p_{x}}}{dt}=\frac{1}{i\hbar}[\hat{p_{x}},\hat{H}]=\frac{1}{i\hbar}[\hat{p_{x}},V]=-\frac{\partial V}{\partial x}
\end{empheq}
类似地,求得
\begin{empheq}{equation*}\label{eq39.18'}
	\frac{d\hat{p_{y}}}{dt}=-\frac{\partial V}{\partial y},\quad\frac{d\hat{p_{z}}}{dt}=-\frac{\partial V}{\partial z}	\tag{$3.9.18^{\prime}$}
\end{empheq}
写成矢量形式,就是
\begin{empheq}{equation*}\label{eq39.18''}
	\frac{d\hat{\boldsymbol{p}}}{dt}=0=-\nabla V=\boldsymbol{F}	\tag{$3.9.18^{\prime\prime}$}
\end{empheq}
这公式相当于经典力学中的牛顿第二定律.如果粒子作自由运动,$V$等于恒量,即得$\frac{d\hat{\boldsymbol{p}}}{dt}=0$,动量为守恒量.

再看角动量$\boldsymbol{L=r\times p}$的运动方程,由于$\hat{\boldsymbol{L}}$与$\hat{\boldsymbol{p}^{2}}$对易,所以
\eqindent{6}
\begin{empheq}{align}\label{eq39.19}
	\frac{dL_{x}}{dt}&=\frac{1}{i\hbar}[\hat{L_{x}},\hat{H}]=\frac{1}{i\hbar}[y\hat{p_{z}}-z\hat{p_{y}},V(\boldsymbol{r})]	\nonumber\\
	&=-\bigg(y\frac{\partial V}{\partial z}-z\frac{\partial V}{\partial y}\bigg)=-(\boldsymbol{r}\times\nabla V)
\end{empheq}\eqnormal
$y$分量和$z$分量也有类似公式.写成矢量形式,就是
\begin{empheq}{equation*}\label{eq39.19'}
	\frac{d\hat{\boldsymbol{L}}}{dt}=-\boldsymbol{r}\times\nabla V=\boldsymbol{r}\times\boldsymbol{F}	\tag{$3.9.19^{\prime}$}
\end{empheq}
经典力学中也有相应的公式.当粒子作自由运动$(\boldsymbol{F}=0)$,或在中心力场$V(r)$中运动时,$\boldsymbol{r}$和$\boldsymbol{F}$平行,$\boldsymbol{r}\times\boldsymbol{F}=0$,这时$\frac{d\hat{\boldsymbol{L}}}{dt}=0$,$\boldsymbol{L}$的各个分量均为守恒量.

通过以上的计算,我们看到经典力学中的主要守恒定律在量子力学中仍然成立,(只是通过算符的形式表现出来.)这并不奇怪,因为守恒定律是物质运动规律统一性的具体表现,正因为存在着微观的(量子力学的)守恒定律,才导致相应的宏观(经典力学的)守恒定律.

{\heiti 3. 海森堡运动方程}

平均值公式\eqref{eq39.13}可以写成
\begin{empheq}{equation}\label{eq39.20}
	\overline{A}(t)=\int\varPsi^{*}(\boldsymbol{r},0)\hat{A}(t)\varPsi(\boldsymbol{r},0)d\tau
\end{empheq}
其中
\begin{empheq}{equation}\label{eq39.21}
	\boxed{\hat{A}(t)=\hat{U}^{*}\hat{A}\hat{U}(t)=e^{-\hat{H}t/(i\hbar)}\hat{A}e^{\hat{H}t/(i\hbar)}}
\end{empheq}
称为力学量算符$\hat{A}$的海森堡图像.按照海森堡的观点,当$\hat{H}$不随时间改变,物理体系的量子力学状态是不随时间变化的,而力学量算符则要随时间变化,变化规律就是\eqref{eq39.21}式.对量子力学规律的这种描述方式称为海森堡图像.\eqref{eq39.20}式就是海森堡图像中力学量的平均值公式.在此以前的内容则属于薛定谔图像.两种图像显然是等价的.

在海森堡图像中,算符的时间变化率可以按经典方式定义,比较自然.\eqref{eq39.21}式对$t$求导,并利用\eqref{eq39.10}式,得到
\eqindent{6}
\begin{empheq}{align}\label{eq39.22}
	\frac{d}{dt}\hat{A}(t) &=\frac{d\hat{U}^{+}}{dt}\hat{A}\hat{U}+\hat{U}^{+}A\frac{d\hat{U}}{dt}	\nonumber\\
	&=\frac{1}{i\hbar}(\hat{U}^{+}\hat{A}\hat{U}\hat{H}-\hat{H}\hat{U}^{+}\hat{A}\hat{U})	\nonumber\\
	&=\frac{1}{i\hbar}\{\hat{A}(t)\hat{H}-\hat{H}\hat{A}(t)\}
\end{empheq}\eqnormal
由于
\begin{empheq}{equation}\label{eq39.23}
	\hat{H}(t)=\hat{U}^{+}(t)\hat{H}\hat{U}(t)=\hat{U}^{+}\hat{U}\hat{H}=\hat{H}
\end{empheq}
与$t$无关,\eqref{eq39.22}式亦即
\begin{empheq}{equation}\label{eq39.24}
	\boxed{\frac{d}{dt}\hat{A}(t)=\frac{1}{i\hbar}[\hat{A}(t),\hat{H}(t)]}
\end{empheq}
称为力学量算符的海森堡运动方程.注意\eqref{eq39.24}式与\eqref{eq39.15}式结构相似,$\hat{A}$换成$\hat{A}(t)$而已.\eqref{eq39.22}或\eqref{eq39.24}式也可以写成
\begin{empheq}{equation}\label{eq39.24'}
	\frac{d}{dt}\hat{A}(t)=\frac{1}{i\hbar}\hat{U}^{+}(t)[\hat{A},\hat{H}]\hat{U}(t)	\tag{$3.9.24^{\prime}$}
\end{empheq}
由\eqref{eq39.21}或\eqref{eq39.24'}式易见,如$\hat{A},\hat{U}$对易,则$\hat{A},\hat{U}(t)$也对易,因此$\hat{A}(t)=\hat{A}$,不随时间变化,$\frac{d\hat{A}(t)}{dt}=0$, 亦即$\hat{A}(t)$为守恒量.

由于\eqref{eq39.24}式与\eqref{eq39.15}式结构相似,在(2)节中得到的基本力学量的运动方程\eqref{eq39.17}至\eqref{eq39.19'}式在海森堡图像中仍然成立,只需将$\hat{A}$换成$\hat{A}(t)$.因此\eqref{eq39.15}式有时也称海森堡运动方程.

{\heiti 4. 关于守恒量和定态}

如果初始波函数$\varPsi(\boldsymbol{r},0)$是$\hat{H}$的本征函数,即$\varPsi(\boldsymbol{r},0)=\varPsi_{n}(\boldsymbol{r})$,则$\varPsi(\boldsymbol{r},t)$为定态波函数,
\begin{empheq}{equation}\label{eq39.25}
	\varPsi(\boldsymbol{r},t)=\varPsi_{n}(\boldsymbol{r})e^{E_{n}t/i\hbar}
\end{empheq}
在定态下,任何力学量$A(\boldsymbol{r,p})$的平均值均不随时间改变.
\begin{empheq}{align}\label{eq39.26}
	\overline{A}(t)
	&=\int\varPsi^{*}(\boldsymbol{r},t)\hat{A}\varPsi(\boldsymbol{r},t)d\tau	\nonumber\\
	&=\int\varPsi_{n}^{*}(\boldsymbol{r})e^{-E_{n}t/i\hbar}\hat{A}\varPsi_{n}(\boldsymbol{r})e^{E_{n}t/i\hbar}d\tau	\nonumber\\
	&=\int\varPsi_{n}^{*}(\boldsymbol{r})\hat{A}\varPsi_{n}(\boldsymbol{r})d\tau=\bar{A}(t=0)
\end{empheq}
不仅如此,如将$\varPsi_{n}(r)$展开成$\hat{A}$的本征函数$\{\Phi_{l}\}$的线性叠加:
\begin{empheq}{equation}\label{eq39.27}
	\varPsi_{n}(\boldsymbol{r})=\sum_{l}C_{l}\Phi_{l}(\boldsymbol{r})
\end{empheq}
其中$\Phi_{l}$满足本征方程:
\begin{empheq}{equation*}
	\hat{A}\Phi_{l}=a_{l}\Phi_{l}
\end{empheq}
$a_{l}$为$\hat{A}$的本征值.将\eqref{eq39.27}式代入\eqref{eq39.24}式,得到
\begin{empheq}{equation}\label{eq39.28}
	\varPsi(\boldsymbol{r},t)=\sum_{l}C_{l}e^{E_{n}t/i\hbar}\Phi_{l}(\boldsymbol{r})
\end{empheq}
按照波函数的普遍概率诠释,在$\varPsi(\boldsymbol{r},t)$态下,力学量$A$取本征值$a_{l}$的概率是
\begin{empheq}{equation*}
	|C_{l}e^{E_{n}t/i\hbar}|^{2}=|C_{l}|^{2}
\end{empheq}
与时间无关.这就是说,在定态下,任何不显含$t$的力学量取各个本征值的概率不随时间改变.因此,定态是能量具有确定值的稳定状态.“稳定”的意思是各种物理性质均不随时间改变.

如果是在非定态下,波函数由\eqref{eq39.5}式表示.这时守恒量和非守恒量的性质就有根本的区别.对于非守恒量$\hat{A}$,由于$\bar{A}$随$t$变化,显然$A$取各个本征值的概率也将随$t$变化.守恒量的情况就不同了.设$F(\boldsymbol{r,p})$为守恒量,即$\hat{F},\hat{H}$对易,它们必然存在一个共同本征函数系$\{\varPsi_{ni}\}$,构成完备函数系.$\varPsi_{ni}$满足本征方程:
\begin{empheq}{equation}\label{eq39.29}
	\hat{H}\varPsi_{ni}=E_{n}\varPsi_{ni},\quad \hat{F}\varPsi_{ni}=f_{i}\varPsi_{ni}
\end{empheq}
初始波函数$\varPsi(\boldsymbol{r},0)$可以展开成各$\varPsi_{ni}$的线性叠加:
\begin{empheq}{equation}\label{eq39.30}
	\varPsi(\boldsymbol{r},0)=\sum_{n,i}C_{ni}\varPsi_{ni}(\boldsymbol{r})
\end{empheq}
因此
\begin{empheq}{align}\label{eq39.31}
	\varPsi(\boldsymbol{r},t) &=e^{\hat{H}t/i\hbar}\varPsi(\boldsymbol{r},0)	\nonumber\\
	&=\sum_{n,i}C_{ni}e^{E_{n}t/i\hbar}\varPsi_{ni}(\boldsymbol{r})
\end{empheq}
按照波函数的概率解释,在$\varPsi(\boldsymbol{r},t)$态下,在$t$时刻,$(H,F)$取本征值$(E_{n},f_{i})$的概
率为
\begin{empheq}{equation*}
	|C_{ni}e^{E_{n}t/i\hbar}|^{2}=|C_{ni}|^{2}
\end{empheq}
这个概率与$t$无关.这就是说,在任何运动态(包括定态和非定态)下,守恒量取各个本征值的概率不随时间改变.当然,平均值也不变.

如果初始状态刚好是守恒量$F$的本征态,相应于本征值$F=f_{i}$,则\eqref{eq39.30}和\eqref{eq39.31}式中下标$t$只有单一的取值,($n$仍可能有多种取值,这要看$\varPsi(\boldsymbol{r},0)$是否$\hat{H}$的本征函数而定.)因此在任何时刻$t$,$\varPsi(\boldsymbol{r},t)$仍是$\hat{F}$的本征态,本征值不变.正因如此,标志守恒量本征值的量子数称为好量子数.

在能级存在简并的情况下,必然存在独立于$\hat{H}$的其他守恒量.这种情况下,通常总是选取一组互相对易的包括$\hat{H}$在内的守恒量作为力学量完全集,即以它们的共同本征函数作为\eqref{eq39.5}式中的$\varPsi_{n}$或\eqref{eq39.31}式中的$\varPsi_{ni}$,用它们的线性叠加表示薛定谔方程\eqref{eq39.1}通解.由此可见,研究一个物理体系的性质,首要的课题就是找出主要的守恒量,确定守恒量完全集并求出它们的共同本征函数.

\example 设$\hat{A},\hat{B}$均为力学量算符,证明
\begin{empheq}{equation}\label{eq39.32}
	\frac{d}{dt}(\hat{A}\hat{B})=\hat{A}\frac{d\hat{B}}{dt}+\frac{d\hat{A}}{dt}\hat{B}
\end{empheq}
并就$\hat{H}=\hat{T}+V$的情形,求$\frac{d}{dt}\hat{x}^{2}$.

\solution 按照\eqref{eq39.15}式, 应有
\eqindent{6}
\begin{empheq}{align*}
	\frac{d}{dt}(\hat{A}\hat{B})&=\frac{1}{i\hbar}[\hat{A}\hat{B},\hat{H}]=\frac{1}{i\hbar}\{\hat{A}[\hat{B},\hat{H}]+[\hat{A},\hat{H}]\hat{B}\}	\\
	&=\hat{A}\frac{d\hat{B}}{dt}+\frac{d\hat{A}}{dt}\hat{B}
\end{empheq}\eqnormal
此即\eqref{eq39.32}式.如$\hat{A}=\hat{B}$,上式成为
\begin{empheq}{equation}\label{eq39.33}
	\frac{d}{dt}\hat{A}^{2}=\hat{A}\frac{d\hat{A}}{dt}+\frac{d\hat{A}}{dt}\hat{A}
\end{empheq}
注意$\hat{A}$和$\frac{d\hat{A}}{dt}$可能不对易.

如果$\hat{H}=\hat{T}+V$,这时
\begin{empheq}{equation*}
	\frac{d\hat{x}}{dt}=\frac{1}{i\hbar}[\hat{x},\hat{H}]=\frac{\hat{p_{x}}}{m}
	\tag{$3.9.17$}
\end{empheq}
则由\eqref{eq39.33}式得到
\begin{empheq}{equation}\label{eq39.34}
	\frac{d}{dt}\hat{x}^{2}=\frac{1}{m}(\hat{x}\hat{p_{x}}+\hat{p_{x}}\hat{x})
\end{empheq}








