\section[不确定度关系]{不确定度关系} \label{sec:03.08} % 
% \makebox[5em][s]{} % 短题目拉间距

考虑两个力学量$A,B$及其算符$\hat{A},\hat{B}$,任意给定一个归一化的波函数$\varPsi$.如$\varPsi$是$\hat{A}$或$\hat{B}$的本征函数,则力学量$A$或$B$将取单一的本征值,这种情况不予讨论.如$\varPsi$不是$\hat{A}$或$\hat{B}$的本征函数,则$A$和$B$的取值均将具有某种分布,定义分布宽度(涨落)$\Delta A$,$\Delta B$,
\begin{empheq}{equation}\label{eq38.1}
	\begin{aligned}
		(\Delta A)^{2}&=\overline{(\hat{A}-\bar{A})^{2}}=\overline{A^{2}}-\bar{A}^{2}	\\
		(\Delta B)^{2}&=\overline{(\hat{B}-\bar{B})^{2}}=\overline{B^{2}}-\bar{B}^{2}
	\end{aligned}
\end{empheq}
试问,对于各种不同的波函数$\varPsi$,$\Delta A\cdot\Delta B$的下限是什么?如果$\hat{A},\hat{B}$对易,可取$\varPsi$尽量接近$\hat{A},\hat{B}$的共同本征函数,使$\Delta A,\Delta B$都很小,所以$\Delta A\cdot\Delta B$的下限为0.如$\hat{A},\hat{B}$不对易,令
\begin{empheq}{equation}\label{eq38.2}
	[\hat{A},\hat{B}]\equiv\hat{A}\hat{B}-\hat{B}\hat{A}=i\hat{C}
\end{empheq}
$\hat{C}$必为厄密算符,它在$\varPsi$态下的平均值
\begin{empheq}{equation}\label{eq38.3}
	\bar{C}=-i\int\varPsi^{*}(\hat{A}\hat{B}-\hat{B}\hat{A})\varPsi d\tau
\end{empheq}
必为实数,可以证明
\begin{empheq}{equation}\label{eq38.4}
	\boxed{\Delta A\cdot\Delta B\geqslant\frac{1}{2}|\bar{C}|}
\end{empheq}
称为不确定度关系(曾称为测不准关系).\eqref{eq38.4}式的证明如下.

对于任意线性算符$\hat{F}$及其共轭$\hat{F}^{*}$,$\hat{F}^{*}\hat{F}$的平均值为
\eqindent{6}
\begin{empheq}{align}\label{eq38.5}
	\overline{\hat{F}^{*}\hat{F}}&=\int\varPsi^{*}\hat{F}^{*}\hat{F}\varPsi d\tau	\nonumber\\
	&=\int(\hat{F}\varPsi)(\hat{F}\varPsi)^{*}d\tau=\int|\hat{F}\varPsi|^{2}d\tau\geqslant 0
\end{empheq}\eqnormal
令
\begin{empheq}{equation}\label{eq38.6}
	\hat{F}=\hat{A}+i\xi\hat{B}
\end{empheq}
$\xi$为实参数(注意,如$A,B$量纲相同,$\xi$是无量纲参数;如$A,B$量纲不同,则$\xi$有量纲.)
\eqindent{6}
\begin{empheq}{align}\label{eq38.7}
	\hat{F}^{*}\hat{F}&=(\hat{A}-i\xi\hat{B})(\hat{A}+i\xi\hat{B})	\nonumber\\
		&=\hat{A}^{2}+\xi^{2}\hat{B}^{2}+i\xi[\hat{A},\hat{B}]=\hat{A}^{2}+\xi^{2}\hat{B}^{2}-\xi\hat{C}
\end{empheq}\eqnormal
代入\eqref{eq38.5}式, 即得
\begin{empheq}{equation}\label{eq38.8}
	\overline{A^{2}}+\xi^{2}\overline{B^{2}}-\xi\overline{C}\geqslant 0
\end{empheq}
等号成立的条件为
\begin{empheq}{equation}\label{eq38.9}
	\hat{F}\varPsi=(\hat{A}+i\xi\hat{B})\varPsi=0
\end{empheq}
\eqref{eq38.8}式对任何$\xi$值(实数),任何归一化波函数$\varPsi$,均可成立.但如$\varPsi$是$\hat{A}$或$\hat{B}$的本征函数,这时$\bar{C}=0$,\eqref{eq38.8}式没有实际价值.

\eqref{eq38.8}式中取$\xi=\pm 1$(如$\bar{C}>0$,取$\xi=1$;如$\bar{C}<0$,取$\xi=-1$)得到
\begin{empheq}{equation}\label{eq38.10}
	\overline{A^{2}}+\overline{B^{2}}\geqslant|\bar{C}|
\end{empheq}
\eqref{eq38.8}式中取$\xi=\bar{C}/(2\overline{B^{2}})$或$\xi=\pm\sqrt{\overline{A^{2}}/\overline{B^{2}}}$($\bar{C}>0$,取正号;$\bar{C}<0$,取负号),得到
\begin{empheq}{equation}\label{eq38.11}
	\overline{A^{2}}\overline{B^{2}}\geqslant\frac{1}{4}\bar{C}^{2}
\end{empheq}
这实际上是\eqref{eq38.8}式成立的充要条件.\eqref{eq38.11}式其实已经包含了\eqref{eq38.10}式.

\eqref{eq38.8}至\eqref{eq38.11}式适用于任何厄密算符$\hat{A},\hat{B}$.如将$\hat{A}$换成$(\hat{A}-\bar{A})$,$\hat{B}$换成$(\hat{B}-\bar{B})$,则
\begin{empheq}{align*}
	[\hat{A}-\bar{A},\hat{B}-\bar{B}]=[\hat{A},\hat{B}]=i\hat{C}	\\
	\overline{(\hat{A}-\bar{A})^{2}}=(\Delta A)^{2},\quad\overline{(\hat{B}-\bar{B})^{2}}=(\Delta B)^{2} 
\end{empheq}
\eqref{eq38.10}式换成
\begin{empheq}{equation}\label{eq38.12}
	(\Delta A)^{2}+(\Delta B)^{2}\geqslant|\bar{C}|=|\overline{AB-BA}|
\end{empheq}
注意,\eqref{eq38.10}式和\eqref{eq38.12}式仅适用于$A,B$量纲相同的情形.\eqref{eq38.11}式换成
\begin{empheq}{equation*}
	\Delta A\cdot\Delta B\geqslant\frac{1}{2}|\bar{C}|=\frac{1}{2}|\overline{AB-BA}|
\end{empheq}
这就是不确定度关系\eqref{eq38.4}式成立,\eqref{eq38.12}式必然成立.

狭义的不确定度关系是指$A=x,B=p_{x}$的情形,这时
\begin{empheq}{equation*}
	[x,p_{x}]=i\hbar
	\tag{$3.8.4$}
\end{empheq}
相当于$\hat{C}=\hbar$,这时\eqref{eq38.11}式和\eqref{eq38.4}式变成
\begin{empheq}{equation}\label{eq38.13}
	\overline{x^{2}}\overline{p_{x}^{2}}\geqslant\frac{\hbar^{2}}{4}
\end{empheq}
\begin{empheq}{equation}\label{eq38.14}
	\boxed{\Delta x\cdot\Delta p_{x}\geqslant\frac{\hbar}{2}}
\end{empheq}
历史上, 不确定关系是海森堡(W. Heisenberg)首先提出来的.海森堡分析了若干典型实验后指出,微观粒子的位置和动量不可能同时予以精确测定,而只能确定到(在量级的意义上)
\begin{empheq}{equation*}
	\Delta x\cdot\Delta p\gtrsim h
\end{empheq}
的程度.\eqref{eq38.14}式则是严格的定量结果.注意,\eqref{eq38.14}式只规定了$\Delta x\cdot\Delta p_{x}$的下限,并未规定出上限.对许多典型定态问题的计算结果表明,对于多数问题的基态,$\Delta x\Delta p_{x}$接近于$\hbar$;而对于高激发态,$\Delta x\Delta p_{x}$的值可以很大.例如,一维谐振子的基态$\varPsi_{0}$,$\Delta x\Delta p_{x}=\frac{\hbar}{2}$,刚好是\eqref{eq38.14}式的下限;对于$\varPsi_{n}$态,$\Delta x\Delta p_{x}=\bigg(n+\frac{1}{2}\bigg)\hbar$.

不确定度关系集中反映了量子力学规律的特点,规定了经典力学轨道概念的适用限度.例如,经典力学认为质点有绝对静止状态,位置完全确定($\Delta x=0$),动量为0.而按照不确定度关系\eqref{eq38.14},这种经典静止状态是不可能的.实验事实正是这样,粒子的位置越确定($\Delta x$小),动量的涨落$\Delta p$就越大,动能的数值也越大(参看下面的例题).又如经典力学认为质点均有运动轨道,在任何时刻质点均有明确的位置和动量,$\Delta x=0,\Delta p_{x}=0$.而不确定关系\eqref{eq38.14}告诉我们, 在任何时刻粒子的$\Delta x$与$\Delta p_{x}$之积不小于$\frac{\hbar}{2}$,这就根本否定了轨道运动的概念.但是,在\eqref{eq38.14}式的限制下,如果$\Delta x$和$\Delta p_{x}$实际上都很小,可以忽略不计,则轨道概念仍可近似成立.例如电子在宏观尺度上运动,$x$的量级如取为厘米,如果$\Delta x\sim 10^{-4}\si{cm}$,这是不易观测到的,按照\eqref{eq38.14}式,$\Delta p_{x}$的下限为
\begin{empheq}{equation*}
	\Delta p_{x}\sim\frac{\hbar}{\Delta x}\sim 10^{-28}\si{kg\cdot m\cdot s^{-1}}
\end{empheq}
如果电子的动能等于1\si{eV},则动量等于
\begin{empheq}{equation*}
	p=\sqrt{2m_{e}E}\sim \num{5.4}\times 10^{-25}\si{kg\cdot m\cdot s^{-1}}
\end{empheq}
和$p$相比,$\Delta p_{x}$也是微不足道的,因此轨道概念可以近似成立,电子的宏观运动可以用经典力学来处理.电子在原子范围内运动,情况就完全不同了,这时电子的位置(从原子核处算起)的量级约为
\begin{empheq}{equation*}
	x\sim a_{0}\sim\num{0.53}\times 10^{-10}\si{m}\text{(玻尔半径)}
\end{empheq}
电子的动能盐级约为10\si{eV},因此
\begin{empheq}{equation*}
	p\sim \sqrt{2m_{e}\times10\si{eV}}\sim 1.7\times10^{-24}\si{kg\cdot m\cdot s^{-1}}	
\end{empheq}
\begin{empheq}{equation*}
	x\cdot p\sim 8.5\times 10^{-35} \si{J\cdot s}\sim\hbar	
\end{empheq}
由于\eqref{eq38.14}式的限制,已经不可能同时使$\Delta x\ll x,\Delta p\ll p$,亦即经典轨道的概念不再适用.

\example 用不确定度关系估算原子核中核子(质子,中子)动能的量级,并解释电子不可能是原子核的结构单元.

\solution 中等大小的原子核,半径$R$约为5\si{fm}左右.核子在核内的分布基本上是均匀的,就一个核子来说,
\begin{empheq}{equation*}
	\overline{r^{2}}\approx\frac{1}{\tau}\int_{\tau}r^{2}d\tau=\frac{4\pi}{4\pi R^{3}/3}\int_{0}^{R}r^{4}dr=\frac{3}{5}R^{2}	
\end{empheq}
\begin{empheq}{equation*}
	\overline{x^{2}}\approx\frac{1}{3}\overline{r^{2}}\approx\frac{1}{5}R^{2}	
\end{empheq}
对于全体核子的总平均,可取
\begin{empheq}{equation*}
	\overline{p_{x}^{2}}\sim\frac{\hbar^{2}}{x^{2}}\sim\frac{5\hbar^{2}}{R^{2}}	
\end{empheq}
\begin{empheq}{equation*}
	\overline{p^{2}}\sim3\overline{p_{x}^{2}}\sim\frac{15\hbar^{2}}{R^{2}}	
\end{empheq}
核子平均动能约为
\begin{empheq}{equation*}
	\frac{\overline{p^{2}}}{2m_{p}}\sim\frac{15\hbar^{2}}{2m_{p}R^{2}}=\frac{15\hbar^{2}c^{2}}{2m_{p}c^{2}R^{2}}\sim 12\si{MeV}	
\end{empheq}
核子动能的实验值约为$10\sim 20$\si{MeV}.

如果设想原子核的结构单元中也有电子,则电子的动量估计值仍如上述,$p^{2}\sim\frac{15\hbar^{2}}{R^{2}}$,如用公式$\frac{p^{2}}{2m_{e}}$计算动能,则
\begin{empheq}{equation*}
	\frac{p^{2}}{2m_{e}}\sim\frac{m_{p}}{m_{e}}\times 12 \si{MeV}\sim 2.2\times 10^{4} \si{MeV}
\end{empheq}
远大于$m_{e}c^{2}$.这种情况下应该用相对论公式,
\begin{empheq}{equation*}
	E\sim pc \sim\sqrt{15}\frac{\hbar c}{R}\sim 150\si{MeV}
\end{empheq}
欲将动能这么大的电子束缚在原子核中,需要量级相同的负的势能.但是能够吸引电子的只有质子-电子间的库仑吸引力,其势能约为
\begin{empheq}{equation*}
	V\sim -\frac{Ze^{2}}{R}\quad \text{(Z:核内质子总数)}
\end{empheq}
如取$Z\sim40$,则$V\sim-12\si{MeV}$,数值远小于动能估计值150\si{MeV}.由此可见,电子不可能是原子核的结构单元.

事实上,原子核发生$\beta^{-}$衰变时,临时产生出来的电子,其动能最多只有几个\si{MeV},其德布罗意波长
\begin{empheq}{equation*}
	\lambda=\frac{h}{p}\sim\frac{hc}{E}\sim\frac{1.24\times 10^{3}\si{MeV}\cdot\si{fm}}{E}
\end{empheq}
$\lambda$远大于核半径,所以$\beta^{-}$电子不可能在核内久留,几乎立刻逸出核外.




