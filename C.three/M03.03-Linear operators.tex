\section[线性算符]{线性算符} \label{sec:03.03} % 
% \makebox[5em][s]{} % 短题目拉间距

线性算符(子)是量子力学的基本数学工具,本节对此作初浅介绍,叙述力求简明,不追求论证的严格性, 以利初学者阅读.

量子力学中,可以作用于波函数并且符合下列运算法则的算符$\hat{A}$称为线性算符,
\begin{empheq}{equation}\label{eq33.1}
	\hat{A}(C_{1}\varPsi_{1}+C_{2}\varPsi_{2})=C_{1}\hat{A}\varPsi_{1}+C_{2}\hat{A}\varPsi_{2}
\end{empheq}
其中$\varPsi_{1},\varPsi_{2}$是任意波函数(表示可以实现的状态),$C_{1},C_{2}$是常数(复数).

$\S$\ref{sec:03.01}所列举的力学量算符显然都是线性算符.\eqref{eq33.1}式表明,线性算符作用于几个波函数的叠加式时,可以逐项分别作用,而对常系数则不起作用.

诸如开方,取绝对值,取复共轭等运算,则为非线性运算,如果定义出相应的算符,就是非线性算符.例如定义平方根算符$\sqrt{\quad}$,作用规则为
\begin{empheq}{equation*}
	\sqrt{\quad}\varPsi=\sqrt{\varPsi}=(\varPsi)^{\frac{1}{2}}
\end{empheq}
则
\begin{empheq}{equation*}
	\sqrt{\quad}C\varPsi=\sqrt{C\varPsi}=\sqrt{C}\sqrt{\varPsi}
\end{empheq}
常系数$C$也受到作用(被开方),所以$\sqrt{\quad}$算符为非线性算符.

以后凡是提到“算符”,均指线性算符.

算符$\hat{A},\hat{B}$之和(差)按下式定义,
\begin{empheq}{equation}\label{eq33.2}
	(\hat{A}\pm\hat{B})\varPsi=\hat{A}\varPsi\pm\hat{B}\varPsi
\end{empheq}
其中$\varPsi$是任意波函数.上式中$\hat{A}\varPsi$和$\hat{B}\varPsi$的性质都是波函数,次序可以颠倒,因此
\begin{empheq}{equation*}
	(\hat{A}+\hat{B})\varPsi=\hat{A}\varPsi+\hat{B}\varPsi=\hat{B}\varPsi+\hat{A}\varPsi=(\hat{B}+\hat{A})\varPsi
\end{empheq}
亦即
\begin{empheq}{equation}\label{eq33.3}
	\hat{A}+\hat{B}=\hat{B}+\hat{A}\quad\text{(加法交换律)}
\end{empheq}

算符$\hat{A},\hat{B}$之积仍为算符,按下式定义,
\begin{empheq}{equation}\label{eq33.4}
	(\hat{A}\hat{B})\varPsi=\hat{A}(\hat{B}\varPsi)
\end{empheq}
其中$\varPsi$为任意波函数(下同).例如
\begin{empheq}{align*}
	(\hat{p}_{x\hat{p}y})\varPsi &=\hat{p}_{x}(\hat{p}_{y}\varPsi)
	=-i\hbar\frac{\partial}{\partial x}\bigg(-i\hbar\frac{\partial}{\partial y}\varPsi\bigg)	\\
	&=-\hbar^{2}\frac{\partial^{2}}{\partial x\partial y}\varPsi
\end{empheq}
所以
\begin{empheq}{equation*}
	\hat{p}_{x}\hat{p}_{y}=-\hbar^{2}\frac{\partial^{2}}{\partial x\partial y}
\end{empheq}
显然
\begin{empheq}{equation}\label{eq33.5}
	\hat{p}_{x}\hat{p}_{y}=\hat{p}_{y}\hat{p}_{x}
\end{empheq}
又如
\begin{empheq}{align*}
	(\hat{p}_{x}\hat{x})\varPsi &=\hat{p}_{x}(\hat{x}\varPsi)
	=-i\hbar\frac{\partial}{\partial x}(x\varPsi)	\\
	&=-i\hbar\bigg(x\frac{\partial}{\partial x}\varPsi+\varPsi\bigg)
	=(\hat{x}\hat{p}_{x}-i\hbar)\varPsi
\end{empheq}
所以算符$\hat{x}\hat{p_{x}}$与$\hat{p_{x}}\hat{x}$不相等,它们的关系是
\begin{empheq}{equation}\label{eq33.6}
	\hat{x}\hat{p_{x}}-\hat{p_{x}}\hat{x}=i\hbar
\end{empheq}
这里将$i\hbar$(常数)也作为算符对待.

\eqref{eq33.5}式和\eqref{eq33.6}式代表了两种典型的算符关系,即对易和不对易.一般说,如$\hat{A}\hat{B}=\hat{B}\hat{A}$,则称$\hat{A}$与$\hat{B}$可对易,或简称$\hat{A},\hat{B}$对易;如$\hat{A}\hat{B}\neq\hat{B}\hat{A}$,则称$\hat{A}$与$\hat{B}$不(可)对易.\eqref{eq33.5}式表明,动量算符的各个分量是相互对易的.\eqref{eq33.6}式表明,$\hat{p}$与$x$(作为算符)是不对易的容易发现,动量算符和位置算符的不同分量是对易的,如
\begin{empheq}{equation}\label{eq33.7}
	\hat{x}\hat{p}_{y}=\hat{p}_{y}\hat{x},\quad \hat{y}\hat{p}_{x}=\hat{p}_{x}\hat{y}
\end{empheq}
等等.力学扯算符之间的不可对易性,是量子力学特有的概念,经典物理中没有类似的概念.

算符$\hat{A}$的$n$(正整数)次幕定义成$n$个$\hat{A}$连乘,如
\begin{empheq}{equation}\label{eq33.8}
	\hat{A}^{2}=\hat{A}\hat{A},\quad \hat{A}^{3}=\hat{A}\hat{A}\hat{A}
\end{empheq}
如再定义
\begin{empheq}{equation}\label{eq33.9}
	\hat{A}^{0}=1
\end{empheq}
显然就有
\begin{empheq}{equation}\label{eq33.10}
	\hat{A}^{n}\hat{A}^{m}=\hat{A}^{m}\hat{A}^{n}=\hat{A}^{n+m}
\end{empheq}
$n,m$为非负整数.

如存在算符$\hat{B}$,使$\hat{B}\hat{A}=\hat{A}\hat{B}=1$,则可以定义$\hat{B}=\hat{A}^{-1}$,称为$\hat{A}$的逆算符.什么样的算符存在逆算符?这问题说来话长,暂不讨论.

波函数$\varPsi$的复共轭记为$\varPsi^{*}$,即
\begin{empheq}{equation*}
	\varPsi=u+iv,\quad \varPsi^{*}=u-iv\quad\text{(}u,n\text{为实数)}
\end{empheq}

对于各种可能的状态(波函数)$\varPsi_{1},\varPsi_{2},\cdots,\varPsi_{i},\cdots$,定义算符$\hat{A}$的“矩阵元”
\begin{empheq}{equation}\label{eq33.11}
	A_{ij}=\int_{\text{全}}\varPsi_{i}^{*}\hat{A}\varPsi_{j}d\tau
\end{empheq}
显然,定义一个算符$\hat{A}$,相当于规定了全体$A_{ij}$之值.注意$A_{ij}$一般是复数.

以下引入共轭算符的概念.$\hat{A}$的共枙记为$\hat{A}^{+}$,定义如下:对于任意波函数$\varPsi_{1},\varPsi_{2},\cdots,\varPsi_{i},\cdots$,定义$\hat{A}^{+}$的“矩阵元”为
\setlength{\mathindent}{5em}
\begin{empheq}{equation}\label{eq33.12}
	A_{ij}^{+}=\int_{\text{全}}\varPsi_{i}^{*}\hat{A}^{*}\varPsi_{j}d\tau=(A_{ij})^{*}
	=\int_{\text{全}}\varPsi_{j}(\hat{A}\varPsi_{i})^{*}d\tau
\end{empheq}
这种定义形式上很抽象,但是对于一般的常用算符,利用\eqref{eq33.12}式很容易找出其共轭算符.例如:

(1) $\hat{A}=\lambda$(常数),由\eqref{eq33.12}式易见$\hat{A}^{+}=\lambda^{*}$.

(2) $\hat{A}=\frac{\partial}{\partial x}$(以一维问题为例),由\eqref{eq33.12}式可得
\begin{empheq}{align*}
	\int_{-\infty}^{\infty}\varPsi_{i}^{*}\bigg(\frac{\partial}{\partial x}\bigg)^{*}\varPsi_{j}dx
	&=\int_{-\infty}^{\infty}\varPsi_{j}\bigg(\frac{\partial}{\partial x}\varPsi_{i}\bigg)^{*}dx	\\
	&=\varPsi_{j}\varPsi_{i}^{*}\bigg|_{-\infty}^{\infty}-\int_{-\infty}^{\infty}\varPsi_{i}^{*}\frac{\partial}{\partial x}\varPsi_{j}dx	\\
	&=\int_{-\infty}^{\infty}\varPsi_{i}^{*}\bigg(-\frac{\partial}{\partial x}\bigg)\varPsi_{j}dx
\end{empheq}\eqnormal
(由于$\varPsi_{i},\varPsi_{j}$描述真实状态,$x\rightarrow\pm\infty$处$\varPsi_{i}$,$\varPsi_{j}\rightarrow 0$)比较上式两端,即得
\begin{empheq}{equation}\label{eq33.13}
	\bigg(\frac{\partial}{\partial x}\bigg)^{*}=-\frac{\partial}{\partial x}
\end{empheq}
同样应有
\begin{empheq}{equation*}\label{eq33.13'}
	\bigg(\frac{\partial}{\partial y}\bigg)^{*}=-\frac{\partial}{\partial y},\quad \bigg(\frac{\partial}{\partial z}\bigg)^{*}=-\frac{\partial}{\partial z}
	\tag{$3.3.13^{\prime}$}
\end{empheq}
因此
\begin{empheq}{equation}\label{eq33.14}
	\nabla^{*}=-\nabla
\end{empheq}

利用定义式\eqref{eq33.12},容易证明:
\begin{empheq}{equation}\label{eq33.15}
	(\hat{A}^{+})^{+}=\hat{A}
\end{empheq}

对于任意两个线性算符$\hat{A},\hat{B}$,容易证明:
\begin{empheq}{equation}\label{eq33.16}
	(\hat{A}\pm\hat{B})^{+}=\hat{A}^{+}\pm\hat{B}^{+}
\end{empheq}
\begin{empheq}{equation}\label{eq33.17}
	(\hat{A}\hat{B})^{+}=\hat{B}^{+}\hat{A}^{+}
\end{empheq}
\eqref{eq33.17}式证明如下:反复利用\eqref{eq33.12}式,可得
\setlength{\mathindent}{5em}
\begin{empheq}{align*}
	 &\int\varPsi_{i}^{*}(\hat{A}\hat{B})^{*}\varPsi_{j}d\tau
	=\bigg[\int\varPsi_{j}^{*}\hat{A}(\hat{B}\varPsi_{i})d\tau\bigg]^{*}
	=\int(\hat{B}\varPsi_{i})^{*}\hat{A}^{+}\varPsi_{j}d\tau	\\
	=&\int(\hat{A}^{+}\varPsi_{j})(\hat{B}\varPsi_{i})^{*}d\tau
	=\int\varPsi_{i}^{*}\hat{B}^{+}(\hat{A}^{+}\varPsi_{j})d\tau	\\
	=&\int\varPsi_{i}^{*}(\hat{B}^{+}\hat{A}^{+})\varPsi_{j}d\tau
\end{empheq}\eqnormal
比较两端,即得\eqref{eq33.17}式.上式中所有积分均为全空间积分.

如果$\hat{A}^{+}=\hat{A}$,则称$\hat{A}$为厄密算符或自共轭算符.利用\eqref{eq33.13}至\eqref{eq33.17}式,容易验明$\S$\ref{sec:03.01}中提到的$\boldsymbol{r},\boldsymbol{p},T,V,H,\boldsymbol{L}$等力学量算符都是厄密算符.

注意,如$\hat{A},\hat{B}$均为厄密算符,即$\hat{A}^{+}=\hat{A},\hat{B}^{+}=\hat{B}$,则($\hat{A}+\hat{B}$)是厄密算符,而$\hat{A}\hat{B}$是否为厄密算符取决于$\hat{A},\hat{B}$是否对易.如$\hat{A}\hat{B}=\hat{B}\hat{A}$,则$\hat{A}\hat{B}$是厄密算符;如$\hat{A}\hat{B}\neq\hat{B}\hat{A}$,则$\hat{A}\hat{B}$就不是厄密算符.

关于厄密算符的本征值和本征函数,可以证明下述定理:

定理(1)$\quad$厄密算符的本征值为实数.

证明:设$\hat{A}^{+}=\hat{A}$,以$a$和$\varPsi_{a}$表示$\hat{A}$的一个本征值和相应的本征函数,满足本征方程:
\begin{empheq}{equation}\label{eq33.18}
	\hat{A}\varPsi_{n}=a\varPsi_{a}
\end{empheq}
以$\varPsi_{n}^{*}$左乘上式,并对全空间积分,得到
\setlength{\mathindent}{5em}
\begin{align*}
	a\int\varPsi_{n}^{*}\varPsi_{n}d\tau
	&=\int\varPsi_{a}^{*}\hat{A}\varPsi_{a}d\tau=\in\varPsi_{a}^{*}\hat{A}^{*}\varPsi_{n}d\tau
	\intertext{利用\eqref{eq33.13}式,}
	&=\bigg(\int\varPsi_{a}^{*}\hat{A}\varPsi_{a}d\tau\bigg)^{*}=a^{*}\int\varPsi_{a}^{*}\varPsi_{a}d\tau
\end{align*}\eqnormal
比较两端,即得$a=a^{*}$,即$a$为实数.

两个波函数$\varPsi_{1},\varPsi_{2}$如果满足关系:
\begin{empheq}{equation}\label{eq33.19}
	\int_{\text{全}}=\varPsi_{1}^{*}\varPsi_{2}d\tau=0
\end{empheq}
则称$\varPsi_{1},\varPsi_{2}$“正交”.

定理(2)$\quad$厄密算符的任何两个对应于不同本征值的本征函数互相正交.

证明:设$\hat{A}=\hat{A}^{+}$,以$a_{1},a_{2}$表示两个本征值,$\varPsi_{1},\varPsi_{2}$为相应的本征函数,满足本征方程:
\begin{empheq}{equation*}
	\hat{A}\varPsi_{1}=a_{1}\varPsi_{1}
\end{empheq}
\begin{empheq}{equation*}
	\hat{A}\varPsi_{2}=a_{2}\varPsi_{2}
\end{empheq}
以$\varPsi_{1}^{*}$左乘第一式,对全空间积分,得到
\begin{empheq}{equation}\label{eq33.20}
	a_{1}\int\varPsi_{2}^{*}\varPsi_{1}d\tau=\int\varPsi_{2}^{*}\hat{A}\varPsi_{1}d\tau
\end{empheq}
以$\varPsi_{1}^{*}$左乘第二式,也对全空间积分,得到
\begin{empheq}{equation*}
	a_{2}\int\varPsi_{1}^{*}\varPsi_{2}d\tau=\int\varPsi_{1}^{*}\hat{A}\varPsi_{2}d\tau
\end{empheq}
取复共轭,并利用\eqref{eq33.13}式,即得
\setlength{\mathindent}{6em}
\begin{empheq}{equation}\label{eq33.20'}
	a_{2}\int\varPsi_{2}^{*}\varPsi_{1}d\tau
	=\bigg(\int\varPsi_{1}^{*}\hat{A}\varPsi_{1}d\tau\bigg)^{*}
	=\int\varPsi_{2}^{*}\hat{A}\varPsi_{1}d\tau
	\tag{$3.3.20^{\prime}$}
\end{empheq}\eqnormal
\eqref{eq33.20}和\eqref{eq33.20'}式右端相等,由于已设$a_{1}\neq a_{2}$,所以
\begin{empheq}{equation*}
	\int_{\text{全}}\varPsi_{2}^{*}\varPsi_{1}d\tau=0
\end{empheq}
即$\varPsi_{1},\varPsi_{2}$正交.

有时,厄密算符($\hat{A}$)的某个本征值($a_{n}$)对应的独立本征函数不止一个.设与本征值$a_{n}$对应的本征函数为$\varPsi_{n1},\varPsi_{n2},\cdots,\varPsi_{nf}$,($f$称为本征值$a_{n}$的简并度)它们互相线性独立,但是可能不正交,可以证明(略),将这$f$个简并态线性叠加,总能重新组成$f$个互相独立而且互相正交的新态,它们仍然是相应于本征值$a_{n}$的本征态.有此了解,以后凡是提到某个厄密算符的全体本征函数,一律理解成是互相正交的.

\example 设波函数中$\varphi_{1},\varphi_{2}$均已归一化,但是不正交,
\begin{empheq}{equation*}
	\int_{\text{全}}\varphi_{1}^{*}\varphi_{2}d\tau=b\text{(实数)},\quad 0<b<1
\end{empheq}
试将$\varphi_{1},\varphi_{2}$线性叠加,组成两个正交归一化的波函数$\varPsi_{1},\varPsi_{2}$.

\solution 设$\varPsi_{1}=C_{1}\varphi_{1}+C_{2}\varphi_{2}$,$\varPsi_{2}=C_{3}\varphi_{1}+C_{4}\varphi_{2}$.按题目要求,$\varPsi_{1},\varPsi_{2}$正交,即
\setlength{\mathindent}{4em}
\begin{empheq}{align*}
	0&=\int_{\text{全}}\varPsi_{1}^{*}\varPsi_{2}d\tau
	=\int_{\text{全}}(C_{1}^{*}\varphi_{1}^{*}+C_{2}^{*}\varphi_{2}^{*})(C_{3}\varphi_{1}+C_{4}\varphi_{2})d\tau	\\
	&=C_{1}^{*}C_{3}+C_{2}^{*}C_{4}+(C_{1}^{*}C_{4}+C_{2}^{*}C_{3})b
\end{empheq}
有许多选择可以满足上式.如不管归一化条件,则$C_{1},C_{2},C_{3},C_{4}$中可以任意给定3个(但最多有一个等于0).例如:

(i) $\varPsi_{1}=\varphi_{1}+\varphi_{2}$,$\varPsi_{2}=\varphi_{1}-\varphi_{2}$.相当于$C_{1}=C_{2}=C_{3}=1,C_{4}=-1$.归一化后,成为
\begin{empheq}{equation}\label{eq33.21}
	\begin{aligned}
		\varPsi_{1}&=[2(1+b)]^{-\frac{1}{2}}(\varphi_{1}+\varphi_{2})
		\varPsi_{2}&=[2(1-b)]^{-\frac{1}{2}}(\varphi_{1}-\varphi_{2})
	\end{aligned}
\end{empheq}\eqnormal

(ii) $\varPsi_{1}=\varphi_{1}$,$\varPsi_{2}=b\varphi_{1}-\varphi_{2}$.相当于$C_{1}=1,C_{2}=0,C_{3}=b,C_{4}=-1$.归一化后,成为
\setlength{\mathindent}{6em}
\begin{empheq}{equation}\label{eq33.22}
	\varPsi_{1}=\varphi_{1},\quad \varPsi_{2}=(1-b^{2})^{-\frac{1}{2}}(b\varphi_{1}-\varphi_{2})
\end{empheq}

(iii) $\varPsi_{1}=\varphi_{1}-b\varphi_{2}$,$\varPsi_{2}=\varphi_{2}$.相当于$C_{1}=1,C_{2}=-b,C_{3}=0,C_{4}=1$.归一化后,成为
\begin{empheq}{equation}\label{eq33.23}
	\varPsi_{1}=(1-b^{2})^{-\frac{1}{2}}(\varphi_{1}-b\varphi_{2}),\quad \varPsi_{2}=\varphi_{2}
\end{empheq}\eqnormal

