\begin{exercises}
	
\exercise $n\times n$矩阵$M$的“迹”定义为$\tr M=\sum_{k=1}^{n}M_{kk}$,即对角矩阵元之和证明:任意2阶矩阵$M$可以表示成泡利矩阵$\sigma_{x},\sigma_{y},\sigma_{z}$与单位矩阵$I$的线性叠加,如下式:
\eqlong
\begin{empheq}{align*}
	M &=\frac{1}{2}(\tr M)I+\frac{1}{2}(\tr\boldsymbol{\sigma}M)\cdot\boldsymbol{\sigma}	\\
	&= \frac{1}{2} \{(\tr M)I+(\tr\sigma_{x}M)\sigma_{x}+(\tr\sigma_{y}M)\sigma_{y}+(\tr\sigma_{z}M)\sigma_{z}\}
\end{empheq}\eqnormal
 
\exercise 在$S_{z}$表象中,求$\sigma_{x}$的本征函数及$\sigma_{y}$的本征函数.
 	
\exercise 在$S_{z}$表象中,求$\sigma_{n}$的矩阵表示及其本征函数.$\sigma_{n}=\boldsymbol{n}\cdot\boldsymbol{\sigma}$,$\boldsymbol{n}$为任意$(\theta,\varphi)$方向单位矢量.
 	
\exercise 对于上题求得两个$\sigma_{n}$本征函数,计算$\langle\boldsymbol{\sigma}\rangle$,并求$\Delta S_{x},\Delta S_{y},\Delta S_{z}$,验证不确定关系.
 	
\exercise 证明$(\sigma_{x}+i\sigma_{y})^{2}=0,(\sigma_{x}-i\sigma_{y})^{2}=0$.用$(\sigma_{x}\pm i\sigma_{y})$的矩阵表示($S_{z}$表象)验证这两个公式.
 	
\exercise 讨论下列算符是否存在.如存在,将其表示成$I,\sigma_{x},\sigma_{y},\sigma_{z}$的线性叠加.

(a)$(1+\sigma_{x})^{\frac{1}{2}}$\quad(b)$(1+\sigma_{x}+i\sigma_{y})^{\frac{1}{2}}$\quad (c)$(1+\sigma_{z})^{-1}$
 
\exercise 证明公式$e^{i\lambda\sigma_{z}}=\cos\lambda+i\sigma_{z}\sin\lambda$.($\lambda$为实参数,$\sigma_{z}$为算符)并在$S_{z}$表象中证明
\begin{empheq}{equation*}
	e^{i\lambda\sigma_{z}}=\begin{bmatrix}
		e^{i\lambda} & 0 	\\
		0 & e^{-i\lambda}	\\
	\end{bmatrix}
\end{empheq}
 
\exercise 设$\boldsymbol{A}$为常数矢量,令$\boldsymbol{A}\cdot\boldsymbol{\sigma}=A\sigma_{A}$.证明
\begin{empheq}{equation*}
	e^{i\boldsymbol{A}\cdot\boldsymbol{\sigma}}=\cos A+i\sigma_{A}\sin A
\end{empheq}
 	
\exercise 设$\boldsymbol{A},\boldsymbol{B},\boldsymbol{C}$为常数矢量,求$\tr(\boldsymbol{A}\cdot\boldsymbol{\sigma}),\tr[(\boldsymbol{A}\cdot\boldsymbol{\sigma})(\boldsymbol{B}\cdot\boldsymbol{\sigma})]$及$\tr[(\boldsymbol{A}\cdot\boldsymbol{\sigma})(\boldsymbol{B}\cdot\boldsymbol{\sigma})(\boldsymbol{C}\cdot\boldsymbol{\sigma})]$.
 	
\exercise 证明$\tr(e^{i\boldsymbol{A}\cdot\boldsymbol{\sigma}})=2\cos A$,$\tr(e^{i\lambda\sigma_{x}} e^{i\lambda\sigma_{y}}=2\cos^{2}A)$.
 	
\exercise 证明(a) $\sigma_{z}(\sigma_{x}\pm i\sigma_{y})=(\sigma_{x}\pm i\sigma_{y})(\sigma_{z}\pm 2)$

(b) $f(\sigma_{z})(\sigma_{x}\pm i\sigma_{y})=(\sigma_{x}\pm i\sigma_{y})f(\sigma_{z}\pm 2)$

(c) $e^{i\lambda\sigma_{z}}(\sigma_{x}\pm i\sigma_{y})=e^{\pm2i\lambda}(\sigma_{x}\pm i\sigma_{y})e^{i\lambda\sigma_{z}}$
 	
\exercise 证明$e^{i\lambda\sigma_{z}}\sigma_{x}e^{-i\lambda\sigma_{z}}=\sigma_{x}\cos2\lambda-\sigma_{y}\sin2\lambda$

$e^{i\lambda\sigma_{z}}\sigma_{y}e^{-i\lambda\sigma_{z}}=\sigma_{x}\sin2\lambda+\sigma_{y}\cos2\lambda$
 	
\exercise 令$\hbar=1$,对于电子,证明
\eqlong
\begin{empheq}{equation*}
	(2\boldsymbol{S}\cdot\boldsymbol{L}+1)^{2}=\boldsymbol{J}^{2}+\frac{1}{4},\quad (\boldsymbol{\sigma}\cdot\boldsymbol{J})(\boldsymbol{\sigma}\cdot\boldsymbol{J}-1)=\boldsymbol{J}^{2}
\end{empheq}\eqnormal
 	
\exercise 对于电子的$(\boldsymbol{L}^{2},\boldsymbol{J}^{2},J_{z})$共同本征态$\varPsi_{ljm_{j}}$,对于属于同一个$l$值的那些状态,定义算符
\begin{empheq}{equation*}
	\Lambda_{l}^{+}=\frac{1}{2l+1}(l+1+\boldsymbol{\sigma}\cdot\boldsymbol{L}),\quad \Lambda_{l}^{-}=\frac{1}{2l+1}(l-\boldsymbol{\sigma}\cdot\boldsymbol{L})
\end{empheq}
试确定它们对$\varPsi_{ljm_{j}}$的作用规则,并找出它们满足的基本代数关系.
 	
\exercise  (a) 证明$\sigma_{r}=\dfrac{\boldsymbol{\sigma}\cdot\boldsymbol{r}}{r}$与$\boldsymbol{J}$的各分量对易,即$[\sigma_{r},\boldsymbol{J}]=0$

(b) 对于$\varPsi_{ljm_{j}}$,将$l=j-\dfrac{1}{2}$者记为$\phi_{jm_{j}}^{A}$,$l=j+\dfrac{1}{2}$者记为$\phi_{jm_{j}}^{B}$,证明
\begin{empheq}{equation*}
	\sigma_{r}\phi_{jm_{j}}^{A}=-\phi_{jm_{j}}^{B},\quad \sigma_{r}\phi_{jm_{j}}^{B}=-\phi_{jm_{j}}^{A}
\end{empheq}
并加以解释.

[提示:利用附录\ref{A04}\eqref{eqA4.45}式、\eqref{eqA4.46}式,并注意问题的宇称性]

 	
\exercise 电子的总磁矩算符是
\begin{empheq}{equation*}
	\boldsymbol{\mu}=\boldsymbol{\mu}_{L}+\boldsymbol{\mu}_{s}=-\dfrac{e}{2m_{e}c}(\boldsymbol{L}+2\boldsymbol{S})
\end{empheq}
试对于$\varPsi_{ljj}$态$(m_{j}=j)$计算$\mu_{z}$平均值(结果用量子数$j$表示).
 
\exercise 某电子态波函数为$\Psi(r,\theta,\varphi,S_{z})=R(r)Y_{l0}(\theta,\varphi)\chi_{1/2}(S_{z})$.这是否$\boldsymbol{J}^{2}$或$J_{z}$本征函数?试计算$\langle\boldsymbol{J}^{2}\rangle$,从而求出$\boldsymbol{J}^{2}$的可能测值及相应概率.
 	
\exercise 自旋为0的带电粒子在均匀磁场中运动,哈密顿算符为
\begin{empheq}{equation*}
	\hat{H}=\frac{m}{2}\hat{\boldsymbol{v}}^{2}=\frac{1}{2m}\bigg(\hat{\boldsymbol{p}}-\frac{q}{c}\boldsymbol{A}\bigg)^{2}
\end{empheq}

设磁场沿$z$轴方向,矢势取为$\boldsymbol{A}=\dfrac{1}{2}\boldsymbol{B}\times\boldsymbol{r}$,即
\begin{empheq}{equation*}
	A_{x}=-\frac{By}{2},\quad A_{y}=\frac{Bx}{2},\quad A_{z}=0
\end{empheq}

(a) 计算$v_{x},v_{y},v_{z}$之间的对易式

(b) 设粒子电荷$q>0$.证明$\hat{H}$可以表示成
\begin{empheq}{equation*}
	\hat{H}=\frac{\hbar qB}{2mc}(\hat{Q}^{2}+\hat{P}^{2})+\frac{1}{2m}\hat{p}_{z}^{2}
\end{empheq}
确定$\hat{Q},\hat{P}$,证明$\hat{Q}\hat{P}-\hat{P}\hat{Q}$.

(c) 利用谐振子的结果,确定本题的能谱.
 	
\exercise 将上题的$\hat{H}$表示成\eqref{eq74.7'}式的形式,找出主要的守恒力学量,再找出能级公式.(质量$m$改为$\mu$,以免与量子数混淆.)
 	
\exercise 有一个定域电子(不考虑“轨道”运动),受到沿正$x$方向的均匀磁场作用,磁作用势为
\begin{empheq}{equation*}
	H=\frac{eB}{m_{e}c}S_{x}=\frac{e\hbar B}{2m_{e}c}\sigma_{x}
\end{empheq}
设$t=0$时电子的自旋状态为$\sigma_{z}=1$的本征态,即$\chi_{1/2}=\left[\begin{smallmatrix}
	1 \\ 0 
\end{smallmatrix}\right]$.求任意$t>0$时自旋波函数$\chi(t)=\left[\begin{matrix}
	a(t)	\\	b(t)
\end{matrix}\right]$,并求$\overline{S_{x}}(t),\overline{S_{y}}(t),\overline{S_{z}}(t)$.
 
\exercise 用海森伯运动方程处理上题,求$\overline{S_{x}}(t),\overline{S_{y}}(t),\overline{S_{z}}(t)$.注意先确定$t=0$时它们的初始值.
 	
\exercise 对于氢原子,求自旋轨道耦合作用引起的能级分裂$\Delta E_{nl}$[\eqref{eq73.15}式、\eqref{eq73.12}式]

[提示:$\langle r^{-3}\rangle$利用5-15题结果.]
 
\exercise 设氢原子中电子处于s态$(l=0)$,考虑电子与质子的自旋角动量,$S_{1}$与$S_{2}$.设电子自旋态为$\alpha$,质子自旋态为$\beta$,即总自旋态为$\alpha(1)\beta(2)$.求总自旋平方$\boldsymbol{S}^{2}=(\boldsymbol{S}_{1}+\boldsymbol{S}_{2})^{2}$的可能取值及相应概率.
 	
\exercise 同上题,如电子处于$S_{1z}=\dfrac{\hbar}{2}$的状态(即$\chi_{1/2}$),质子处于$S_{2x}=\dfrac{\hbar}{2}$的状态,求总自旋平方$\boldsymbol{S}^{2}$及总$S_{z}=S_{1z}+S_{2z}$的可能取值及相应概率.
 	
\exercise 考虑由3个自旋$\dfrac{1}{2}$的可分辨粒子组成的体系.

(a) 求总自旋平方$\boldsymbol{S}^{2}=(\boldsymbol{S}_{1}+\boldsymbol{S}_{2}+\boldsymbol{S}_{3})^{2}$的本征值.

(b) 设体系能量算符(略去“轨道”运动)为
\begin{empheq}{equation*}
	H=(\boldsymbol{S}_{1}\cdot\boldsymbol{S}_{2}+\boldsymbol{S}_{2}\cdot\boldsymbol{S}_{3}+\boldsymbol{S}_{3}\cdot\boldsymbol{S}_{1})\omega_{0}
\end{empheq}
求能谱,并说明各能级的简并度.
 	
\exercise 两个角动量耦合问题,设$j_{1}=j_{2}=1$,则$m_{1}$及$m_{2}$均有1,0,-1三种取值,相应的本征态记为$\alpha,\beta,\gamma$,$|j_{1}=1,m_{1}=1\rangle$记为$\alpha(1)$,$|j_{2}=1,m_{2}=0\rangle$记为$\beta(2)$,依此类推.$\boldsymbol{J}_{1}^{2},\boldsymbol{J}_{2}^{2},\boldsymbol{J}^{2},J_{z}$共同本征态记为$\varPsi_{jM}$.对于本题$(j_{1}=j_{2}=1)$,共有多少种$\varPsi_{jM}$?将它们在无耦合表象中求出来.(计算时,取$\hbar=1$.)

[ 提示:利用$J_{1+}\gamma(1)=\sqrt{2}\beta(1),J_{1+}\beta(1)=\sqrt{2}\alpha(1),J_{1+}\alpha(1)=0$,等等.$\boldsymbol{J}^{2}=\boldsymbol{J}_{1}^{2}+\boldsymbol{J}_{2}^{2}+2\boldsymbol{J}_{1z}\boldsymbol{J}_{2z}+J_{1+}J_{2-}+J_{1-}J_{2+},\boldsymbol{J}_{1}^{2}=\boldsymbol{J}_{2}^{2}=2$.]
 	
\exercise 对于二电子体系的自旋三重态和单态$(\chi_{00})$,证明它们都是$\boldsymbol{\sigma_{1}}\cdot\boldsymbol{\sigma_{2}}$的本征态,本征值分别为1和-3.
 
\exercise 同上题,验证$\chi_{00}$是$\sigma_{1x}\sigma_{2x},\sigma_{1y}\sigma_{2y},\sigma_{1z}\sigma_{2z}$的共同本征态,本征值均为-1.
 	
\exercise 对于自旋单态$\chi_{00}$,求$(\boldsymbol{a}\cdot\boldsymbol{\sigma_{1}})(\boldsymbol{b}\cdot\boldsymbol{\sigma_{2}})$的平均值.其中$\boldsymbol{a},\boldsymbol{b}$为常矢量.

\exercise 对于自旋单态和三重态$\chi_{00},\chi_{11},\chi_{10},\chi_{1-1}$,找出使它们互相转化的算符(共12种)建议不要用角动量升降算符,而用尽量简单的、存在逆算符的单体或二体算符.
	
\end{exercises}