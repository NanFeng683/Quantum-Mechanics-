\section[电子的总角动量]{电子的总角动量} \label{sec:07.02} % 
% \makebox[5em][s]{} % 短题目拉间距

电子的总角动量$\boldsymbol{J}$为轨道角动量$\boldsymbol{L}=\boldsymbol{r}\times\boldsymbol{p}$与自旋角动量$\boldsymbol{S}$之和,
\begin{empheq}{equation}\label{eq72.1}
	\boldsymbol{J}=\boldsymbol{L}+\boldsymbol{S}
\end{empheq}
即
\begin{empheq}{equation*}\label{eq72.1'}
	J_{\alpha}=L_{\alpha}+S_{\alpha},\quad \alpha=x,y,z
\end{empheq}
$\boldsymbol{L}$与$\boldsymbol{S}$属于不同自由度,相应的算符互相对易,即
\begin{empheq}{equation}\label{eq72.2}
	[L_{\alpha},S_{\beta}]=0,\quad \alpha,\beta=x,y,z
\end{empheq}
因此
\begin{empheq}{align}\label{eq72.3}
	[J_{x},J_{y}]&=[L_{x},L_{y}]+[S_{x},S_{y}]	\nonumber\\
	&=i\hbar(L_{z}+S_{z})=i\hbar J_{z}
\end{empheq}
等等总之,$\boldsymbol{J}$仍满足$\S$\ref{sec:04.07}所述角动量的对易式
\begin{empheq}{equation}\label{eq72.4}
	\boldsymbol{J}\times\boldsymbol{J}=i\hbar\boldsymbol{J}
\end{empheq}
由于$\boldsymbol{L},\boldsymbol{S}$互相对易,$\boldsymbol{J}^{2}$可以写成
\eqlong
\begin{empheq}{align}\label{eq72.5}
	\boldsymbol{J}^{2}&=\boldsymbol{J}\cdot\boldsymbol{J}=(\boldsymbol{L}+\boldsymbol{S})\cdot(\boldsymbol{L}+\boldsymbol{S})	\nonumber\\
	&=\boldsymbol{L}^{2}+\boldsymbol{S}^{2}+2\boldsymbol{S}\cdot\boldsymbol{L}=\boldsymbol{L}^{2}+\boldsymbol{S}^{2}+\hbar\boldsymbol{\sigma}\cdot\boldsymbol{L}
\end{empheq}\eqnormal
其中
\begin{empheq}{equation}\label{eq72.6}
	\boldsymbol{\sigma}\cdot\boldsymbol{L}=\sigma_{x}L_{x}+\sigma_{y}L_{y}+\sigma_{z}L_{z}
\end{empheq}
这是一个重要算符.它和$\boldsymbol{\sigma}$各分量的对易式可以利用\eqref{eq71.27}式而直接写出(令$\boldsymbol{A}=\boldsymbol{L}$),
\begin{empheq}{equation}\label{eq72.7}
	[\boldsymbol{\sigma},\boldsymbol{\sigma}\cdot\boldsymbol{L}]=2i\boldsymbol{L}\times\boldsymbol{\sigma}=-2i\boldsymbol{\sigma}\times\boldsymbol{L}
\end{empheq}
$\boldsymbol{\sigma}\cdot\boldsymbol{L}$与$\boldsymbol{L}$各分量的对易式可以利用$\boldsymbol{L}\times\boldsymbol{L}=i\hbar\boldsymbol{L}$导出,例如
\eqlong
\begin{empheq}{align*}
	[L_{x},\boldsymbol{\sigma}\cdot\boldsymbol{L}]&=\sigma_{y}[L_{x},L_{y}]+\sigma_{z}[L_{x},L_{y}]	\\
	&=i\hbar(\sigma_{y}L_{z}-\sigma_{z}L_{y})=i\hbar(\boldsymbol{\sigma}\times\boldsymbol{L})_{x}
\end{empheq}\eqnormal
亦即
\begin{empheq}{equation}\label{eq72.8}
	[\boldsymbol{L},\boldsymbol{\sigma}\cdot\boldsymbol{L}]=i\hbar\boldsymbol{\sigma}\times\boldsymbol{L}
\end{empheq}
综合\eqref{eq72.7}、\eqref{eq72.8}式,易得
\begin{empheq}{equation}\label{eq72.9}
	[\boldsymbol{J},\boldsymbol{\sigma}\cdot\boldsymbol{L}]=0
\end{empheq}
$\boldsymbol{J}$的各分量显然与$\boldsymbol{L}^{2}$对易,也与$\boldsymbol{S}^{2}$对易$\bigg(\boldsymbol{S}^{2}=\frac{3\hbar^{2}}{4}$为常数$\bigg)$.这样,$\boldsymbol{L}^{2},\boldsymbol{J}^{2},J_{z}$(或$J_{x},J_{y}$)三者互相对易,存在共同本征函数.按照常规,本征值和本征函数可以由本征方程同步解出.在这里我们将充分考虑电子自旋的特殊性,先求出本征值,然后再求本征函数.

$\boldsymbol{L}^{2}$的本征值已在$\S$\ref{sec:04.07}求出,为
\begin{empheq}{equation}\label{eq72.10}
	\boldsymbol{L}^{2}=l(l+1)\hbar^{2},\quad l=0,1,2,\cdots
\end{empheq}
为了求出$\boldsymbol{J}^{2}$的本征值,只须求$\boldsymbol{\sigma}\cdot\boldsymbol{L}$的本征值[参看\eqref{eq72.5}式].为此,利用\eqref{eq71.31}式.\eqref{eq72.5}式告诉我们,$\boldsymbol{J}^{2},\boldsymbol{L}^{2}$的共同本征态也是$\boldsymbol{\sigma}\cdot\boldsymbol{L}$的本征态.将\eqref{eq71.31}式作用于这个共同本征态,式中各算符就转化成本征值.因此$\boldsymbol{\sigma}\cdot\boldsymbol{L}$的本征值满足方程
\begin{empheq}{equation}\label{eq72.11}
	(\boldsymbol{\sigma}\cdot\boldsymbol{L})^{2}+\hbar\boldsymbol{\sigma}\cdot\boldsymbol{L}-l(l+1)\hbar^{2}=0
\end{empheq}
亦即
\begin{empheq}{equation*}
	(\boldsymbol{\sigma}\cdot\boldsymbol{L}-l\hbar)[\boldsymbol{\sigma}\cdot\boldsymbol{L}+(l+1)\hbar]=0
\end{empheq}
解为
\begin{empheq}{equation}\label{eq72.12}
	\boxed{\begin{aligned}
			\boldsymbol{\sigma}\cdot\boldsymbol{L}&\text{(本征值)}=l\hbar,	\\
			-&(l+1)\hbar
	\end{aligned}}
\end{empheq}
[当$l=0$,算符$\boldsymbol{L}$对$\boldsymbol{L}^{2}$本征函数$(Y_{00})$的作用效果为0,$\boldsymbol{\sigma}\cdot\boldsymbol{L}$的本征值为0,而不能取\eqref{eq72.12}式中第二个值.]将\eqref{eq72.12}式代入\eqref{eq72.5}式,并取$\boldsymbol{L}^{2}=l(l+1)\hbar^{2}$,$\boldsymbol{S}^{2}=\frac{3\hbar^{2}}{4}$,就得到$\boldsymbol{J}^{2}$的本征值.结果是
\eqlong
\begin{empheq}{equation}\label{eq72.13}
	\boxed{\begin{aligned}
		&\boldsymbol{J}^{2}\text{(本征值)}=j(j+1)\hbar^{2}	\\
		\boldsymbol{\sigma}\cdot\boldsymbol{L}=l\hbar&,\quad j=l+\frac{1}{2}\quad(l=0,1,2,\cdots)	\\
		\boldsymbol{\sigma}\cdot\boldsymbol{L}=-(l+&1)\hbar,j=l-\frac{1}{2}\quad(l=1,2,3,\cdots)
	\end{aligned}}
\end{empheq}\eqnormal
按照角动量普遍理论($\S$\ref{sec:04.07}),对于每一种$\boldsymbol{J}^{2}$本征值,$\boldsymbol{J}_{z}$本征值(记为$m\hbar$)有$(2j+1)$种:
\begin{empheq}{equation}\label{eq72.14}
	m_{j}=j,j-1,\cdots,(-j)
\end{empheq}
注意,$j$和$m_{j}$都是整半数(整数加$\frac{1}{2}$).许多书中都采用符号$m_{j}=m+\frac{1}{2}$,$m$为整数.[这里$m$仅是量子数$m_{j}$的一种表示符号,与$L_{z}$的本征值没有关系.事实上,$\boldsymbol{L}^{2},\boldsymbol{J}^{2},J_{z}$的共同本征态中,绝大部分都不是$L_{z}$的本征态,$L_{z}$没有明确的(单一的)取值.]

下面讨论$\boldsymbol{L}^{2},\boldsymbol{J}^{2},J_{z}$的共同本征函数.顾及自旋自由度,电子波函数的一般形式可以写成
\eqlong
\begin{empheq}{equation}\label{eq72.15}
	\varPsi(\boldsymbol{r},S_{z},t)=\varPsi_{1}(\boldsymbol{r},t)\chi_{1/2}(S_{z})+\varPsi_{2}(\boldsymbol{r},t)\chi_{-1/2}(S_{z})
\end{empheq}
其中$\varPsi_{1}$表示$S_{z}=\frac{\hbar}{2}$时$\varPsi$的函数值,$\varPsi_{2}$表示$S_{z}=-\frac{\hbar}{2}$时$\varPsi$的函数值.如采用$(x,y,z,S_{z})$表象,上式可以表示成
\begin{empheq}{equation*}\label{eq72.15'}
	\varPsi(\boldsymbol{r},S_{z},t)=\varPsi_{1}(\boldsymbol{r},t)\begin{bmatrix}
		1 \\ 0
	\end{bmatrix}+\varPsi_{2}(\boldsymbol{r},t)\begin{bmatrix}
		1 \\ 0
	\end{bmatrix}=\begin{bmatrix}
		\varPsi_{1}(\boldsymbol{r},t)	\\	\varPsi_{2}(\boldsymbol{r},t)
	\end{bmatrix}	\tag{$7.2.15^{\prime}$}
\end{empheq}
归一化条件为
\begin{empheq}{equation}\label{eq72.16}
	\iiint_{-\infty}^{\infty}\varPsi^{*}\varPsi dxdydz=\int(\varPsi_{1}^{*}\varPsi_{2}+\varPsi_{2}^{*}\varPsi_{1})d\tau=1
\end{empheq}\eqnormal
其中$\int\cdots d\tau$即$\iiint_{-\infty}^{\infty}dxdydz$(全空间积分).如$(x,y,z)$空间采用球坐标$(r,\theta,\varphi)$,则$\varPsi_{1},\varPsi_{2}$均为$(r,\theta,\varphi)$及时间$t$的函数.因角动量算符$\boldsymbol{L},\boldsymbol{J}$与径向距离$r$无关,研究$\boldsymbol{L}^{2},\boldsymbol{J}^{2},J_{z}$共同本征函数时,可暂不考虑$\varPsi$与$r$的关系,而将本征函数表示成
\eqlong
\begin{empheq}{align}\label{eq72.17}
	\varPsi(\theta,\varphi,S_{z})&=\varPsi_{1}(\theta,\varphi)\chi_{1/2}(S_{z})+\varPsi_{2}(\theta,\varphi)\chi_{-1/2}(S_{z})	\nonumber\\
	&=\begin{bmatrix}
		\varPsi_{1}(\theta,\varphi) \\ \varPsi_{2}(\theta,\varphi)
	\end{bmatrix}
\end{empheq}\eqnormal
归一化条件\eqref{eq72.16}式中$\int\cdots d\tau$也相应改成$\int\cdots d\Omega,d\Omega=\sin\theta d\theta d\varphi$.

作为$\boldsymbol{L}^{2}$的本征函数[本征值$(l(l+1)$],\eqref{eq72.17}式中也与也显然应该是$l$值相同的球谐函数.试令
\begin{empheq}{equation*}
	\varPsi=C_{1}Y_{lm_{1}}(\theta\varphi)\chi_{1/2}+C_{2}Y_{lm2}\chi_{-1/2}
\end{empheq}
作为$J_{z}$的本征函数,$\varPsi$应该满足本征方程
\begin{empheq}{equation*}
	J_{z}\varPsi=(L_{z}+S_{z})\varPsi=m_{j}\hbar\varPsi=\bigg(m+\frac{1}{2}\bigg)\hbar\varPsi
\end{empheq}
容易看出,为此只须$m_{1}=m,m_{2}=m+1$,因此
\begin{empheq}{equation}\label{eq72.18}
	\varPsi=C_{1}Y_{lm}\chi_{1/2}+C_{2}Y_{lm+1}\chi_{-1/2}
\end{empheq}
现在只须确定系数$C_{1},C_{2}$.$\varPsi$作为$\boldsymbol{J}^{2}$的本征函数,应该满足$\boldsymbol{\sigma}\cdot\boldsymbol{L}$的本征方程,$\boldsymbol{\sigma}\cdot\boldsymbol{L}$的本征值见\eqref{eq72.12}式.$\boldsymbol{\sigma}\cdot\boldsymbol{L}$由\eqref{eq72.6}式表示,作用于$\varPsi$时,利用附录\ref{A04}\eqref{eqA4.43}式及\eqref{eq71.17}式,有下列结果
\eqllong
\begin{empheq}{align*}
	\sigma_{z}L_{z}Y_{lm}&\chi_{1/2}=m\hbar Y_{lm}\chi_{1/2}	\\
	\sigma_{z}L_{z}Y_{lm+1}\chi_{-1/2}&=-(m+1)\hbar Y_{lm+1}\chi_{-1/2}	\\
	(\sigma_{x}L_{x}+\sigma_{y}L_{y})Y_{lm}\chi_{1/2}&=(L_{x}+iL_{y})Y_{lm}\chi_{-1/2}	\\
	&=\sqrt{(l+m+1)(l-m)}\hbar Y_{lm+1}\chi_{-1/2}	\\
	(\sigma_{x}L_{x}+\sigma_{y}L_{y})Y_{lm+1}\chi_{-1/2}&=(L_{x}-iL_{y})Y_{lm+1}\chi_{1/2}	\\
	&=\sqrt{(l+m+1)(l-m)}\hbar Y_{lm}\chi_{1/2}
\end{empheq}
总之,$\sigma_{z}L_{z}$对\eqref{eq72.18}式中二项作用的结果,各得一个本征值;而$(\sigma_{x}L_{x}+\sigma_{y}L_{y})$作用于\eqref{eq72.18}式,使其中二项互相转化.

将\eqref{eq72.18}式代入$\boldsymbol{\sigma}\cdot\boldsymbol{L}$的本征方程,对两种本征值分别作出计算,就可以求出$\frac{C_{1}}{C_{2}}$,再利用归一化条件,就可定出$C_{1},C_{2}$.结果如下:

(i) $\boldsymbol{\sigma}\cdot\boldsymbol{L}=l\hbar,\bigg(j=l+\frac{1}{2}\bigg)$
\begin{empheq}{equation*}
	\frac{C_{1}}{C_{2}}=\bigg(\frac{l+m+1}{l-m}\bigg)^{1/2},\quad C_{1}=\bigg(\frac{l+m+1}{2l+1}\bigg)^{1/2},\quad C_{2}=\bigg(\frac{l-m}{2l+1}\bigg)^{1/2}
\end{empheq}

(ii) $\boldsymbol{\sigma}\cdot\boldsymbol{L}=-(l+1)\hbar,\bigg(j=l-\frac{1}{2}\bigg)$
\begin{empheq}{equation*}
	\frac{C_{1}}{C_{2}}=-\bigg(\frac{l-m}{l+m+1}\bigg)^{1/2},\quad C_{1}=-\bigg(\frac{l-m}{2l+1}\bigg)^{1/2},\quad C_{2}=\bigg(\frac{l+m+1}{2l+1}\bigg)^{1/2}
\end{empheq}\eqnormal
这样得到的$(\boldsymbol{L}^{2},\boldsymbol{J}^{2},J_{z})$共同本征函数,通常记为$\varPsi_{ljm_{j}}$,它们是

(i) $j=l+\frac{1}{2},\quad m_{j}=m+\frac{1}{2}$
\eqindent{3}
\begin{empheq}{align}\label{eq72.19}
	&\varPsi_{ljm_{j}}=\bigg(\frac{l+m+1}{2l+1}\bigg)^{1/2}Y_{lm}\chi_{1/2}+\bigg(\frac{l-m}{2l+1}\bigg)^{1/2}Y_{lm+1}\chi_{-1/2}	\nonumber\\
	=&\frac{1}{\sqrt{2l+1}}\begin{bmatrix}
		\sqrt{l+m+1}Y_{lm}	\\	\sqrt{l-m}Y_{lm+1}
	\end{bmatrix}
\end{empheq}

(ii) $j=l-\frac{1}{2},\quad m_{j}=m+\frac{1}{2}$
\begin{empheq}{align}\label{eq72.20}
	&\varPsi_{ljm_{j}}=-\bigg(\frac{l-m}{2l+1}\bigg)^{1/2}Y_{lm}\chi_{1/2}+\bigg(\frac{l+m+1}{2l+1}\bigg)^{1/2}Y_{lm+1}\chi_{-1/2}	\nonumber\\
	=&\frac{1}{\sqrt{2l+1}}\begin{bmatrix}
		-\sqrt{l-m}Y_{lm}	\\	\sqrt{l+m+1}Y_{lm+1}
	\end{bmatrix}
\end{empheq}\eqnormal
二者物理性质的主要区别是,前者$\boldsymbol{S}\cdot\boldsymbol{L}\geqslant 0$,后者$\boldsymbol{S}\cdot\boldsymbol{L}<0$,因此习惯上称前者为$\boldsymbol{S},\boldsymbol{L}$“平行”态,后者为$\boldsymbol{S},\boldsymbol{L}$“反平行”态.\eqref{eq72.20}式中两项的系数$(C_{1},C_{2})$正负号相反,通常规定$C_{1}$取负值,这是为了和角动量耦合理论中的相因子规定取得一致.

\example 对于$\varPsi_{ljm_{j}}$态,计算$\boldsymbol{\sigma}$各分量的平均值.

\solution 对于任何由$\boldsymbol{S},\boldsymbol{L}$构成的算符,平均值公式是

\begin{empheq}{equation}\label{eq72.21}
	\langle \boldsymbol{F} \rangle=\int\varPsi^{+}\hat{F}\varPsi d\Omega 
\end{empheq}
当$\sigma_{x},\sigma_{y}$作用于$\varPsi_{ljm_{j}}$时,总是将$\chi_{1/2}$变成$\chi_{-1/2}$,$\chi_{-1/2}$变成$\chi_{1/2}$,例如$\sigma_{x}$作用于\eqref{eq72.18}式,得到
\begin{empheq}{align*}
	\sigma_{x}\varPsi_{ljm_{j}}&=C_{1}Y_{lm}\sigma_{x}\chi_{1/2}+C_{2}Y_{lm+1}\sigma_{x}\chi_{-1/2}	\\
	&=C_{1}Y_{lm}\chi_{-1/2}+C_{2}Y_{lm+1}\chi_{1/2}
\end{empheq}
这里的每一项均与\eqref{eq72.18}式中每一项正交.$\sigma_{y}$也是这样.因此
\begin{empheq}{equation}\label{eq72.22}
	\langle \sigma_{x}\rangle=\langle\sigma_{y}\rangle=0
\end{empheq}
对于$\sigma_{z}$,结果就不同了,由于$\chi_{1/2}$及$\chi_{-1/2}$是$\sigma_{z}$的本征函数,我们可得
\begin{empheq}{equation*}
	\sigma_{z}\varPsi_{ljm_{j}}=C_{1}Y_{lm}\chi_{1/2}-C_{2}Y_{lm+1}\chi_{-1/2}=\begin{bmatrix}
		C_{1}Y_{lm} \\ -C_{2}Y_{lm+1}
	\end{bmatrix}
\end{empheq}
因此
\begin{empheq}{align*}
	\langle\sigma_{z}\rangle&=\int\varPsi_{ljm_{j}}^{+}\sigma_{z}\varPsi_{ljm_{j}}d\Omega	\\
	&=\int[C_{1}^{*}Y_{lm}^{*}\quad C_{2}^{*}Y_{lm+1}^{*}]\begin{bmatrix}
		C_{1}Y_{lm}	\\	-C_{2}Y_{lm+1}
	\end{bmatrix}d\Omega	\\
	&=\int(C_{1}^{*}C_{1}Y_{lm}^{*}Y_{lm}-C_{2}^{*}C_{2}Y_{lm+1}^{*}Y_{lm+1})d\Omega	\\
	&=C_{1}^{*}C_{1}-C_{2}^{*}C_{2}
\end{empheq}
对于$j=l+\frac{1}{2}$的情形,结果是
\begin{empheq}{equation}\label{eq72.23}
	\langle\sigma_{z}\rangle=\frac{2m+1}{2l+1}=\frac{m_{j}}{j}
\end{empheq}
对于$j=l-\frac{1}{2}$的情形,结果是
\begin{empheq}{equation}\label{eq72.24}
	\langle\sigma_{z}\rangle=-\frac{2m+1}{2l+1}=\frac{m_{j}}{j+1}
\end{empheq}

本题也可以避开$\varPsi_{ljm_{j}}$的具体构造式,而利用$\boldsymbol{\sigma},\boldsymbol{L}$的算符公式以及$\boldsymbol{\sigma}\cdot\boldsymbol{L}$的本征值公式,计算出结果.考虑到这种算符方法有推广价值,介绍如下.由\eqref{eq71.26}式,取$\boldsymbol{A}=\boldsymbol{L}$,
\begin{empheq}{equation}\label{eq72.25}
	\boldsymbol{\sigma}(\boldsymbol{\sigma}\cdot\boldsymbol{L})+(\boldsymbol{\sigma}\cdot\boldsymbol{L})\sigma=2\boldsymbol{L}
\end{empheq}
在$\varPsi_{ljm_{j}}$态下求上式平均值,其中$\boldsymbol{\sigma}\cdot\boldsymbol{L}$变成本征值,因此得到
\begin{empheq}{equation}\label{eq72.26}
	\langle\boldsymbol{\sigma}\rangle(\boldsymbol{\sigma}\cdot\boldsymbol{L}\text{本征值})=\langle\boldsymbol{L}\rangle 
\end{empheq}
由于$\boldsymbol{J}=\boldsymbol{L}+\frac{\hbar}{2}\boldsymbol{\sigma}$,上式亦即
\begin{empheq}{equation}\label{eq72.27}
	\langle\boldsymbol{\sigma}\rangle \bigg[\frac{\hbar}{2}+(\boldsymbol{\sigma}\cdot\boldsymbol{L}\text{本征值})\bigg]=\langle\boldsymbol{J}\rangle 
\end{empheq}
在$J_{z}$的本征态下,$\langle J_{x}\rangle=\langle J_{y}\rangle=0$,所以$\langle\sigma_{x}\rangle=\langle\sigma_{y}\rangle=0$.而
\eqindent{2}
\begin{empheq}{equation}\label{eq72.28}
	\frac{\hbar}{2}+(\boldsymbol{\sigma}\cdot\boldsymbol{L}\text{本征值})=
	\begin{dcases}
		\bigg(l+\frac{1}{2}\bigg)\hbar	\\
		-\bigg(l+\frac{1}{2}\bigg)\hbar
	\end{dcases}=
	\begin{dcases}
		\quad j\hbar,\quad\qquad	j=l+\frac{1}{2}	\\
		-(j+1)\hbar,\quad j=l-\frac{1}{2}
	\end{dcases}
\end{empheq}\eqnormal
代入\eqref{eq72.27}式,即得\eqref{eq72.23}式、\eqref{eq72.24}式.
