\section[与时间有关的微扰论]{与时间有关的微扰论} \label{sec:09.01} % 
% \makebox[5em][s]{} % 短题目拉间距

考虑一个物理体系(例如一个原子),其能量算符为$H_{0}$(不显含$t$),$H_{0}$的正交归一化的本征函数记为$\varPsi_{n}(x)$,相应的能级记为$E_{n}$.[$x$代表波函数所涉及的全体独立变量,也可以理解为某种守恒量完全集($H_{0}$在内)的共同本征函数,$n$代表全体量子数.]设开始时体系处于定态$\varPsi_{k}$.如果没有外界作用,体系将继续处于$\varPsi_{k}$态,波函数的时间变化表现为一个位相因子,
\begin{empheq}{equation}\label{eq91.1}
	\varPsi_{k}(x,t)=\varPsi_{k}(x)e^{-iE_{k}t/\hbar}
\end{empheq}
这就是定态.

设$t>0$时体系受到外界作用,作用势$H^{\prime}(x,t)$,即体系的总能量算符变成
\eqshort
\begin{empheq}{equation}\label{eq91.2}
	H=H_{0}+H^{\prime}
\end{empheq}\eqnormal
设$[H_{0},H^{\prime}]\neq0$,因此$t>0$时$H_{0}$不再是守恒量.与此相应,波函数满足薛定谔方程:
\begin{empheq}{equation}\label{eq91.3}
	i\hbar\frac{\partial}{\partial t}\varPsi(x,t)=H\varPsi=(H_{0}+H^{\prime})\varPsi(x,t)
\end{empheq}
按照态叠加原理,$\varPsi(x,t)$可以表示成$H_{0}$的本征函数的线性叠加,即
\begin{empheq}{equation}\label{eq91.4}
	\varPsi(x,t)=\sum_{n}C_{n}(t)\varPsi_{n}(x)e^{-iE_{n}t/\hbar}
\end{empheq}
初始条件为
\begin{empheq}{equation}\label{eq91.5}
	\varPsi(x,0)=\varPsi_{k}(x)\quad\text{即}\quad C_{n}(0)=\delta_{nk}
\end{empheq}
如求出了各$C_{n}(t)$,也就是求出了$\varPsi(x,t)$.设在$t=T$时去除外界作用$H^{\prime}$,并随即测量体系的能量,即测量$H_{0}$,按照波函数的普遍概率解释,测得$H_{0}=E_{f}$的概率为$C_{f}^{*}(T)C_{f}(T)$,这也就是$t=T$时体系处于$\varPsi_{f}$态的概率.或者说,$C_{f}^{*}(T)C_{f}(T)$就是到时刻$T$为止体系已由原先的$\varPsi_{k}$态跃迁到$\varPsi_{f}$态的概率.$C_{f}^{*}C_{f}$的时间变化率称为由$\varPsi_{k}$态变到$\varPsi_{f}$态的跃迁速率,记为
\begin{empheq}{equation}\label{eq91.6}
	w_{k\rightarrow f}(t)=\frac{d}{dt}[C_{f}^{*}(t)C_{f}(t)]
\end{empheq}
$\varPsi_{k}$称为初态,$\varPsi_{f}$称为终态.

为了求出$C_{f}(t)$,将\eqref{eq91.4}式代入\eqref{eq91.3}式,得到
\begin{empheq}{equation*}
	i\hbar\sum_{n}\frac{dC_{n}}{dt}\varPsi_{n}e^{-iE_{n}t/\hbar}=\sum_{n}H^{\prime}\varPsi_{n}C_{n}e^{-iE_{n}t/\hbar}
\end{empheq}
以$\varPsi_{f}^{\prime}$左乘上式,对全空间积分,并注意利用正交归一化条件
\begin{empheq}{equation}\label{eq91.7}
	\int\varPsi_{f}^{*}(x)\varPsi_{n}(x)dx=\delta_{fn}
\end{empheq}
可得
\begin{empheq}{equation}\label{eq91.8}
	i\hbar\frac{dC_{f}}{dt}e^{-E_{f}t/\hbar}=\sum_{n}H_{fn}^{\prime}C_{n}e^{-iE_{n}t/\hbar}
\end{empheq}
其中
\begin{empheq}{equation}\label{eq91.9}
	H_{fn}^{\prime}=\int\varPsi_{f}^{*}H^{\prime}\varPsi_{n}dx=\langle \varPsi_{f}|H^{\prime}|\varPsi_{n} \rangle 
\end{empheq}
是$H_{0}$表象中$H^{\prime}$的矩阵元,它与时间$t$有关.

\eqref{eq91.8}式是严格的,它代表一组联立方程$(f=1,2,\cdots)$,如能严格解出,当然很好.但一般\eqref{eq91.8}式不易严格解出,需用近似解法.本节只介绍微扰论解法,条件是$H^{\prime}$较弱而且作用时间也不长,时刻$t$时体系已由初态$\varPsi_{k}$跃迁到各个可能终态的总概率远小于1,即
\begin{empheq}{equation}\label{eq91.10}
	\sum_{n}^{\prime}C_{n}^{*}(t)C_{n}(t)\ll 1
\end{empheq}
在这条件下可以略去\eqref{eq91.8}式右端所有$n\neq k$的$C_{n}$,并取$C_{k}(t)\approx1$,从而将\eqref{eq91.8}式近似为
\begin{empheq}{equation}\label{eq91.11}
	i\hbar\frac{d}{dt}C_{f}=H_{fk}^{\prime}(t)e^{i\omega_{fk}t}
\end{empheq}
积分,即得满足初始条件\eqref{eq91.5}式的解为
\begin{empheq}{equation}\label{eq91.12}
	\boxed{C_{f}(t)=\frac{1}{i\hbar}\int_{0}^{t}H_{fk}^{\prime}e^{i\omega_{fk}t}dt}
\end{empheq}
其中$\omega_{fk}=\frac{E_{f}-E_{k}}{\hbar}$.这个结果相当于视$H^{\prime}$为微扰而求出的一级近似.如将\eqref{eq91.12}式再代入\eqref{eq91.8}式右端(令$f\rightarrow n$),就可求出$C_{f}(t)$的二级近似.不过通常只取一级近似,即\eqref{eq91.12}式,这是本章的基本公式.

从\eqref{eq91.12}式可知,如果$H_{fk}^{\prime}(t)=0(0<t<T)$则$C_{f}(t)=0$,即由$\varPsi_{k}$态到$\varPsi_{f}$态的跃迁是禁戒的.为了使跃迁$\varPsi_{k}\rightarrow\varPsi_{f}$成为可能,须$H_{fk}^{\prime}\neq0$,为此而出现的量子数$f$与$n$间的制约关系即所谓选择定则.

如果考虑反方向的跃迁过程,即初态为$\varPsi_{f}$,终态为$\varPsi_{k}$,重复上述计算,$C_{k}$的公式显然只需在\eqref{eq91.12}式中将$f$、$k$互换.由于$H^{\prime}$应该是厄密的$(H^{\prime}=H^{\prime+})$,必有$H_{kf}^{\prime}=(H_{fk}^{\prime})^{*}$,因此现在问题中的$-C_{k}(t)$等于\eqref{eq91.12}式决定的$C_{f}^{*}(t)$.这就是说,在同一种外界作用下($H^{\prime}$相同),$\varPsi_{k}\rightarrow\varPsi_{f}$及$\varPsi_{f}\rightarrow\varPsi_{k}$这两种正、反跃迁过程的跃迁速率相等.
\pskip

\example 有一个量子力学体系,总能量算符$H_{0}$的本征函数(已正交归一化)为$\varPsi_{n}(x)$,能级$E_{n},n=1,2,\cdots$.已知$t<0$时体系处于基态$\varPsi_{k}$,$t>0$时受到外来微扰$H^{\prime}(x,t)=F(x)e^{-t/\tau}$的作用.试用微扰论(一级近似)求$t\gg\tau$时$(t\rightarrow\infty)$体系处于各激发态$(\varPsi_{n},E_{n}>E_{1})$的概率.

\solution 微扰$H^{\prime}$的矩阵元($H_{0}$表象)为
\begin{empheq}{align*}
	H_{n1}^{\prime}(t) &=e^{-t/\tau}\int\varPsi_{n}^{*}(x)F(x)\varPsi_{1}(x)dx	\\
	&=F_{n1}e^{-t/\tau}
\end{empheq}
如$F_{n1}\neq0$,$H^{\prime}$将引起由$\varPsi_{k}$向$\varPsi_{n}$的跃迁,所求概率等于$C_{n}^{*}(\infty)C_{n}(\infty)$.按照\eqref{eq91.12}式,
\begin{empheq}{align*}
	C_{n}(t) &=\frac{F_{n1}}{i\hbar}\int_{0}^{t}\exp\left(i\omega_{n1}t-\frac{t}{\tau}\right)dt	\\
	&=\frac{\frac{F_{n1}}{i\hbar}\left[\exp\left(i\omega_{n1}t-\frac{t}{\tau}\right)-1\right]}{i\omega_{n1}-\frac{1}{\tau}}
\end{empheq}
当$t\rightarrow\infty(t\gg\tau)$,即得
\begin{empheq}{equation*}
	C_{n}(\infty)=\frac{F_{n1}}{E_{n}-E_{1}+\frac{i\hbar}{\tau}}
\end{empheq}
体系处于$\varPsi_{n}$态的概率为
\begin{empheq}{equation*}
	|C_{n}(\infty)|^{2}=\frac{|F_{n1}|^{2}}{(E_{n}-E_{1})^{2}+\frac{\hbar^{2}}{\tau^{2}}}
\end{empheq}
微扰论适用条件为$\sum^{\prime}|C_{n}|^{2}\gg1(n\neq1)$,为此必须$|C_{n}|^{2}\gg1$,即
\begin{empheq}{equation*}
	|F_{n1}|\gg E_{n}-E_{1}\quad\text{或}\quad |F_{n1}|\gg\frac{\hbar}{\tau}
\end{empheq}







