\section[量子力学公式及其矩阵表示]{量子力学公式及其矩阵表示} \label{sec:04.02} % 
% \makebox[5em][s]{} % 短题目拉间距

代表力学量(可观测量)的算符,如$\S$\ref{sec:03.04}所述,必须是厄密算符,其本征矢有正交归一性和完备性,因此可以选作态矢空间的基矢组.

选定一个力学量完全集$Q$,代表它的厄密算符记为$\hat{Q}$,其本征值记为$q_{n}$,本征矢记为$|\varPsi_{n} \rangle $,简记为$|n \rangle $,满足本征方程
\begin{empheq}{equation}\label{eq42.1}
	\hat{Q}|n \rangle=q_{n}|n \rangle ,\quad \langle n|\hat{Q}=q_{n}\langle n|
\end{empheq}
正交归一条件
\begin{empheq}{equation}\label{eq42.2}
	\langle \varPsi_{n}|\varPsi_{k} \rangle=\langle n|k \rangle =\delta_{nk}
\end{empheq}
以及综合了本征矢的正交归一性与完备性的恒等变换公式
\begin{empheq}{equation}\label{eq42.3}
	\sum_{n}|\varPsi_{n} \rangle\langle \varPsi_{n}|=\sum_{n}|n \rangle\langle n|=1  
\end{empheq}
以$\{|n \rangle,n=1,2,\cdots\}$作为基矢组,任何态矢$|\varPsi \rangle $可以表示成
\eqindent{4}
\begin{empheq}{equation}\label{eq42.4}
	|\varPsi \rangle =\sum_{n}|n \rangle\langle n|\varPsi \rangle =\sum_{n}C_{n}|n \rangle,\quad
	C_{n}\langle n|\varPsi \rangle 
\end{empheq}\eqnormal
如$|\varPsi \rangle $已经归一化,$\langle \varPsi|\varPsi \rangle=1$,就有
\begin{empheq}{equation}\label{eq42.5}
	\langle \varPsi|\varPsi \rangle =\sum_{n}C_{n}^{*}C_{n}=1
\end{empheq}
其中
\eqindent{6}
\begin{empheq}{equation}\label{eq42.6}
	C_{n}^{*}C_{n}=\text{在$|\varPsi\rangle$态中力学量$Q$取值为$q_{n}$的概率}
\end{empheq}\eqnormal
这正是态矢$|\varPsi \rangle $的概率解释.力学量$Q$的平均值公式可以表示成
\begin{empheq}{align}\label{eq42.7}
	\bar{Q}&=\sum_{n}q_{n}C_{n}C_{n}^{*}=\sum_{n}q_{n}C_{n}\langle \varPsi|n \rangle	\nonumber\\
	&=\langle \varPsi|\sum_{n}C_{n}\hat{Q}|n \rangle =\langle \varPsi|\hat{Q}\sum_{n}C_{n}|n \rangle 	\nonumber\\
	&= \langle \varPsi|\hat{Q}|n \rangle 
\end{empheq}
如利用基矢$|n \rangle $的投影算符
\begin{empheq}{equation}\label{eq42.8}
	\hat{P}_{n}=|n \rangle\langle n|
\end{empheq}
\eqref{eq42.6}式中概率$C_{n}^{*}C_{n}$可以表示成
\eqindent{6}
\begin{empheq}{equation}\label{eq42.9}
	C_{n}^{*}C_{n}=\langle \varPsi|n \rangle \langle n|\varPsi \rangle =\langle \varPsi|\hat{P}_{n}|\varPsi \rangle =\bar{P}_{n}
\end{empheq}
这是投影算符的又一层含义.

在$|\varPsi \rangle $的展开式\eqref{eq42.4}中,如将各$C_{n}$排成一个列矢量(单列矩阵),记为$\varPsi(Q)$,即定义
\begin{empheq}{equation}\label{eq42.10}
	\varPsi(Q)=\begin{bmatrix}
		C_{1}\\ \vdots\\ C_{n}\\ \vdots
	\end{bmatrix}
	,\quad
	\varPsi^{*}(Q)=\begin{bmatrix}
		C_{1}^{*} & \cdots & C_{n}^{*} & \cdots
	\end{bmatrix}
\end{empheq}
$\varPsi^{*}(Q)$是$\varPsi(Q)$的共轭矩阵,是一个行矢量.按照矩阵运算法则,
\begin{empheq}{equation}\label{eq42.11}
	\varPsi^{+}(Q)\varPsi(Q)=\begin{bmatrix}
		\cdots & C_{n}^{*} & \cdots
	\end{bmatrix}	\begin{bmatrix}
		\vdots \\ C_{n} \\ \vdots
	\end{bmatrix}
\end{empheq}
$\varPsi(Q)$作为列矢量,称为态矢$|\varPsi\rangle$在$Q$表象中的矩阵表示.$\varPsi^{*}(Q)$作为行矢量,则为$\langle \varPsi|$在$Q$表象中的矩阵表示.

$\varPsi(Q)$也可以称为$|\varPsi\rangle$在$Q$表象中的“波函数”($\hat{Q}$的本征值的函数),其中$C_{n}=\varPsi(q_{n})$.

各基矢$|n \rangle $是$\hat{Q}$的本征矢,它们在$Q$表象中的矩阵表示(波函数)显然应该是
\begin{empheq}{equation}\label{eq42.12}
	\varPsi_{1}(Q)=\begin{bmatrix}
		1 \\ 0 \\ 0 \\ \vdots
	\end{bmatrix},\cdots,\quad
	\varPsi_{n}(Q)=\begin{bmatrix}
		\vdots \\ 0 \\ 1 \\ 0 \\ \vdots
	\end{bmatrix}\cdots
	\text{第$n$行}	
\end{empheq}
展开式\eqref{eq42.4}的矩阵表示为
\begin{empheq}{equation*}
	\varPsi(Q)=\begin{bmatrix}
		C_{1}\\ \vdots\\ C_{n}\\ \vdots
	\end{bmatrix}=C_{1}\varPsi_{1}(Q)+\cdots+C_{n}\varPsi_{n}(Q)+\cdots
\end{empheq}\eqnormal
相当于第三章讲过的波函数按本征函数的展开式.

{\heiti 线性算符的矩阵表示}

设有任意线性算符$\hat{A}$作用于任意态矢$|\varPsi \rangle $,将其转变成$|\phi \rangle $,即设
\begin{empheq}{equation}\label{eq42.13}
	\hat{A}|\varPsi \rangle =|\phi \rangle 
\end{empheq}
$|\varPsi\rangle$及其$Q$表象中的波函数$\varPsi(Q)$由\eqref{eq42.4}、\eqref{eq42.10}式表示.类似地,$|\phi \rangle $及其波函数$\phi(Q)$表示成
\eqindent{6}
\begin{empheq}{equation}\label{eq42.14}
	|\phi \rangle =\sum_{n}b_{n}|n \rangle ,\quad \phi(Q)=\begin{bmatrix}
		b_{1} \\ \vdots \\ b_{n} \\ \vdots
	\end{bmatrix},\quad
	b_{n=\langle n|\phi \rangle }
\end{empheq}
$Q$表象中算符$\hat{A}$的矩阵表示记为$[A]$,它应该可以与列矢量和行矢量相乘,而有
\begin{empheq}{equation*}
	[A]\begin{bmatrix}
		\cdot \\ \cdot \\ \cdot \\ \cdot
	\end{bmatrix}\rightarrow\begin{bmatrix}
		\cdot \\ \cdot \\ \cdot \\ \cdot
	\end{bmatrix},\quad
	\begin{bmatrix}
		\cdot & \cdot & \cdot & \cdot & \cdot
	\end{bmatrix}[A]\rightarrow\begin{bmatrix}
		\cdot & \cdot & \cdot & \cdot & \cdot
	\end{bmatrix}
\end{empheq}
具体说,与\eqref{eq42.13}式相应,$[A]$,$\varPsi(Q)$,$\phi(Q)$间应该有关系
\begin{empheq}{equation}\label{eq42.15}
	[A]\varPsi(Q)=[A]\begin{bmatrix}
		C_{1}\\ \vdots\\ C_{n}\\ \vdots
	\end{bmatrix}=\begin{bmatrix}
		b_{1}\\ \vdots\\ b_{n}\\ \vdots
	\end{bmatrix}=\phi(Q)
\end{empheq}\eqnormal
易见,符合上述要求的$[A]$应该是行、列数相同的矩阵,即方阵.$[A]$的$k$行$n$列矩阵元记为$A_{kn}$,可以证明
\begin{empheq}{equation}\label{eq42.16}
	A_{kn}=\langle k|\hat{A}|n \rangle 
\end{empheq}
证明如下.\eqref{eq42.15}式中列矢量$\phi(Q)$的第$k$个矩阵元是
\begin{empheq}{equation}\label{eq42.17}
	b_{k}=\sum_{n}A_{kn}C_{n}
\end{empheq}
另一方面,以基左矢$\langle k|$乘\eqref{eq42.13}式,得到
\begin{empheq}{align*}
	\langle k|\phi \rangle &=b_{k}=\langle k|\hat{A}|\varPsi \rangle =\sum_{n}\langle k|\hat{A}|n \rangle \langle n|\varPsi \rangle 	\\
	&=\sum_{n}\langle k|\hat{A}|n \rangle C_{n}
\end{empheq}
比较二式,即得\eqref{eq42.16}式.

\eqref{eq42.13}式的共轭方程是
\begin{empheq}{equation*}\label{eq42.13'}
	\langle \varPsi|\hat{A}^{+}=\langle \phi|	\tag{$4.2.13^{\prime}$}
\end{empheq}
其矩阵表示应该就是\eqref{eq42.15}式的共轭方程,即
\eqindent{3}
\begin{empheq}{equation*}\label{eq42.15'}
	\varPsi^{+}(Q)[A^{+}]=[C_{1}^{*}\cdots C_{n}^{*}\cdots][A^{+}]=
	[b_{1}^{*}\cdots b_{n}^{*}\cdots]=\phi^{*}(Q)
	\tag{$4.2.15^{\prime}$}
\end{empheq}\eqnormal
则
\begin{empheq}{equation*}
	b_{k}^{*}=\sum_{n}C_{n}^{+}A_{nk}^{+}
\end{empheq}
与\eqref{eq42.17}式的复共轭$(^{*})$比较,即得
\begin{empheq}{equation}\label{eq42.18}
	A_{nk}^{+}=(A_{kn})^{*}
\end{empheq}
这正是共轭矩阵的定义.

如$\hat{F}$是厄密算符,$\hat{F}=\hat{F}^{+}$,则在$Q$表象中其矩阵表示应该是厄密矩阵,即$[F]=[F]^{+},F_{nk}=(F_{kn})^{*}$.

$Q$表象中$\hat{Q}$本身的矩阵元是
\begin{empheq}{equation}\label{eq42.19}
	Q_{kn}=\langle k|\hat{Q}|n \rangle =q_{n}\langle k|n \rangle =q_{n}\delta_{kn}
\end{empheq}
矩阵$[Q]$是对角化方阵,对角矩阵元等于$\hat{Q}$的各本征值,非对角矩阵元全部为零,即
\eqindent{6}
\begin{empheq}{equation*}\label{eq42.19'}
	[Q]=\begin{bmatrix}
		q_{1} & 0 & 0 & \cdots & 0 & \cdots	\\
		0 & q_{2} & 0 & \cdots & 0 & \cdots	\\
		0 & 0 & q_{3} & \cdots & 0 & \cdots	\\
		\vdots & \vdots & \vdots & \quad & \vdots & 	\\
		0 & 0 & 0 & \cdots & q_{n} & \cdots	\\
		\vdots & \vdots & \vdots & \quad & \vdots & 	\\
	\end{bmatrix}	\tag{$4.2.19^{\prime}$}
\end{empheq}\eqnormal

{\heiti 平均值公式及其矩阵表示}

平均值公式\eqref{eq42.7}显然适用于任何可观测量$F$.即,对于任何归一化的状态$\varPsi$,$F$的平均值为
\begin{empheq}{equation}\label{eq42.20}
	\bar{F}=\langle \varPsi|\hat{F}|\varPsi \rangle 
\end{empheq}
利用\eqref{eq42.3}、\eqref{eq42.4}式及\eqref{eq42.10}、\eqref{eq42.16}式,可将上式化成矩阵形式,如下:
\eqindent{6}
\begin{empheq}{align}\label{eq42.21}
	\bar{F}&=\langle \varPsi|\hat{F}|\varPsi \rangle =\sum_{n}\sum_{k}C_{n}^{*}\langle n|\hat{F}|k \rangle C_{k}	\nonumber\\
	&=\sum_{n}\sum_{k}C_{n}^{*}F_{nk}C_{k}=[C_{1}^{*}\cdots C_{n}^{*}\cdots][F]\begin{bmatrix}
		C_{1} \\ \vdots \\ C_{n} \vdots
	\end{bmatrix}	\nonumber\\ &=\varPsi^{+}(Q)[F]\varPsi(Q)
\end{empheq}\eqnormal
这就是$\bar{F}$在$Q$表象中的矩阵表示式.从\eqref{eq42.20}式$\rightarrow$\eqref{eq42.7}式,相当于
\begin{empheq}{equation*}
	\langle \varPsi|\rightarrow \varPsi^{+}(Q),\quad |\varPsi \rangle \rightarrow \varPsi(Q),\quad \hat{F}\rightarrow [F]
\end{empheq}
\newpage
{\heiti 本征方程的矩阵表示}

求解力学量算符本征方程
\begin{empheq}{equation}\label{eq42.22}
	\hat{F}|\varPsi_{\lambda} \rangle =\lambda|\varPsi_{\lambda} \rangle 
\end{empheq}
的问题,也可以在$Q$表象中用矩阵方法来解决.$|\varPsi_{\lambda} \rangle $仍表示成\eqref{eq42.4}式,波函数$\varPsi_{\lambda}(Q)$仍由\eqref{eq42.10}式表示,各$C_{n}$待定.\eqref{eq42.22}式的矩阵表示为
\begin{empheq}{equation}\label{eq42.23}
	[F]\varPsi_{\lambda}(Q)=\lambda\varPsi_{\lambda}(Q)
\end{empheq}
亦即
\eqindent{6}
\begin{empheq}{equation*}\label{eq42.23'}
	\begin{bmatrix}
		F_{11}-\lambda & F_{12} & \cdots & F_{1n} & \cdots	\\
		F_{21} & F_{22}-\lambda & \cdots & F_{2n} & \cdots	\\
		\vdots & \vdots &  & \vdots & 	\\
		F_{n1}-\lambda & F_{n2} & \cdots & F_{nn}-\lambda & \cdots	\\
		\vdots & \vdots &  & \vdots & 	\\
	\end{bmatrix}\begin{bmatrix}
		C_{1} \\ C_{2} \\ \vdots \\ C_{n} \\ \vdots
	\end{bmatrix}=0	\tag{$4.2.23^{\prime}$}
\end{empheq}\eqnormal
这就是本征方程\eqref{eq42.22}式的矩阵表示.其中$[F]$的矩阵元均在$Q$表象中定义,$F_{kn}=\langle k|\hat{F}|n \rangle$.只要先求出$Q$表象中$\hat{F}$的矩阵表示,即求出各$F_{kn}$,由\eqref{eq42.23'}式即可解出本征值$\lambda$及各$C_{n}$,亦即解出$\lambda$及$\varPsi_{\lambda}(Q)$.具体说,\eqref{eq42.23'}式存在非平庸解(各$C_{n}$不全为零)的充要条件为
\eqindent{0}
\begin{empheq}{equation}\label{eq42.24}
	\det(F-\lambda I)=0,\text{即}\begin{bmatrix}
		F_{11}-\lambda & F_{12} & \cdots & F_{1n} & \cdots	\\
		F_{21} & F_{22}-\lambda & \cdots & F_{2n} & \cdots	\\
		\vdots & \vdots &  & \vdots & 	\\
		F_{n1}-\lambda & F_{n2} & \cdots & F_{nn}-\lambda & \cdots	\\
		\vdots & \vdots &  & \vdots & 	\\
	\end{bmatrix}=0
\end{empheq}\eqnormal
由此解出$\lambda=\lambda_{1},\lambda_{2},\cdots$将每个$\lambda$值代回\eqref{eq42.23'}式,求出相应的$C_{n}(n=1,2,\cdots)$,再代回\eqref{eq42.10}式,就得本征函数$\varPsi_{\lambda}(Q)$.

{\heiti 表象变换}

态矢空间的基矢组可以有多种选择.如另选一个力学量完全集$\boldsymbol{R}$,代表它的厄密算符记为$\hat{R}$,其正交归一的本征矢记为
\begin{empheq}{equation*}
	|\phi_{\nu} \rangle,\text{简记为}|\nu \rangle,\quad \nu=1,2,\cdots,\mu,\cdots
\end{empheq}
满足
\begin{empheq}{equation}\label{eq42.25}
	\langle \phi_{\nu}|\phi_{\mu} \rangle=\langle \nu|\mu \rangle =\delta_{\nu\mu}
\end{empheq}
\begin{empheq}{equation}\label{eq42.26}
	\sum_{\nu}|\phi_{\nu} \rangle\langle \phi_{\nu}|=\sum_{\nu}|\nu \rangle\langle \nu|=1 
\end{empheq}
以$\{|\phi_{\nu} \rangle \}$作为基矢组,任意态矢$|\varPsi \rangle $可以表示成
\begin{empheq}{equation}\label{eq42.27}
	|\varPsi \rangle=\sum_{\nu}a_{\nu}|\phi_{\nu} \rangle,\quad a_{\nu}=\langle \phi_{\nu}|\varPsi \rangle 
\end{empheq}
由各$a_{\nu}$组成的列矢量就是$|\varPsi \rangle $在$R$表象中的波函数:
\begin{empheq}{equation}\label{eq42.28}
	\varPsi(R)=\begin{bmatrix}
		a_{1} \\ \vdots \\ a_{\nu} \\ \vdots
	\end{bmatrix},\quad \varPsi^{+}(R)=[a_{1}^{*}\cdots a_{\nu}^{*}\cdots]
\end{empheq}
任意线性算符$\hat{A}$在$R$表象中的矩阵表示记为$A(R)$,其矩阵元为
\begin{empheq}{equation}\label{eq42.29}
	A_{\mu\nu}=\langle \phi_{\mu}|\hat{A}|\phi_{\nu} \rangle=\langle \mu|\hat{A}|\nu \rangle 
\end{empheq}
前述$Q$表象中$|\varPsi \rangle $的波函数符号仍用$\varPsi(Q)$,见\eqref{eq42.10}式.$Q$表象中$\hat{A}$的矩阵表示现在改记为$A(Q)$,其矩阵元由\eqref{eq42.16}式表示:
\begin{empheq}{equation*}
	A_{kn}=\langle \varPsi_{k}|\hat{A}|\varPsi_{n} \rangle =\langle k|\hat{A}|\nu \rangle 
	\tag{$4.2.16$}
\end{empheq}
试问,同一个态矢$|\varPsi \rangle $在两种表象中的矩阵表示即波函数$\varPsi(Q)$、$\varPsi(R)$之间有什么关系?同一个算符$\hat{A}$在两种表象中的矩阵表示$A(Q)$、$A(R)$之间又有什么关系?不妨设想$\varPsi(Q)$、$\varPsi(R)$可以用一个变换矩阵$S$联系起来:
\begin{empheq}{equation}\label{eq42.30}
	\varPsi(R)=S\varPsi(Q)
\end{empheq}
亦即
\eqindent{6}
\begin{empheq}{equation*}\label{eq42.30'}
	\begin{bmatrix}
		a_{1} \\ \vdots \\ a_{\nu} \\ \vdots
	\end{bmatrix}=[S]\begin{bmatrix}
		C_{1} \\ \vdots \\ C_{n} \\ \vdots
	\end{bmatrix}=\begin{bmatrix}
		S_{11} & \cdots & S_{1n} & \cdots	\\
		\vdots &  & \vdots & 	\\
		S_{\nu 1} & \cdots & S_{\nu n} & \cdots	\\
		\vdots &  & \vdots & 	\\
	\end{bmatrix}\begin{bmatrix}
		C_{1} \\ \vdots \\ C_{n} \\ \vdots
	\end{bmatrix}	\tag{$4.2.30^{\prime}$}
\end{empheq}\eqnormal
则
\begin{empheq}{equation*}
	a_{\nu}=\sum_{n}S_{\nu n}C_{n},\quad \text{即}\langle \phi_{\nu}|\varPsi \rangle =
	\sum_{n}S_{\nu n}\langle \varPsi_{n}|\varPsi \rangle 
\end{empheq}
利用恒等变换\eqref{eq42.3}式,有
\begin{empheq}{equation*}
	\langle \phi_{\nu}|=\sum_{n}\langle \phi_{\nu}|\varPsi_{n} \rangle \langle \varPsi_{n}|
\end{empheq}
代入上式, 即得
\begin{empheq}{equation}\label{eq42.31}
	S_{\nu n}=\langle \phi_{\nu}|\varPsi_{n} \rangle 
\end{empheq}
注意变换矩阵$S$完全由基矢组$\{|\varPsi_{n} \rangle\}$,$\{|\phi_{\nu} \rangle \}$的相互关系(具体说,就是内积)决定,而与$|\varPsi \rangle $无关.$S$的共轭矩阵记为$S^{+}$,其矩阵元为
\begin{empheq}{equation*}\label{eq42.31'}
	S_{n\nu}^{+}=(S_{\nu n})^{*}=\langle \varPsi_{n}|\phi_{\nu} \rangle 
	\tag{$4.2.31^{\prime}$}
\end{empheq}
$S^{+}S$和$SS^{+}$仍为矩阵,其矩阵元为
\eqindent{6}
\begin{empheq}{align}\label{eq42.32}
	\begin{aligned}
		[S^{+}S]_{nk} &=\sum_{\nu}S_{n\nu}^{+}S_{\nu k}=\sum_{\nu}\langle \varPsi_{n}|\phi_{\nu} \rangle\langle \phi_{\nu}|\varPsi_{k} \rangle  \\
		&=\langle \varPsi_{n}|\varPsi_{k} \rangle =\delta_{nk}
	\end{aligned}
	\\
	\begin{aligned} \notag
		[SS^{+}]_{\nu\mu} &=\sum_{n}S_{\nu n}S_{n\mu}^{+}=\sum_{n}\langle \phi_{\nu}|\varPsi_{n} \rangle\langle \varPsi_{n}|\phi_{\mu} \rangle  \\
		&=\langle \phi_{\nu}|\phi_{\mu} \rangle =\delta_{\nu\mu}
	\end{aligned}
\end{empheq}
可见$S^{+}S$和$SS^{+}$都是单位矩阵:
\begin{empheq}{equation}\label{eq42.33}
	S^{+}S=SS^{+}=\begin{bmatrix}
		1 & 0 & \cdots & 0	\\
		0 & 1 & \cdots & 0	\\
		\vdots & \vdots &  & \vdots	\\
		0 & 0 & \cdots & 1	\\
	\end{bmatrix}\Rightarrow 1
\end{empheq}\eqnormal
$S$与$S^{+}$因此称为么正矩阵.利用\eqref{eq42.33}式,容易得到\eqref{eq42.30}式的逆变换为
\begin{empheq}{equation}\label{eq42.34}
	\varPsi(Q)=S^{+}\varPsi(R)
\end{empheq}
$\varPsi^{*}(Q)$、$\varPsi^{*}(R)$的变换关系则是
\begin{empheq}{equation}\label{eq42.35}
	\varPsi^{+}(R)=\varPsi^{+}(Q)S^{+},\quad \varPsi^{+}(Q)=\varPsi^{+}(R)S
\end{empheq}
再看$A(Q)$、$A(R)$之间的关系.利用恒等变换\eqref{eq42.3}式,可得
\eqindent{4}
\begin{empheq}{align}\label{eq42.36}
	A_{\mu\nu}&=\langle \phi_{\mu}|\hat{A}|\phi_{\nu} \rangle=\sum_{n}\sum_{k}\langle \phi_{\mu}|\varPsi_{n} \rangle \langle \varPsi_{n}|\hat{A}|\varPsi_{k} \rangle \langle \varPsi_{k}|\phi_{\nu} \rangle 	\nonumber\\
	&=\sum_{n}\sum_{k}S_{\nu n}A_{nk}S_{k\nu}^{+}
\end{empheq}\eqnormal
按照矩阵乘法,上式相当于
\begin{empheq}{equation}\label{eq42.37}
	A(R)=SA(Q)S^{+}
\end{empheq}
逆变换为
\begin{empheq}{equation}\label{eq42.38}
	A(Q)=S^{+}A(R)S
\end{empheq}
表象变换时,态矢和算符的具体矩阵表示与所选表象有关,但它们所描述的物理内容则不受表象选择的影响.例举如下.

(1) 态矢的模方不变.态矢间内积不变.以模方为例,\eqref{eq42.11}式已证
\begin{empheq}{equation*}
	\langle \varPsi|\varPsi \rangle =\varPsi^{+}(Q)\varPsi(Q)
\end{empheq}
利用\eqref{eq42.33}、\eqref{eq42.34}、\eqref{eq42.35}式,即得
\begin{empheq}{align}\label{eq42.39}
	\langle \varPsi|\varPsi \rangle &=\varPsi^{+}(Q)\varPsi(Q)=\varPsi^{+}(R)SS^{+}\varPsi(R)	\nonumber\\
	&=\varPsi^{+}(R)\varPsi(R)
\end{empheq}
同样可证
\begin{empheq}{equation}\label{eq42.40}
	\langle \varPsi|\phi \rangle =\varPsi^{+}(Q)\phi(Q)=\varPsi^{+}(R)\phi(R)
\end{empheq}

(2) 力学量算符的本征值与表象选择无关.在$Q$表象中,$\hat{F}$的本征方程由\eqref{eq42.23}式表示,即
\begin{empheq}{equation*}
	F(Q)\varPsi_{\lambda}(Q)=\lambda\varPsi_{\lambda}(Q)
	\tag{$4.2.23$}
\end{empheq}
利用\eqref{eq42.33}、\eqref{eq42.34}、\eqref{eq42.38}式,上式成为
\begin{empheq}{equation*}
	S^{+}F(R)SS^{+}\varPsi_{\lambda}(R)=\lambda S^{+}\varPsi_{\lambda}(R)
\end{empheq}
用$S$乘上式,并利用\eqref{eq42.33}式,就有
\begin{empheq}{equation}\label{eq42.41}
	F(R)\varPsi_{\lambda}(R)=\lambda\varPsi_{\lambda}(R)
\end{empheq}
这正是$R$表象中$\hat{F}$的本征方程.注意由\eqref{eq42.23}式导出\eqref{eq42.41}式的过程中,本征值$\lambda$保持不变.

(3) 平均值公式不变前面\eqref{eq42.21}式已经证明
\begin{empheq}{equation}\label{eq42.42}
	\bar{F}=\langle \varPsi|\hat{F}|\varPsi \rangle =\varPsi^{+}(Q)F(Q)\varPsi(Q)
\end{empheq}
再用\eqref{eq42.33}、\eqref{eq42.34}、\eqref{eq42.35}、\eqref{eq42.38}式,就有
\eqindent{5}
\begin{empheq}{equation}\label{eq42.43}
	\bar{F}=\varPsi^{+}(R)SS^{+}F(R)SS^{+}\varPsi(R)=\varPsi^{+}(R)F(R)\varPsi(R)
\end{empheq}\eqnormal

(4) 算符间相互关系不变.举例说明如下.
设有算符关系$\hat{F}=\hat{A}\hat{B}$,在$Q$表象中显然有
\begin{empheq}{align*}
	F_{kl}&=\langle k|\hat{F}|l \rangle =\langle k|\hat{A}\hat{B}|l \rangle \\
	&=\sum_{n}\langle k|\hat{A}|n \rangle \langle n|\hat{B}|l \rangle =\sum_{n}A_{kn}B_{nl}
\end{empheq}
亦即
\begin{empheq}{equation*}
	F(Q)=A(Q)B(Q)
\end{empheq}
再利用\eqref{eq42.33}、\eqref{eq42.37}、\eqref{eq42.38}式,就有
\eqindent{6}
\begin{empheq}{equation*}
	F(R)=SF(Q)S^{+}=SA(Q)S^{+}SB(Q)S^{+}=A(R)B(R)
\end{empheq}\eqnormal


