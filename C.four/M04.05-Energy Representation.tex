\section[能量表象]{能量表象} \label{sec:04.05} % 
% \makebox[5em][s]{} % 短题目拉间距

选取一个包含$\hat{H}$在内的守恒量完全集,其共同本征态矢盘记为$|\varPsi_{n}\rangle$,简写成$|n \rangle $.以$\{|n \rangle \}$作为态矢量空间的基矢组,就得到能量表象.为明确起见,设能级是分立的,以$n=1,2,3,\cdots$作为基矢的编号,并规定$E_{1}\leqslant E_{2}\leqslant E_{3}\leqslant\cdots$.基矢$|n \rangle $满足本征方程
\begin{empheq}{equation}\label{eq45.1}
	\hat{H}|n \rangle =E_{n}|n \rangle ,\quad \langle n|\hat{H}=E_{n}\langle n|
\end{empheq}
基矢的正交归一化条件为
\begin{empheq}{equation}\label{eq45.2}
	\langle m|n \rangle \equiv\langle \varPsi_{m}|\varPsi_{n} \rangle =\delta_{mn}
\end{empheq}
完备性条件为
\begin{empheq}{equation}\label{eq45.3}
	\sum_{n}|n \rangle\langle n|=1
\end{empheq}
在能量表象中,$\hat{H}$的矩阵元为
\begin{empheq}{equation}\label{eq45.4}
	\hat{H}\equiv\langle m|\hat{H}|n \rangle =E_{n}\delta_{mn}
\end{empheq}
亦即$\hat{H}$表示成对角方阵:
\eqindent{6}
\begin{empheq}{equation}\label{eq45.5}
	H=\begin{bmatrix}
		E_{1} & 0 & 0 & \cdots & 0 & \cdots	\\
		0 & E_{2} & 0 & \cdots & 0 & \cdots	\\
		0 & 0 & E_{3} & \cdots & 0 & \cdots	\\
		\vdots & \vdots & \vdots &  & \vdots & 	\\
		0 & 0 & 0 & \cdots & E_{n} & \cdots	\\
		\vdots & \vdots & \vdots &  & \vdots & 	\\
	\end{bmatrix}
\end{empheq}\eqnormal
完全集中其他守恒量的矩阵表示也是对角方阵,对角矩阵元等于其本征值.

考虑任意力学量$A(\boldsymbol{r,p})$,它的能量表象矩阵元定义为
\begin{empheq}{equation}\label{eq45.6}
	A_{mn}\equiv \langle m|\hat{A}|n \rangle 
\end{empheq}
按照\eqref{eq39.15}式$\hat{A}$的变化率算符是
\begin{empheq}{equation}\label{eq45.7}
	\frac{d\hat{A}}{dt}=\frac{1}{i\hbar}[\hat{A},\hat{H}]=\frac{1}{i\hbar}(\hat{A}\hat{H}-\hat{H}\hat{A})
\end{empheq}
其矩阵元为
\eqindent{6}
\begin{empheq}{align}\label{eq45.8}
	\bigg(\frac{d\hat{A}}{dt}\bigg)_{mn}&=\frac{1}{i\hbar}\langle m|(\hat{A}\hat{H}-\hat{H}\hat{A})|n \rangle \nonumber\\
	&=\frac{1}{i\hbar}(E_{n}-E_{m})\langle m|\hat{A}|n \rangle =i\omega_{mn}A_{mn}
\end{empheq}\eqnormal
其中
\begin{empheq}{equation}\label{eq45.9}
	\omega_{mn}=\frac{(E_{m}-E_{n})}{\hbar}
\end{empheq}
如果$A$是守恒量(包括不属于完全集的其他守恒量),$\hat{A}\hat{H}=\hat{H}\hat{A}$,则$\frac{d\hat{A}}{dt}=0$,由\eqref{eq45.8}式可知
\begin{empheq}{equation}\label{eq45.10}
	A_{mn}=0,\quad E_{m}\neq E_{n} 
\end{empheq}
(但在能量相同的状态之间,$\hat{A}$的非对角矩阵元仍有可能不等于0.)

{\heiti 海森堡图像}

如果采取力学量算符的海森堡图像[$\S$\ref{sec:03.09}(3.)]
\begin{empheq}{equation}\label{eq45.11}
	\hat{A}(t)\equiv e^{i\hat{H}t/i\hbar}\hat{A}e^{\hat{H}t/i\hbar}
\end{empheq}
由于$A(t)$显含$t$,则其能量表象矩阵元将是$t$的函数.由\eqref{eq45.1}式,
\begin{empheq}{equation}\label{eq45.12}
	\begin{aligned}
		e^{\hat{H}t/i\hbar}|n \rangle =e^{E_{n}t/i\hbar}|n \rangle \\
		\langle m|e^{-\hat{H}t/i\hbar}=e^{E_{m}t/i\hbar}\langle m|
	\end{aligned}
\end{empheq}
因此
\begin{empheq}{align}\label{eq45.13}
	A(t)_{mn}&=\langle m|\hat{A}(t)|n \rangle 	\nonumber\\
	&=e^{-E_{m}t/i\hbar}\langle m|\hat{A}|n \rangle ^{E_{n}t/i\hbar}	\nonumber\\
	&=A_{mn}e^{i\omega_{mn}t}
\end{empheq}
显然
\begin{empheq}{equation}\label{eq45.14}
	\frac{d}{dt}A(t)_{mn}=i\omega_{mn}A(t)_{mn}
\end{empheq}
另一方面,$A(t)$的变化率是[\eqref{eq39.22}]
\begin{empheq}{equation}\label{eq45.15}
	\frac{d\hat{A}(t)}{dt}=\frac{1}{i\hbar}\{\hat{A}(t)\hat{H}-\hat{H}\hat{A}(t)\}
\end{empheq}
则
\eqindent{6}
\begin{empheq}{align}\label{eq45.16}
	\bigg(\frac{d\hat{A}(t)}{dt}\bigg)_{mn}&=\frac{1}{i\hbar}\langle m|\{\hat{A}(t)\hat{H}-\hat{H}\hat{A}(t)\}|n \rangle \nonumber\\
	&=\frac{1}{i\hbar}(E_{n}-E_{m})\langle m|\hat{A}(t)|n \rangle =i\omega_{mn}A(t)_{mn}
\end{empheq}
综合\eqref{eq45.14}、\eqref{eq45.16}式,有如下结果:
\begin{empheq}{equation}\label{eq45.17}
	\boxed{\bigg(\frac{d\hat{A}(t)}{dt}\bigg)_{mn}=\frac{d}{dt}A(t)_{mn}=i\omega_{mn}A(t)_{mn}}
\end{empheq}
这公式相当于薛定谔图像的\eqref{eq45.8}式.实际上,两种图像是等价的.

\eqref{eq45.17}式可以继续对$t$微分,从而得到
\begin{empheq}{equation}\label{eq45.18}
	\bigg(\frac{d^{2}\hat{A}(t)}{dt^{2}}\bigg)_{mn}=\frac{d^{2}}{dt^{2}}A(t)_{mn}=-(\omega_{mn})^{2}A(t)_{mn}
\end{empheq}\eqnormal
等等.$A(t)_{mn}$及其各阶微商之间的关系,很像经典力学中的谐振动.

以下为叙述方便起见,公式均按薛定谔图像给出.

如果粒子(质量$\mu$)在势场$V(r)$中运动,则总能量算符为
\begin{empheq}{equation}\label{eq45.19}
	\hat{H}=\frac{1}{2\mu}\hat{\boldsymbol{p}}^{2}+V(\boldsymbol{r})
\end{empheq}
这种情形下,
\begin{empheq}{equation}\label{eq45.20}
	\frac{d\hat{x}}{dt}=\frac{1}{i\hbar}[\hat{x},\hat{H}]=\frac{\hat{p}_{x}}{\mu}
\end{empheq}
按照\eqref{eq45.8}式,有关系
\begin{empheq}{equation}\label{eq45.21}
	\boxed{(p_{x})_{mn}=i\mu\omega_{mn}x_{mn}}
\end{empheq}
$y$分量和$z$分量也有类似的关系.另外,由基本对易式
\begin{empheq}{equation*}
	\hat{x}\hat{p}_{x}-\hat{p}_{x}\hat{x}=i\hbar
\end{empheq}
两端对$|m \rangle $态求平均值,得到
\begin{empheq}{align*}
	i\hbar&=(xp_{x})_{mm}-(p_{x}x)_{mm}	\\
	&=\sum_{n}\{x_{mn}(p_{x})_{nm}-(p_{x})_{mn}x_{nm}\}	\\
	&=i\mu\sum_{n}x_{nm}x_{mn}(\omega_{nm}-\omega_{mn})	\\
	&=2i\mu\sum_{n}\omega_{nm}|x_{nm}|^{2}
\end{empheq}
亦即
\begin{empheq}{equation}\label{eq45.22}
	\boxed{\sum_{n}(E_{n}-E_{m})|x_{nm}|^{2}=\frac{\hbar^{2}}{2\mu}}
\end{empheq}
这就是著名的Thomas-Reich-Kuhn求和规则.

能量表象中的求和规则(Sum rule)很多.下面介绍几个常用的公式.给定任意算符$\hat{F}(\boldsymbol{r,p})$及具共轭$\hat{F}^{+}$,有矩阵元关系
\eqindent{6}
\begin{empheq}{equation}\label{eq45.23}
	(F_{nm})^{*}=F_{mn}^{+}=\langle m|\hat{F}^{+}|n \rangle =(\langle n|\hat{F}|m \rangle )^{*}
\end{empheq}\eqnormal
以及对易式
\eqindent{4}
\begin{empheq}{equation}\label{eq45.24}
	[\hat{F}^{+},[\hat{H},\hat{F}]]=\hat{F}^{+}\hat{H}\hat{F}+\hat{F}\hat{H}\hat{F}^{+}-\hat{H}\hat{F}\hat{F}^{+}-\hat{F}^{+}\hat{F}\hat{H}
\end{empheq}\eqnormal
将\eqref{eq45.24}式中各项在$|m \rangle $态下求平均值,并利用恒等变换\eqref{eq45.3}式,得到
\begin{subequations}\label{eq45.25}
	\eqindent{6}
	\begin{align}\label{eq45.25a}
		\langle m|\hat{F}^{+}H\hat{F}|m \rangle &=\sum_{n}\langle m|\hat{F}^{+}H|n \rangle\langle n|\hat{F}|m \rangle 	\nonumber\\
		&=\sum_{n}E_{n}\langle m|\hat{F}^{+}|n \rangle\langle n|\hat{F}|m \rangle 	\nonumber\\
		&=\sum_{n}E_{n}F_{mn}^{+}F_{nm}=\sum_{n}E_{n}|F_{nm}|^{2}
	\end{align}\eqnormal
类似地,
	\begin{equation}\label{eq45.25b}
		\langle m|\hat{F}\hat{H}\hat{F}^{+}|m \rangle =E_{n}\sum_{n}|F_{mn}|^{2}
	\end{equation}
以及
	\begin{align}
		\langle m|\hat{H}\hat{F}\hat{F}^{+}|m \rangle =E_{m}\sum_{n}|F_{mn}|^{2}	\label{eq45.25c}\\
		\langle m|\hat{F}^{+}\hat{F}\hat{H}|m \rangle =E_{m}\sum_{n}|F_{nm}|^{2}	\label{eq45.25d}
	\end{align}

\end{subequations}
\noindent 合起来,即得
\eqindent{4}
\begin{empheq}{equation}\label{eq45.26}
	\sum_{n}(E_{n}-E_{m})(|F_{nm}|^{2}+|F_{mn}|^{2})=\langle m|[\hat{F}^{+},[\hat{H},\hat{F}]]|m \rangle 
\end{empheq}\eqnormal
如$\hat{F}$为厄密算符,$\hat{F}^{+}=\hat{F}$,则
\begin{empheq}{equation*}
	F_{nm}=F_{mn}^{+}\quad |F_{nm}|^{2}=|F_{mn}|^{2}
\end{empheq}
这时\eqref{eq45.26}式简化成
\begin{empheq}{equation}\label{eq45.27}
	\sum_{n}(E_{n}-E_{m})|F_{nm}|^{2}=\frac{1}{2}\langle m|[\hat{F},[\hat{H},\hat{F}]]|m \rangle 
\end{empheq}
当$\hat{H}$取\eqref{eq45.19}式时,如取$\hat{F}=\hat{x}$,则
\begin{empheq}{equation*}
	[\hat{x},[\hat{H},\hat{x}]]=\bigg[\hat{x},-\frac{i\hbar}{\mu}p_{x}\bigg]=\frac{\hbar^{2}}{\mu}
\end{empheq}
\eqref{eq45.27}式就给出\eqref{eq45.22}式.

\example 一维谐振子的能量算符和能级为
\begin{empheq}{equation}\label{eq45.28}
	\hat{H}=\frac{\hat{p}^{2}}{2\mu}+\frac{1}{2}\mu\omega^{2}\hat{x}^{2}
\end{empheq}
\begin{empheq}{equation}\label{eq45.29}
	E_{n}=\bigg(n+\frac{1}{2}\bigg)\hbar\omega,\quad n=0,1,2,\cdots
\end{empheq}
求$x$和$p$的能量表象矩阵元.

\solution 首先,\eqref{eq45.21}式成立,
\begin{empheq}{equation}\label{eq45.30}
	p_{mn}=i\mu\omega_{mn}x_{mn}=i(m-n)\mu\omega x_{mn}
\end{empheq}
其次,
\begin{empheq}{equation*}
	\frac{d\hat{p}}{dt}=\frac{1}{i\hbar}[\hat{p},\hat{H}]=-\frac{\partial V}{\partial x}=-\mu\omega^{2}\hat{x}
\end{empheq}
因此
\begin{empheq}{equation}\label{eq45.31}
	-\mu\omega^{2}x_{mn}=\bigg(\frac{d\hat{p}}{dt}\bigg)_{mn}=i\omega_{mn}p_{mn}
\end{empheq}
与\eqref{eq45.30}式合并,消去$p_{mn}$,得到
\begin{empheq}{equation}\label{eq45.32}
	(\omega_{mn}^{2}-\omega^{2})x_{mn}=0
\end{empheq}
易见仅当$(m-n)=\pm 1$,$x_{mn}$才不为0.亦即$x$的矩阵元中,
\begin{empheq}{equation}\label{eq45.33}
	x_{n+1,n}\neq 0,\quad \text{其余为0}
\end{empheq}
利用\eqref{eq45.22}式,即得
\begin{empheq}{equation}\label{eq45.34}
	|x_{n+1,n}|^{2}-|x_{n-1,n}|^{2}=\frac{\hbar}{2\mu\omega}
\end{empheq}

另一方面,根据海尔曼定理或位力定理($\S$\ref{sec:03.11}),在$|n \rangle $态下势能平均值为
\begin{empheq}{equation}\label{eq45.35}
	V_{nn}=\frac{1}{2}\mu\omega^{2}(x^{2})_{nn}=\frac{1}{2}E_{n}
\end{empheq}
由于
\eqindent{6}
\begin{empheq}{equation*}
	(x^{2})_{nn}=\sum_{nn^{\prime}}x_{n^{\prime}n}=|x_{n+1,n}|^{2}+|x_{n-1,n}|^{2}
\end{empheq}
代入\eqref{eq45.35}式,即得
\begin{empheq}{equation}\label{eq45.36}
	|x_{n+1,n}|^{2}+|x_{n-1,n}|^{2}=\frac{E_{n}}{\mu\omega^{2}}=\bigg(n+\frac{1}{2}\bigg)\frac{\hbar}{\mu\omega}
\end{empheq}
由\eqref{eq45.34}、\eqref{eq45.36}式即可解出
\begin{empheq}{equation}\label{eq45.37}
	|x_{n+1,n}|^{2}=\frac{n+1}{2}\frac{\hbar}{\mu\omega},\quad n=0,1,2,\cdots
\end{empheq}
按惯例,$|n \rangle $的相因子选取原则是保证$x_{n+1,n}$为正实数.这样
\begin{empheq}{equation}\label{eq45.38}
	x_{n+1,n}=x_{n,n+1}=\bigg(\frac{n+1}{2}\frac{\hbar}{\mu\omega}\bigg)^{\frac{1}{2}},\quad n=0,1,2,\cdots
\end{empheq}
再利用\eqref{eq45.30}式,得到
\begin{empheq}{equation}\label{eq45.39}
	p_{n+1,n}=-p_{n,n+1}=i\mu\omega x_{n+1,n}=i\sqrt{\frac{n+1}{2}\mu\omega\hbar}
\end{empheq}\eqnormal
注意,矩阵$x_{mn},p_{mn}$均为厄密矩阵,但$x_{mn}$为实矩阵$x_{n+1,n}=x_{n,n+1}$,所以,$p_{mn}$为纯虚矩阵,所以
\begin{empheq}{equation*}
	p_{n,n+1}=(p_{n+1,n})^{*}=-p_{n+1,n}
\end{empheq}















