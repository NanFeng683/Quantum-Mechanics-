\section[狄拉克符号]{狄拉克符号} \label{sec:04.01} % 
% \makebox[5em][s]{} % 短题目拉间距

量子力学创始人之一狄拉克(P.A.M.Dirac)全面思考了量子力学的理论框架,综合考虑了复数、矢量代数、矩阵等数学领域的运算规则,创造性地发展了关于量子力学基本理论的整套概念与逻辑推理体系,制定了相应的数学符号与运算规则.狄拉克符号概念深刻,简洁方便,被国际物理学界广泛使用.

{\heiti 1.态矢空间}

按照狄拉克的观点,物理体系的每一个微观状态可以用一种抽象的多维矢量空间(Hilbert space)中的矢量来描述,这种矢量称为态矢,全体态矢构成态矢空间.提出这种观点的主要依据之一就是态叠加原理,其表述方式
\begin{empheq}{equation}\label{eq41.1}
	\varPsi=C_{1}\varPsi_{1}+C_{2}\varPsi_{2}+\cdots
\end{empheq}
与矢量间的线性叠加关系极其相似.

态矢空间包括右矢空间与共轭的左矢空间.右矢(ket)符号为$|\rangle$,左矢(bra)符号为$\langle|$.与$\varPsi$态对应的右矢记成$|\varPsi \rangle$,左矢记成$\langle \varPsi|$.共轭运算用符号$^{*}$表示.规定
\begin{empheq}{equation}\label{eq41.2}
	|\varPsi \rangle^{+}=\langle \varPsi|,\quad \varPsi\langle^{+} |=|\varPsi \rangle
\end{empheq}
数(实数或复数)$C$可以与右矢或左矢相乘,仍为右矢或左矢.规定这种乘法与次序无关,即规定
\begin{empheq}{equation}\label{eq41.3}
	C|\varPsi \rangle=\langle \varPsi|C,\quad C\langle \varPsi|=\langle \varPsi|C
\end{empheq}
数$C$的共轭规定成就是复共轭(符号$^{*}$),即
\begin{empheq}{equation}\label{eq41.4}
	C=\alpha+i\beta,\quad C^{+}=C^{*}=\alpha-i\beta
\end{empheq}
\eqindent{4}
\begin{empheq}{equation}\label{eq41.5}
	(C| \varPsi\rangle)^{*}=C^{*}\langle \varPsi|=\langle \varPsi|C^{*},\quad (C\langle \varPsi|)^{*}=C^{*}|\varPsi \rangle=|\varPsi \rangle C^{*}
\end{empheq}\eqnormal
这样,与态叠加原理\eqref{eq41.1}式对应的态右矢关系规定成
\begin{empheq}{equation}\label{eq41.6}
	|\varPsi \rangle =C_{1}|\varPsi_{1} \rangle +C_{2}|\varPsi_{2} \rangle +\cdots=\sum_{n}C_{n}|\varPsi_{n} \rangle 
\end{empheq}
对应的左矢关系则是
\begin{empheq}{equation*}\label{eq41.6'}
	\langle \varPsi|=C_{1}^{*}\langle \varPsi_{1}|+C_{2}^{*}\langle \varPsi_{2}|+\cdots=\sum_{n}\langle \varPsi_{n}|C_{n}^{*}	\tag{$4.1.6^{\prime}$}
\end{empheq}
左矢与右矢可以按“左矢在左、右矢在右”:即$\langle | | \rangle $的方式相乘,规定其性质是数,称为二态矢的“内积”.共轭运算规则为
\begin{empheq}{equation}\label{eq41.7}
	(\langle \varPsi|\phi \rangle )^{+}=\langle \phi|\varPsi \rangle =(\langle \varPsi|\phi \rangle )^{*}
\end{empheq}
其中$\langle \varPsi|\phi \rangle $是$\langle \varPsi||\phi \rangle $的简化写法.

如内积$\langle \varPsi|\phi \rangle=0$,称态矢$|\varPsi \rangle $,$|\phi \rangle $相互“正交”.任意态矢$|\varPsi \rangle $与自身的内积$\langle \varPsi|\varPsi \rangle $必为实数,规定其为正数,称为$|\varPsi \rangle $的“模方”,有时也记为$||\varPsi \rangle |^{2}$.注意
\begin{empheq}{equation}\label{eq41.8}
	|C|\varPsi \rangle |^{2}=\langle \varPsi|C^{*}C|\varPsi \rangle =C^{*}C\langle \varPsi|\varPsi \rangle 
\end{empheq}
如模方$\langle \varPsi|\varPsi \rangle$,称$|\varPsi \rangle $是归一化的.如$|\varPsi \rangle $并未归一化,只要$\langle \varPsi|\varPsi \rangle $有限,容易找到适当系数$C$,使$C|\varPsi \rangle $是归一化的.注意$C|\varPsi \rangle $与$|\varPsi \rangle $描述的是同一种状态.

态矢空间是希尔伯特(Hilbert)空间,原则上存在满足正交归一条件
\begin{empheq}{equation}\label{eq41.9}
	\langle \varPsi_{n}|\varPsi_{k} \rangle =\delta_{nk}
\end{empheq}
的基矢组$\{|\varPsi_{n} \rangle,n=1,2,\cdots \}$,基矢组具有完备性,任意态矢$|\varPsi \rangle $均可表示成基矢的线性叠加:
\begin{empheq}{equation*}
	|\varPsi \rangle =\sum_{n}C_{n}|\varPsi_{n} \rangle 
\end{empheq}
为了简化符号,基矢$|\varPsi_{n} \rangle $记成$|n \rangle $,上式写成
\begin{empheq}{equation}\label{eq41.10}
	|\varPsi \rangle =\sum_{n}C_{n}|n \rangle
\end{empheq}
基矢的正交归一条件\eqref{eq41.9}式即
\eqindent{12}
\begin{empheq}{equation*}\label{eq41.9'}
	\langle n|k \rangle =\delta_{nk}	\tag{$4.1.9^{\prime}$}
\end{empheq}\eqnormal
利用\eqref{eq41.9'}式,容易求出\eqref{eq41.10}式中各$C_{n}$为
\begin{empheq}{equation}\label{eq41.11}
	C_{n}=\langle n|\varPsi \rangle,\quad C_{n}^{+}=\langle \varPsi|n \rangle 
\end{empheq}
$|\varPsi \rangle $的模方为
\begin{empheq}{equation}\label{eq41.12}
	\langle \varPsi|\varPsi \rangle =\sum_{n}C_{n}^{*}C_{n}>0
\end{empheq}
考虑另一个态矢
\begin{empheq}{equation}\label{eq41.13}
	|\phi \rangle =\sum_{n}b_{n}|n \rangle ,\quad b_{n}=\langle n|\phi \rangle 
\end{empheq}
则$|\varPsi \rangle $,$|\phi \rangle $的内积为
\eqindent{6}
\begin{empheq}{equation*}
	\langle \varPsi|\phi \rangle =\sum_{n}C_{n}^{*}b_{n},\quad \langle \phi|\varPsi \rangle =\sum_{n}b_{n}^{*}C_{n}=(\langle \varPsi|\phi \rangle )^{*}
\end{empheq}\eqnormal

{\heiti 2.线性算符}

以$\hat{A},\hat{B},\hat{F},\hat{G}$等表示线性算符,规定它们可以作用于右矢及左矢,结果仍为右矢及左矢.允许的作用方式为
\eqindent{12}
\begin{empheq}{equation*}
	\hat{A}| \rangle \rightarrow | \rangle ,\quad \langle |\hat{B}\rightarrow \langle |
\end{empheq}\eqnormal
诸如$\hat{A}\langle |$,$| \rangle \hat{A}$等方式则是不允许的,没有意义.当态矢由\eqref{eq41.6}、\eqref{eq41.6'}表示时,线性算符$\hat{A}$的作用规则是
\begin{empheq}{equation}\label{eq41.14}
	\hat{A}\bigg(\sum_{n}C_{n}|\varPsi_{n} \rangle \bigg)=\sum_{n}C_{n}\hat{A}|\varPsi_{n} \rangle 
\end{empheq}
\begin{empheq}{equation*}\label{eq41.14'}
	\bigg(\sum_{n}\langle \varPsi_{n}|C_{n}^{*} \bigg)\hat{A}=\sum_{n}C_{n}\langle \varPsi_{n}|\hat{A}	\tag{$4.1.14^{\prime}$}
\end{empheq}
原则上,定义一个线性算符$\hat{A}$,需要规定$\hat{A}$对全部基矢的作用结果,即规定$\hat{A}|n \rangle $,$\langle n|\hat{A}$,$n=1,2,\cdots$.等价地,相当于规定全部$\langle k|\hat{A}|n \rangle $.习惯上常用符号
\eqindent{12}
\begin{empheq}{equation}\label{eq41.15}
	\langle k|\hat{A}|n \rangle\equiv A_{kn}
\end{empheq}\eqnormal
注意其性质是“数”.上式常称为算符$\hat{A}$的“矩阵元”,这名称的来由参看\eqref{eq41.17}式以下的说明以及$\S$\ref{sec:04.02}.

算符$\hat{A}$的共轭算符记为$\hat{A}^{*}$,按下式定义:
\eqindent{7}
\begin{empheq}{equation}\label{eq41.16}
	(\hat{A}|\varPsi \rangle )^{*}=\langle \varPsi|\hat{A}^{+},\quad (\langle \varPsi|\hat{A})^{+}=\hat{A}^{+}|\varPsi \rangle 
\end{empheq}
$|\varPsi \rangle $为任意态矢.将上式用于基矢,就有
\begin{empheq}{equation}\label{eq41.17}
	(A_{kn})^{*}=(\langle k|\hat{A}|n \rangle )^{+}=\langle n|\hat{A}^{+}|k \rangle =A_{nk}^{+}
\end{empheq}\eqnormal
也可以将\eqref{eq41.17}式作为$\hat{A}^{+}$的定义.如将各$\hat{A}_{kn}$按行列次序排成矩阵($k$代表行,$n$代表列),各$\hat{A}_{nk}^{+}$也排成矩阵,二者刚好互为共轭矩阵.

如果对于各基矢$|n \rangle $,$|k \rangle $,均有$A_{kn}^{*}$,则必有$\hat{A}=\hat{A}^{+}$.反之亦然.具有这种性质的算符,称为厄密(Hermite) 算符.

如态矢$|\varPsi \rangle $与线性算符$\hat{F}$满足关系(本征方程)
\begin{empheq}{equation}\label{eq41.18}
	\hat{F}|\varPsi \rangle =\lambda|\varPsi \rangle ,\quad \lambda\text{为数}
\end{empheq}
则称$|\varPsi \rangle $为$\hat{F}$的本征矢,$\lambda$为$\hat{F}$的本征值.上式的共轭方程是
\begin{empheq}{equation*}\label{eq41.18'}
	\langle \varPsi|\hat{F}^{+}=\lambda^{*}\langle \varPsi|
\end{empheq}
如果$\hat{F}\neq\hat{F}^{+}$,本征值一般为复数.可以证明,如$\hat{F}$是厄密算符$(\hat{F}=\hat{F}^{+})$,则本征值必为实数.以定理的形式证明如下.

\theo 厄密算符的本征值必为实数.

\prove 设$\hat{F}\neq\hat{F}^{+}$,以$\langle \varPsi|$乘本征方程\eqref{eq41.18},就有
\begin{empheq}{equation*}
	\langle \varPsi|\hat{F}|\varPsi \rangle =\lambda\langle \varPsi|\varPsi \rangle 
\end{empheq}
上式两端均取共轭,由于$\hat{F}\neq\hat{F}^{+}$,就有
\begin{empheq}{equation*}
	(\langle \varPsi|\hat{F}|\varPsi \rangle )^{+}=\langle \varPsi|\hat{F}|\varPsi \rangle =\lambda^{*}\langle \varPsi|\varPsi \rangle 
\end{empheq}
二式左端相同,右端由于$\langle \varPsi|\varPsi \rangle\neq 0$,故有$\lambda=\lambda^{*}$,$\lambda$为实数.

关于厄密算符,还有本征矢的正交定理,如下.

\theo 如$|\varPsi_{1} \rangle$,$|\varPsi_{2} \rangle$是厄密算符$\hat{F}$的两个本征矢,相应于不同的本征值,则$|\varPsi_{1} \rangle$,$|\varPsi_{2} \rangle$正交.

\prove $|\varPsi_{1} \rangle$,$|\varPsi_{2} \rangle$分别满足本征方程
\begin{empheq}{equation}\label{eq41.19}
	\hat{F}|\varPsi_{1} \rangle =\lambda_{1}|\varPsi_{1} \rangle ,\quad \hat{F}|\varPsi_{2} \rangle =\lambda_{2}|\varPsi_{2} \rangle 
\end{empheq}
它们的共轭方程是(注意$\hat{F}=\hat{F}^{+}$,$\lambda_{1},\lambda_{2}$是实数)
\begin{empheq}{equation*}\label{eq41.19'}
	\langle \varPsi_{1}|\hat{F}=\lambda_{1}\langle \varPsi_{1}|,\quad \langle \varPsi_{2}|\hat{F}=\lambda_{2}\langle \varPsi_{2}|	\tag{$4.1.19^{\prime}$}
\end{empheq}
以$\langle \varPsi_{2}|$乘\eqref{eq41.19}第一式,得到
\eqindent{12}
\begin{empheq}{equation*}
	\langle \varPsi_{2}|\hat{F}|\varPsi_{1} \rangle =\lambda_{1}\langle \varPsi_{2}|\varPsi_{1} \rangle 
\end{empheq}
\eqref{eq41.19'}式乘$|\varPsi_{1} \rangle $,则得
\begin{empheq}{equation*}
	\langle \varPsi_{2}|\hat{F}|\varPsi_{1} \rangle =\lambda_{2}\langle \varPsi_{2}|\varPsi_{1} \rangle 
\end{empheq}\eqnormal
二式左端相同,右端由于$\lambda_{1}\neq\lambda_{2}$,必然有$\langle \varPsi_{2}|\varPsi_{1} \rangle=0$.

两个线性算符$\hat{A},\hat{B}$之和记为$(\hat{A}+\hat{B})$,对态矢的作用规则为
\begin{empheq}{equation}\label{eq41.20}
	(\hat{A}+\hat{B})|\varPsi \rangle = \hat{A}|\varPsi \rangle +\hat{B}|\varPsi \rangle 
\end{empheq}
显然$(\hat{A}+\hat{B})=(\hat{B}+\hat{A})$.$\hat{A},\hat{B}$之积记为$(\hat{A}\hat{B})$或$\hat{B}\hat{A}$,对态矢的作用规则为
\eqindent{6}
\begin{empheq}{equation}\label{eq41.21}
	(\hat{A}\hat{B})|\varPsi \rangle =\hat{A}(\hat{B}|\varPsi \rangle ),\quad \langle \varPsi|(\hat{A}\hat{B})=(\langle \varPsi|\hat{A}) \hat{B}
\end{empheq}
一般说,$\hat{A}\hat{B}\neq\hat{B}\hat{A}$.如果$\hat{A}\hat{B}=\hat{B}\hat{A}$,称$\hat{A},\hat{B}$对易.由\eqref{eq41.20}、\eqref{eq41.21}式,再利用\eqref{eq41.16}式,容易证明
\begin{empheq}{equation}\label{eq41.22}
	(\hat{A}+\hat{B})^{+}=\hat{A}^{+}+\hat{B}^{+},\quad (\hat{A}\hat{B})^{+}=\hat{B}^{+}\hat{A}^{+}
\end{empheq}\eqnormal
这样,如$\hat{A},\hat{B}$均为厄密算符,则$(\hat{A}+\hat{B})$也是厄密算符.如$\hat{A},\hat{B}$为互相对易的厄密算符,则$(\hat{A}\hat{B})$也是厄密算符.

前面讲过,构造式$\langle | \rangle $即$\langle || \rangle $的性质是“数”.那么构造$|\rangle \langle| $式的性质又是什么?易见任何左矢$\langle \varPsi|$从左与它相乘,结果仍为左矢;它与右矢$|\varPsi \rangle $相乘($|\varPsi \rangle $在右),结果仍为右矢.所以$|\rangle \langle|$的性质是线性算符.例如,将\eqref{eq41.11}式代入\eqref{eq41.10}式,可得
\begin{empheq}{equation*}
	|\varPsi \rangle =\sum_{n}|n \rangle C_{n}=\sum_{n}|n \rangle\langle n|\varPsi \rangle  
\end{empheq}
因为$|\varPsi \rangle $可以是任何态矢,故知
\eqindent{12}
\begin{empheq}{equation}\label{eq41.23}
	\boxed{\sum_{n}|n \rangle\langle n|=1}
\end{empheq}\eqnormal
称为恒等变换算符.在涉及表象变换的计算中,\eqref{eq41.23}式极其有用.读者应该熟记此式,在需要用它的场合能及时想到.

\example 给定一个归一化的态矢$|\phi \rangle$,并定义“投影算符”
\begin{empheq}{equation}\label{eq41.24}
	\hat{P}_{\phi}=|\phi \rangle\langle \phi|
\end{empheq}
求$\hat{P}_{\phi}$的本征值、本征矢.研究$\hat{P}_{\phi}$作用于任意态矢$|\varPsi \rangle $将得到什么结果.

\solution 显然$\hat{P}_{\phi}$是厄密算符,其本征值应该是实数.另外,
\begin{empheq}{equation*}
	\hat{P}_{\phi}^{2}=|\phi \rangle\langle \phi||\phi \rangle\langle \phi|=|\phi \rangle\langle \phi|=\hat{P}_{\phi}
\end{empheq}
即
\begin{empheq}{equation}\label{eq41.25}
	\hat{P}_{\phi}^{2}-\hat{P}_{\phi}=\hat{P}_{\phi}(\hat{P}_{\phi}-1)=0
\end{empheq}
设$\hat{P}_{\phi}$的本征值为$\lambda$,本征矢为$|\varPsi_{\lambda} \rangle $,满足本征方程
\begin{empheq}{equation}\label{eq41.26}
	\hat{P}_{\phi}|\varPsi_{\lambda} \rangle =\lambda|\varPsi_{\lambda} \rangle 
\end{empheq}
将\eqref{eq41.25}式作用于$|\varPsi_{\lambda} \rangle $,得到
\begin{empheq}{equation}\label{eq41.27}
	\lambda(\lambda-1)=0,\quad \lambda=1,0
\end{empheq}
将$\lambda=1$代入\eqref{eq41.26}式,得到
\begin{empheq}{equation*}
	\hat{P}_{\phi}|\varPsi_{1} \rangle=|\phi \rangle\langle \phi|\varPsi_{1} \rangle=|\varPsi \rangle 
\end{empheq}
可见$|\varPsi_{1} \rangle=C|\phi \rangle $,系数$C$并无实质性意义,可取为1,亦即相应于本征值$\lambda=1$的$\hat{P}_{\phi}$本征矢就是$|\phi \rangle $.

将$\lambda=0$代入\eqref{eq41.26}式,得到
\begin{empheq}{equation*}
	\hat{P}_{\phi}|\varPsi_{0} \rangle=|\phi \rangle\langle \phi|\varPsi_{0} \rangle=0,\quad \langle \phi|\varPsi_{0} \rangle=0
\end{empheq}
可知一切与$|\phi \rangle $正交的态矢$|\varPsi_{0} \rangle $都是$\hat{P}_{\phi}$的本征矢,相应本征值$\lambda=0$.

$\hat{P}_{\phi}$作用于任意态矢$|\varPsi \rangle $,结果为
\eqindent{6}
\begin{empheq}{equation}\label{eq41.28}
	\hat{P}_{\phi}|\varPsi \rangle=|\phi \rangle\langle \phi|\varPsi \rangle=C_{\phi}|\phi \rangle,\quad C_{\phi}=\langle \phi|\varPsi \rangle 
\end{empheq}\eqnormal
可见,$\hat{P}_{\phi}$的作用是将$|\varPsi \rangle $转变成$C_{\phi}|\phi \rangle $,换言之,将态矢$|\varPsi \rangle $向$|\phi \rangle $“方向”投影.














