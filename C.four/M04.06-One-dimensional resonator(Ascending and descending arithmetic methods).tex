\section[一维谐振子(升降算符方法)]{一维谐振子(升降算符方法)} \label{sec:04.06} % 
% \makebox[5em][s]{} % 短题目拉间距

质置为$\mu$的粒子在弹性力场中运动,总能量算符可以写成
\begin{empheq}{equation}\label{eq46.1}
	fuction
\end{empheq}
$\hat{x}$和$\hat{p}$是粒子的坐标和动量算符,满足对易式
\begin{empheq}{equation}\label{eq46.2}
	[\hat{x},\hat{p}]=\hat{x}\hat{p}-\hat{p}\hat{x}=i\hbar
\end{empheq}
唯一重要的守恒量就是$H$本身.设$\hat{H}$的本征值为$E_{n}$,归一化本征态为$|n \rangle $,$n=0$表示基态.$|n \rangle $就是能量表象的基矢,满足正交归一条件
\begin{empheq}{equation}\label{eq46.3}
	\langle n|n^{\prime} \rangle =\delta_{nn^{\prime}}
\end{empheq}
及本征方程
\begin{empheq}{equation}\label{eq46.4}
	\hat{H}|n \rangle =E_{n}|n \rangle 
\end{empheq}

引入无量纲算符
\begin{empheq}{equation}\label{eq46.5}
	\hat{Q}=\sqrt{\frac{\mu\omega}{\hbar}}\hat{x},\quad \hat{P}=\frac{1}{\sqrt{\mu\hbar\omega}}\hat{p}
\end{empheq}
$\hat{H}$可以简化成
\begin{empheq}{equation}\label{eq46.6}
	\hat{H}=\frac{\hbar\omega}{2}(\hat{Q}^{2}+\hat{P}^{2})
\end{empheq}
$\hat{Q}$,$\hat{P}$满足对易式
\begin{empheq}{equation}\label{eq46.7}
	[\hat{Q},\hat{P}]=\hat{Q}\hat{P}-\hat{P}\hat{Q}=i
\end{empheq}
容易求出
\begin{empheq}{equation}\label{eq46.8}
	\begin{aligned}
		\frac{d\hat{Q}}{dt}&=\frac{1}{i\hbar}[\hat{Q},\hat{H}]=\frac{\omega}{2i}[\hat{Q},\hat{P}^{2}]=\omega\hat{P}	\\
		\frac{d\hat{P}}{dt}&=\frac{1}{i\hbar}[\hat{P},\hat{H}]=\frac{\omega}{2i}[\hat{P},\hat{Q}^{2}]=-\omega\hat{Q}
	\end{aligned}
\end{empheq}
因此,如果定义算符
\begin{empheq}{equation}\label{eq46.9}
	\hat{a}=\frac{1}{\sqrt{2}}(\hat{Q}+i\hat{p}),\quad \hat{a}^{+}=\frac{1}{\sqrt{2}}(\hat{Q}-i\hat{p})
\end{empheq}
就有
\begin{empheq}{equation}\label{eq46.10}
	\frac{d\hat{a}}{dt}=\frac{1}{i\hbar}[\hat{a},\hat{H}]=-i\omega\hat{a}
\end{empheq}
\begin{empheq}{equation}\label{eq46.11}
	\frac{d\hat{a}^{+}}{dt}=\frac{1}{i\hbar}[\hat{a}^{+},\hat{H}]=i\omega\hat{a}^{+}
\end{empheq}
利用对易式\eqref{eq46.7},容易证明
\begin{empheq}{equation}\label{eq46.12}
	[\hat{a},\hat{a}^{+}]\equiv\hat{a}\hat{a}^{+}-\hat{a}^{+}\hat{a}=1
\end{empheq}
而$\hat{H}$可以表示成
\begin{empheq}{equation}\label{eq46.13}
	\hat{H}=\bigg(\hat{a}^{+}\hat{a}+\frac{1}{2}\bigg)\hbar\omega
\end{empheq}
\eqref{eq46.10}式可以写成
\begin{empheq}{equation*}\label{eq46.10'}
	\hat{H}\hat{a}=\hat{a}(\hat{H}-\hbar\omega)		\tag{$4.6.10^{\prime}$}
\end{empheq}
上式作用于$|n \rangle $态,得到
\begin{empheq}{equation}\label{eq46.14}
	\hat{H}\hat{a}|n \rangle =(E_{n}-\hbar\omega)\hat{a}|n \rangle 
\end{empheq}
这表明,如$\hat{a}|n \rangle\neq 0$,它就是$\hat{H}$的另一个本征态矢量,本征值为$(E_{n}-\hbar\omega)$.重复这个推理过程,易知
\begin{empheq}{equation}\label{eq46.15}
	E_{n},E_{n}-\hbar\omega,E_{n}-2\hbar\omega,\cdots
\end{empheq}
都是能量本征值.但是,由于$\hat{H}$是正定的,$E>0$,上列数列必须中止于基态能级$E_{0}$,即$(E_{0}-\hbar\omega)$不再是能量本征值,条件为
\begin{empheq}{equation}\label{eq46.16}
	\hat{a}|0 \rangle =0
\end{empheq}
因此,利用\eqref{eq46.13}式,即得
\begin{empheq}{equation*}
	\hat{H}|0 \rangle=\hbar\omega\hat{a}^{+}\hat{a}|0 \rangle +\frac{1}{2}\hbar\omega|0 \rangle =\frac{1}{2}\hbar\omega|0 \rangle 
\end{empheq}
亦即$E_{0}=\frac{\hbar\omega}{2}$.

\eqref{eq46.11}式可以写成
\begin{empheq}{equation}\label{eq46.11'}
	\hat{H}\hat{a}^{+}|n \rangle =\hat{a}^{+}(\hat{H}+\hbar\omega)
	\tag{$4.6.11^{\prime}$}
\end{empheq}
作用于$|n \rangle $,得到
\begin{empheq}{equation}\label{eq46.17}
	\hat{H}\hat{a}^{+}|n \rangle =(E_{n}+\hbar\omega)\hat{a}^{+}|n \rangle 
\end{empheq}
这表明,$\hat{a}^{+}|n \rangle $也是$\hat{H}$的本征态,本征值$(E_{n}+\hbar\omega)$.重复这个推理过程,可知
\begin{empheq}{equation}\label{eq46.18}
	E_{n},E_{n}+\hbar\omega,E_{n}+2\hbar\omega,\cdots
\end{empheq}
都是能量本征值.利用\eqref{eq46.12}式,\eqref{eq46.13}式可以改写成
\begin{empheq}{equation}\label{eq46.13'}
	\hat{H}=\bigg(\hat{a}\hat{a}^{+}-\frac{1}{2}\bigg)\hbar\omega
	\tag{$4.6.13^{\prime}$}
\end{empheq}
将此式作用于$|n \rangle$,由于$E_{n}\leqslant E_{0}=\frac{\hbar\omega}{2}$,因此$\hat{a}^{+}|n \rangle$不可能等于0,由此可知本征值序列\eqref{eq46.18}没有上限.

综合以上分析, 可知全部能谱为
\begin{empheq}{equation}\label{eq46.19}
	\boxed{E_{n}=\bigg(n+\frac{1}{2}\bigg)\hbar\omega,\quad n=0,1,2,\cdots}
\end{empheq}
即能级分布是均匀的,相邻能级间距为$\hbar\omega$.

将\eqref{eq46.13}式作用于$|n \rangle $,易见
\begin{empheq}{equation}\label{eq46.20}
	\boxed{\hat{a}^{+}\hat{a}|n \rangle =n|n \rangle}
\end{empheq}
因此,常称$\hat{a}^{+}\hat{a}$为量子数算符,并记为$\hat{a}^{+}\hat{a}=\hat{n}$.在二次量子化理论中,$\hat{a}^{+}\hat{a}$也称声子数算符.

\eqref{eq46.14}式和\eqref{eq46.17}式表明,算符$\hat{a}$和$\hat{a}^{+}$对能量本征态$|n \rangle $的作用结果是使能级降$\hbar\omega$和升$\hbar\omega$,即量子数$n$降1和升1,
\begin{empheq}{equation}\label{eq46.21}
	\begin{aligned}
		\hat{a}|n \rangle =\lambda(n)|n-1 \rangle \\
		\hat{a}^{+}|n \rangle =\nu(n)|n+1 \rangle 
	\end{aligned}
\end{empheq}
$\lambda$,$\nu$为待定系数,上式的共轭方程是
\begin{empheq}{equation*}\label{eq46.21'}
	\begin{aligned}
		\langle n|\hat{a}=\lambda^{*}(n)\langle n-1| \\
		\langle n|\hat{a}^{+}=\nu^{*}(n)\langle n+1| 
	\end{aligned}	\tag{$4.6.21^{\prime}$}
\end{empheq}
\eqref{eq46.21'}式与\eqref{eq46.21}式取内积,并利用\eqref{eq46.20}式,\eqref{eq46.12}式,以及归一化条件,即得
\begin{empheq}{align*}
	\lambda^{*}(n)\lambda(n)&=\langle n|\hat{a}^{+}\hat{a}|n \rangle =n	\\
	\nu^{*}(n)\nu(n)&=\langle n|\hat{a}\hat{a}^{+}|n \rangle =n+1
\end{empheq}
适当选择态矢量$|n \rangle $的相因子$(e^{i\alpha})$,总可使各$\lambda(n)$,$\nu(n)$为非负实数,因此
\begin{empheq}{equation}\label{eq46.22}
	\lambda(n)=\sqrt{n},\quad \nu(n)=\sqrt{n+1}
\end{empheq}
亦即
\eqindent{6}
\begin{empheq}{equation}\label{eq46.23}
	\boxed{\hat{a}|n \rangle =\sqrt{n|n-1 \rangle },\quad \hat{a}^{+}|n \rangle =\sqrt{n+1}|n+1 \rangle }
\end{empheq}\eqnormal
由此可知,在能量表象中,$\hat{a}$和$\hat{a}^{+}$唯一不等于0的矩阵元类型是
\begin{empheq}{equation}\label{eq46.24}
	\begin{aligned}
		a_{n-1,n}&=\langle n-1|\hat{a}|n \rangle =\sqrt{n}	\\
		a_{n+1,n}^{+}&=\langle n+1|\hat{a}^{+}|n \rangle =\sqrt{n+1}
	\end{aligned}
\end{empheq}
亦即
\begin{empheq}{equation}\label{eq46.25}
	\begin{aligned}
		a_{n^{\prime},n}&=\langle n^{\prime}|\hat{a}|n \rangle =\sqrt{n}\delta_{n-1,n}	\\
		a_{n^{\prime},n}^{+}&=\langle n^{\prime}|\hat{a}^{+}|n \rangle =\sqrt{n+1}\delta_{n+1,n}
	\end{aligned}
\end{empheq}
$\hat{a}^{+}$和$\hat{a}$称为量子数升、降算符.在二次量子化理论中,就是声子的产生和湮灭算符.

算符$\hat{Q},\hat{P}$可以表示成
\begin{empheq}{equation}\label{eq46.26}
	\hat{Q}=\frac{1}{\sqrt{2}}(\hat{a}^{+}+\hat{a}),\quad \hat{P}=\frac{i}{\sqrt{2}}(\hat{a}^{+}-\hat{a})
\end{empheq}
利用\eqref{eq46.23}式,易得
\begin{empheq}{align}
	\hat{Q}|n \rangle &=\sqrt{\frac{n+1}{2}}|n+1 \rangle +\sqrt{\frac{n}{2}}|n-1 \rangle 	\label{eq46.27}	\\
	\hat{P}|n \rangle &=i\sqrt{\frac{n+1}{2}}|n+1 \rangle -i\sqrt{\frac{n}{2}}|n-1 \rangle	\label{eq46.28} 
\end{empheq}
类似地,$\hat{x}$和$\hat{p}$可以表示成
\begin{empheq}{align}%29,30
	\hat{x}&=\sqrt{\frac{\hbar}{\mu\omega}}\hat{Q}=\sqrt{\frac{\hbar}{2\mu\omega}}(\hat{a}^{+}+\hat{a})	\label{eq46.29}\\
	\hat{p}&=\sqrt{\mu\hbar\omega}\hat{P}=i\sqrt{\mu\hbar\omega}(\hat{a}^{+}-\hat{a})	\label{eq46.30}
\end{empheq}
因此
\eqindent{6}
\begin{empheq}{align}%31,32
	\hat{x}|n \rangle &=\sqrt{\frac{\hbar}{2\mu\omega}}( \sqrt{n+1}|n+1 \rangle + \sqrt{n}|n-1 \rangle)	\label{eq46.31}\\
	\hat{p}|n \rangle &=i\sqrt{\frac{\mu\hbar\omega}{2}}( \sqrt{n+1}|n+1 \rangle - \sqrt{n}|n-1 \rangle)	\label{eq46.32}
\end{empheq}
化成$x$表象中的波函数形式,就是
\begin{empheq}{align*}%31',32'
	x\varPsi_{n}(x)&=\sqrt{\frac{\hbar}{2\mu\omega}}[\sqrt{n+1}\varPsi_{n+1}(x)+\sqrt{n}\varPsi_{n-1}(x)]	\tag{$4.6.31^{\prime}$}	\label{eq46.31'}\\
	\frac{d}{dx}\varPsi_{n}(x)&=\sqrt{\frac{\mu\omega}{2\hbar}}[\sqrt{n}\varPsi_{n-1}(x)-\sqrt{n+1}\varPsi_{n+1}(x)]	\tag{$4.6.32^{\prime}$}	\label{eq46.32'}
\end{empheq}\eqnormal

{\heiti 波函数}

基态满足\eqref{eq46.16}式,在$x$表象中,算符$\hat{a}$,$\hat{a}^{+}$可以表示成作用于波函数的算符,
\begin{empheq}{align}\label{eq46.33}
	\hat{a}&=\sqrt{\frac{\mu\omega}{2\hbar}}x+\sqrt{\frac{1}{2\mu\hbar\omega}}\hat{p}	\nonumber\\
	&=\sqrt{\frac{\mu\omega}{2\hbar}}x-i\sqrt{\frac{\hbar}{2\mu\omega}}\frac{\partial}{\partial x}
\end{empheq}
\begin{empheq}{align}\label{eq46.34}
	\hat{a}^{+}&=\sqrt{\frac{\mu\omega}{2\hbar}}x-\sqrt{\frac{1}{2\mu\hbar\omega}}\hat{p}	\nonumber\\
	&=\sqrt{\frac{\mu\omega}{2\hbar}}x+i\sqrt{\frac{\hbar}{2\mu\omega}}\frac{\partial}{\partial x}
\end{empheq}
基态波函数$\varPsi_{0}(x)$满足
\begin{empheq}{equation*}
	\hat{a}\varPsi_{0}(x)=-i\sqrt{\frac{\hbar}{2\mu\omega}}\bigg(\frac{d}{dx}+\frac{\mu\omega}{\hbar}x\bigg)\varPsi_{0}(x)=0
\end{empheq}
令
\begin{empheq}{equation}\label{eq46.35}
	\alpha=\sqrt{\frac{\mu\omega}{\hbar}}
\end{empheq}
上式可以写成
\begin{empheq}{equation}\label{eq46.36}
	\frac{d}{dx}\varPsi_{0}(x)+\alpha^{2}x\varPsi_{0}(x)=0
\end{empheq}
积分,即得
\begin{empheq}{equation}\label{eq46.37}
	\varPsi_{0}(x)=N_{0}e^{-\alpha^{2}x^{2}/2}
\end{empheq}
$N_{0}$为归一化常数,通常取成正实数.由归一化条件
\begin{empheq}{equation}\label{eq46.38}
	\int_{-\infty}^{\infty}|\varPsi_{n}(x)|^{2}dx=1
\end{empheq}
求出
\begin{empheq}{equation}\label{eq46.39}
	N_{0}=\alpha^{\frac{1}{2}}\pi^{-\frac{1}{4}}
\end{empheq}
由\eqref{eq46.23}第二式,取$n=0$,可得$\hat{a}^{+}|0 \rangle=|1 \rangle $,即
\begin{empheq}{equation*}
	\hat{a}^{+}\varPsi_{0}(x)=\varPsi_{1}(x)
\end{empheq}
利用$\hat{a}^{+}$的表示式\eqref{eq46.34},即可求出
\eqindent{6}
\begin{empheq}{equation}\label{eq46.40}
	\varPsi_{1}(x)=\frac{1}{\sqrt{2}}\bigg(\alpha x-\frac{1}{\alpha}\frac{d}{dx}\bigg)\varPsi_{0}=\sqrt{2\alpha}\pi^{-\frac{1}{4}}\alpha xe^{-\alpha^{2}x^{2}/2}
\end{empheq}
普遍地,由\eqref{eq46.23}第二式,取$x$表象,可得
\begin{empheq}{equation}\label{eq46.41}
	\varPsi_{n}(x)=\frac{1}{\sqrt{n}}\hat{a}^{+}\varPsi_{n-1}(x)=\frac{1}{\sqrt{2n}}\bigg(\xi-\frac{d}{d\xi}\bigg)\varPsi_{n-1}
\end{empheq}\eqnormal
其中
\begin{empheq}{equation}\label{eq46.42}
	\xi=\alpha x=\sqrt{\frac{\mu\omega}{\hbar}}x
\end{empheq}
依次递推,得到
\begin{empheq}{align}\label{eq46.43}
	\varPsi_{n}&=(2^{n}n)^{-\frac{1}{2}}\bigg(\xi-\frac{d}{d\xi}\bigg)^{n}\varPsi_{0}	\nonumber\\
	&=\bigg(\frac{\alpha}{\sqrt{\pi}2^{n}n!}\bigg)^{\frac{1}{2}}\bigg(\xi-\frac{d}{d\xi}\bigg)^{n}e^{-\xi^{2}/2}	\nonumber\\
	&=N_{n}H_{n}(\xi)e^{-\xi^{2}/2}
\end{empheq}
其中
\begin{empheq}{equation}\label{eq46.44}
	N_{n}=\bigg(\frac{\alpha}{\sqrt{\pi}2^{n}n!}\bigg)^{\frac{1}{2}}
\end{empheq}
为归一化常数,注意
\begin{empheq}{equation}\label{eq46.45}
	N_{n-1}=\sqrt{2n}N_{n}
\end{empheq}
\eqref{eq46.43}式中,
\begin{empheq}{equation}\label{eq46.46}
	H_{n}(\xi)=e^{\xi^{2}/2}\bigg(\xi-\frac{d}{d\xi}\bigg)^{n}e^{-\xi^{2}/2}
\end{empheq}
易见$H_{n}(\xi)$是$\xi$的$n$次多项式,称为厄密(Hermite)多项式,其数学性质将在附录\ref{A02}详细讨论.

\example 求降算符$\hat{a}$的本征态,将其表示成各能量本征态$|n \rangle $的线性叠加.

\solution 设$\hat{a}$的本征态为$\varPsi_{\alpha}$,态矢量记作$|\alpha \rangle $,本征值$\alpha$.令
\begin{empheq}{equation}\label{eq46.47}
	|\alpha \rangle =\sum_{n=0}^{\infty}C_{n}(\alpha)|n \rangle 
\end{empheq}
代入本征方程
\begin{empheq}{equation}\label{eq46.48}
	\hat{a}|\alpha \rangle =\alpha|\alpha \rangle 
\end{empheq}
利用\eqref{eq46.23}第一式,得到
\begin{empheq}{equation*}
	\hat{a}|\alpha \rangle =\sum_{n}C_{n}(\alpha)\sqrt{n}|n-1 \rangle =\alpha\sum_{n}C_{n}(\alpha)|n \rangle 
\end{empheq}
比较两个$\sum_{n}$项中$|n-1 \rangle$项系数,即得
\begin{empheq}{equation}\label{eq46.49}
	C_{n}(\alpha)=\frac{\alpha}{\sqrt{n}}C_{n-1}(\alpha)
\end{empheq}
依次递推,得到
\begin{empheq}{equation}\label{eq46.50}
	C_{n}(\alpha)=\frac{\alpha^{n}}{\sqrt{n!}}C_{0}(\alpha)
\end{empheq}
$C_{0}$为归一化常数.归一化条件为
\begin{empheq}{equation*}
	\langle \alpha|\alpha \rangle =\sum_{n}|C_{n}|^{2}=|C_{0}|^{2}\sum_{n}\frac{|\alpha|^{2n}}{n!}=1
\end{empheq}
由于
\begin{empheq}{equation*}
	\sum_{n=0}^{\infty}\frac{|\alpha|^{2n}}{n!}=e^{|\alpha|^{2}}
\end{empheq}
所以
\begin{empheq}{equation}\label{eq46.51}
	C_{0}(\alpha)=e^{-\frac{|\alpha|^{2}}{2}}
\end{empheq}
(取$C_{0}$为正实数)这样,就得到$\hat{a}$的归一化本征态矢
\begin{empheq}{equation}\label{eq46.52}
	\boxed{|\alpha \rangle=e^{-\frac{1}{2}|\alpha|^{2}}\sum_{n=0}^{\infty}\frac{\alpha^{n}}{\sqrt{n!}}|n \rangle }
\end{empheq}
由于$\hat{a}$并非厄密算符,本征值$\alpha$可取复数值,不受限制.上式称为谐振子的相干态(coherent states).其中$|n \rangle $态的成分为
\begin{empheq}{equation}\label{eq46.53}
	|\langle n|\alpha \rangle |^{2}=|C_{n}(\alpha)|^{2}=\frac{|\alpha|^{2n}}{n!}e^{-|\alpha|^{2}}
\end{empheq}
呈泊松分布.当$\alpha=0$,\eqref{eq46.52}式变成基态$|0 \rangle $.
