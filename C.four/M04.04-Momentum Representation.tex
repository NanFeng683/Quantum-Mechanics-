\starthis\section[动量表象]{动量表象} \label{sec:04.04} % 
% \makebox[5em][s]{} % 短题目拉间距

考虑质量为$m$的粒子在势场$V(x)$作用下的一维运动,哈密顿算符为
\eqindent{12}
\begin{empheq}{equation}\label{eq44.1}
	\hat{H}=\frac{\hat{p}^{2}}{2m}+\hat{V}(x)
\end{empheq}\eqnormal
薛定谔方程为
\eqindent{6}
\begin{empheq}{equation}\label{eq44.2}
	i\hbar\frac{\partial}{\partial t}|\varPsi(t) \rangle =\hat{H}|\varPsi(t) \rangle=\bigg(\frac{\hat{p}^{2}}{2m}+\hat{V}\bigg)|\varPsi(t) \rangle 
\end{empheq}\eqnormal
采取动量表象,即以$\hat{p}$的本征态矢$|p^{\prime} \rangle $作为基矢.基矢满足本征方程
\begin{empheq}{equation}\label{eq44.3}
	\hat{p}|p^{\prime} \rangle =p^{\prime}|p^{\prime} \rangle ,\quad -\infty<p^{\prime}<\infty
\end{empheq}
以及正交归一化条件和完备性条件:
\begin{empheq}{equation}\label{eq44.4}
	\langle p|p^{\prime} \rangle =\delta(p-p^{\prime})
\end{empheq}
\begin{empheq}{equation}\label{eq44.5}
	\int_{-\infty}^{\infty}|p^{\prime} \rangle dp^{\prime}\langle p^{\prime}|=1
\end{empheq}
将$|\varPsi(t) \rangle $展开成基矢的线性叠加,
\begin{empheq}{align}\label{eq44.6}
	|\varPsi(t) \rangle &=\int|p^{\prime} \rangle dp^{\prime}\langle p^{\prime}|\varPsi(t) \rangle	\nonumber\\
	&=\int|p^{\prime} \rangle  \phi(p^{\prime},t)dp^{\prime}
\end{empheq}
其中
\begin{empheq}{equation}\label{eq44.7}
	\phi(p^{\prime},t)=\langle p^{\prime}|\varPsi(t) \rangle 
\end{empheq}
就是动量表象中的波函数.将展开式\eqref{eq44.6}代入\eqref{eq44.2}式,并以$\langle p|$左乘\eqref{eq44.2}式,得到
\eqindent{6}
\begin{empheq}{align*}
	i\hbar\langle p|\frac{\partial}{\partial t}|\varPsi(t) \rangle &=i\hbar\frac{\partial}{\partial t}\phi(p,t)	\\
	&=\int\langle p|\bigg(\frac{\hat{p}^{2}}{2m}+\hat{V}\bigg)|p^{\prime} \rangle\phi(p^{\prime},t)dp^{\prime}	\\
	&=\int\bigg[\frac{p^{2}}{2m}\delta(p-p^{\prime})+\langle p|\hat{V}|p^{\prime} \rangle \bigg]\phi(p^{\prime},t)dp^{\prime}	\\
	&=\frac{p^{2}}{2m}\phi(p,t)+\int V_{pp^{\prime}}\phi(p^{\prime},t)dp^{\prime}
\end{empheq}
这样,已经求得$p$表象中的薛定谔方程:
\begin{empheq}{equation}\label{eq44.8}
	\boxed{i\hbar\frac{\partial}{\partial t}\phi(p,t)=\frac{p^{2}}{2m}\phi(p,t)+\int V_{pp^{\prime}}\phi(p^{\prime},t)dp^{\prime}}
\end{empheq}\eqnormal
其中$V_{pp^{\prime}}$为$V(x)$的$p$表象矩阵元,即
\begin{empheq}{align}\label{eq44.9}
	V_{pp^{\prime}} &=\langle p|\hat{V}|p^{\prime} \rangle=\int\varPsi_{p}^{*}V(x)\varPsi_{p^{\prime}}(x)dx	\nonumber\\
	&=\frac{1}{2\pi\hbar}\int_{-\infty}^{\infty}V(x)e^{i(p^{\prime}-p)x/\hbar}dx
\end{empheq}
如果$V(x)$可以方便地表示成$x$的正幕级数
\begin{empheq}{equation}\label{eq44.10}
	V(x)=\sum_{n=0}^{\infty}a_{n}x^{n}
\end{empheq}
由于
\begin{empheq}{equation*}
	x^{n}e^{i(p^{\prime}-p)x/\hbar}=\bigg(i\hbar\frac{\partial}{\partial p}\bigg)^{n}e^{i(p^{\prime}-p)x/\hbar}
\end{empheq}
\eqref{eq44.9}式可以化简成
\begin{empheq}{align}\label{eq44.11}
	V_{pp^{\prime}}&=\frac{1}{2\pi\hbar}\sum_{n}a_{n}\bigg(i\hbar\frac{\partial}{\partial p}\bigg)^{n}\int_{-\infty}^{\infty}e^{i(p^{\prime}-p)x/\hbar}dx	\nonumber\\
	&=\sum_{n}a_{n}\bigg(i\hbar\frac{\partial}{\partial p}\bigg)^{n}\delta(p-p^{\prime})	\nonumber\\
	&=V(\hat{x})\delta(p-p^{\prime})
\end{empheq}
其中
\begin{empheq}{equation}\label{eq44.12}
	\hat{x}=i\hbar\frac{\partial}{\partial p}
\end{empheq}
将\eqref{eq44.11}式代入\eqref{eq44.9}式,即得
\begin{empheq}{equation}\label{eq44.13}
	i\hbar\frac{\partial}{\partial t}\phi(p,t)=\bigg[\frac{p^{2}}{2m}+V\bigg(i\hbar\frac{\partial}{\partial p}\bigg)\bigg]\phi(p,t)
\end{empheq}

坐标表象中的波函数$\varPsi(x,t)$与动量表象中的波函数$\phi(p,t)$的关系,可以简单导出如下.以$\langle x|$左乘\eqref{eq44.6}式,并将\eqref{eq44.6}式中$p^{\prime}$改成$p$,得到
\begin{empheq}{equation*}
	\langle x|\varPsi(t) \rangle =\int\langle x|p \rangle \phi(p,t)dp
\end{empheq}
亦即
\begin{empheq}{align}\label{eq44.14}
	\varPsi(x,t)&=\int\varPsi_{p}(x)\phi(p,t)dp	\nonumber\\
	&=\frac{1}{\sqrt{2\pi\hbar}}\int_{-\infty}^{\infty}\varPsi(x,t)e^{ipx/\hbar}dp
\end{empheq}
这正是$\varPsi(x,t)$的傅里叶积分表示式.其逆变换为
\begin{empheq}{equation}\label{eq44.15}
	\phi(p,t)=\frac{1}{\sqrt{2\pi\hbar}}\int_{-\infty}^{\infty}\varPsi(x,t)e^{ipx/\hbar}dx
\end{empheq}
这里的结果与$\S$\ref{sec:03.05}是一致的,只是所用符号不同.

{\heiti 定态}

对于定态,态矢量和波函数随时间的变化均由$e^{-iEt/\hbar}$表示,
\begin{empheq}{align}
	|\varPsi(t) \rangle |\varPsi(t=0) \rangle e^{-iEt/\hbar}	\label{eq44.16}\\
	\phi(p,t)=\phi(p)e^{-iEt/\hbar}	\label{eq44.17}
\end{empheq}
其中
\eqindent{6}
\begin{empheq}{equation}\label{eq44.18}
	\phi(p)=\langle p|\varPsi(t=0) \rangle =\frac{1}{\sqrt{2\pi\hbar}}\int_{-\infty}^{\infty}\varPsi(x,0)e^{ipx/\hbar}dx
\end{empheq}\eqnormal
这种情况下,由\eqref{eq44.8}式可得
\begin{empheq}{equation}\label{eq44.19}
	\boxed{\frac{p^{2}}{2m}\phi(p)+\int V_{pp^{\prime}}\phi(p^{\prime})dp^{\prime}=E\phi(p)}
\end{empheq}
这是动量表象中的定态薛定谔方程,亦即能量本征方程.

\example 在动量表象中求解$\delta$势阱
\begin{empheq}{equation}\label{eq44.20}
	V(x)=-\gamma\delta(x),\quad \gamma>0
\end{empheq}
的束缚态能级和能量本征函数.

\solution 按照\eqref{eq44.9}式,首先算出
\begin{empheq}{align*}
	V_{pp^{\prime}} &=-\frac{\gamma}{2\pi\hbar}\int_{-\infty}^{\infty}\delta(x)e^{i(p^{\prime}-p)x/\hbar}dx\\
	&=-\frac{\gamma}{2\pi\hbar}
\end{empheq}
代入\eqref{eq44.19}式,得到
\eqindent{6}
\begin{empheq}{align}\label{eq44.21}
	\bigg(\frac{p^{2}}{2m}-E\bigg)\phi(p) &=-\int V_{pp^{\prime}}\phi(p^{\prime})dp^{\prime}	\nonumber\\
	&=\frac{\gamma}{2\pi\hbar}\int_{-\infty}^{\infty}\phi(p^{\prime})dp^{\prime}=C\text{与$p$无关}
\end{empheq}\eqnormal
所以
\begin{empheq}{equation}\label{eq44.22}
	\phi(p)=\frac{2mC}{p^{2}-2mE}=\frac{A}{p^{2}-2mE}
\end{empheq}
这就是$p$表象中的能量本征函数,$A$为归一化常数.将\eqref{eq44.22}式代入\eqref{eq44.21}式,得到
\begin{empheq}{equation}\label{eq44.23}
	\int_{-\infty}^{\infty}\frac{dp}{p^{2}-2mE}=\frac{\pi\hbar}{m\gamma}
\end{empheq}
这就是能级方程.束缚态$E<0$,令
\begin{empheq}{equation}\label{eq44.24}
	\beta=\frac{\sqrt{-2mE}}{\hbar}
\end{empheq}
代入\eqref{eq44.23}式,容易算出
\begin{empheq}{equation*}
	\frac{\pi\hbar}{m\gamma}=\int_{-\infty}^{\infty}\frac{dp}{p^{2}+\hbar^{2}\beta^{2}}=\frac{\pi}{\hbar\beta}
\end{empheq}
因此
\begin{empheq}{equation}\label{eq44.25}
	\beta=\frac{m\gamma}{\hbar^{2}},\quad E=-\frac{m\gamma^{2}}{2\hbar^{2}}
\end{empheq}
至此已经求得束缚态能级.波函数由\eqref{eq44.22}式表示,即
\begin{empheq}{equation*}\label{eq44.22'}
	\phi(p)=\frac{A}{p^{2}+\hbar^{2}\beta^{2}}	\tag{$4.4.22^{\prime}$}
\end{empheq}
根据归一化条件,求得
\eqindent{6}
\begin{empheq}{equation*}
	1=\int_{-\infty}^{\infty}|\phi(p)|^{2}dp=A^{2}\int_{-\infty}^{\infty}\frac{dp}{(p^{2}+\hbar^{2}\beta^{2})^{2}}=\frac{\pi A^{2}}{2(\hbar\beta)^{3}}
\end{empheq}\eqnormal
所以
\begin{empheq}{equation}\label{eq44.26}
	A=\bigg(\frac{2}{\pi}\bigg)(\hbar\beta)^{\frac{3}{2}}=\bigg(\frac{2}{\pi}\bigg)\bigg(\frac{m\gamma}{\hbar}\bigg)^{\frac{3}{2}}
\end{empheq}
将\eqref{eq44.22'}式代入\eqref{eq44.14}式,可得$x$表象中的波函数
\begin{empheq}{align*}
	\varPsi(x)&=\frac{1}{\sqrt{2\pi\hbar}}\int_{-\infty}^{\infty}\phi(p)e^{-ipx/\hbar}dp	\\
	&=\frac{A}{\sqrt{2\pi\hbar}}\int_{-\infty}^{\infty}\frac{dp}{p^{2}+\hbar^{2}\beta^{2}}e^{ipx/\hbar}
\end{empheq}
用复平面上围道积分法可以算出
\begin{empheq}{equation}\label{eq44.27}
	\int_{-\infty}^{\infty}\frac{dp}{p^{2}+\hbar^{2}k^{2}}e^{ipx/\hbar}=\frac{\pi}{\hbar\beta}e^{-\beta|x|}
\end{empheq}
因此
\begin{empheq}{equation}\label{eq44.28}
	\varPsi(x)=\sqrt{\beta}e^{-\beta|x|}
\end{empheq}
能级公式\eqref{eq44.25}和波函数\eqref{eq44.25}式与$\S$\ref{sec:02.04}(3.)的结果完全一致.