\section[自由粒子]{自由粒子} \label{sec:05.02} % 
% \makebox[5em][s]{} % 短题目拉间距

自由运动的粒子,总能等于动能(取势能$V=0$),
\begin{empheq}{equation}\label{eq52.1}
	\hat{H}=\frac{\hat{\boldsymbol{p}}^{2}}{2\mu}=-\frac{\hbar^{2}}{2\mu}\nabla^{2}
\end{empheq}
主要守恒量有$\boldsymbol{H},\boldsymbol{p},\boldsymbol{L}$等作为守恒量完全集,可取$(p_{x},p_{y},p_{z}$或$(H,\boldsymbol{L}^{2},L_{z})$.前者的共同本征函数就是自由粒子平面波(参看$\S$\ref{sec:03.05})
\begin{empheq}{equation}\label{eq52.2}
	\varPsi_{\boldsymbol{p}}(\boldsymbol{r})=(2\pi\hbar)^{-\frac{3}{2}}e^{i\boldsymbol{p}\cdot\boldsymbol{r}/\hbar}
\end{empheq}
本节专门讨论$(\hat{\boldsymbol{H}},\hat{\boldsymbol{L}}^{2},\hat{L}_{z})$的共同本征函数,称为自由粒子球面波.按照\eqref{eq51.10}、\eqref{eq51.13}、\eqref{eq51.14}式,波函数可以表示成
\begin{empheq}{equation}\label{eq52.3}
	\varPsi=R_{l}(r)Y_{lm}(\theta,\varphi)=\frac{u_{l}(r)}{kr}Y_{lm}(\theta,\varphi)
\end{empheq}
$u_{l}(r)$满足径向方程
\begin{empheq}{equation}\label{eq52.4}
	\frac{d^{2}}{dr^{2}}u_{l}+\bigg[\frac{2\mu E}{\hbar^{2}}-\frac{l(l+1)}{r^{2}}\bigg]u_{l}=0
\end{empheq}
令
\begin{empheq}{equation}\label{eq52.5}
	k=\frac{\sqrt{2\mu E}}{\hbar},\quad kr=\rho
\end{empheq}
$E,k$可以连续变化(取正值),\eqref{eq52.4}式简化成
\begin{empheq}{equation*}\label{eq52.4'}
	\rho^{2}\frac{d^{2}u_{l}}{d\rho^{2}}+[\rho^{2}-l(l+1)]u_{l}=0
	\tag{$5.2.4^{\prime}$}
\end{empheq}
下面求$u_{l}$及$R_{l}(r)$.根据\eqref{eq51.18}式,令
\begin{empheq}{equation}\label{eq52.6}
	u_{l}=(-1)^{l}\rho^{l+1}W_{l}(\rho)
\end{empheq}
代入\eqref{eq52.4'}式,得到$W_{l}$满足的方程
\begin{empheq}{equation}\label{eq52.7}
	W_{l}^{\prime\prime}+2(l+1)\frac{W_{l}^{\prime}}{\rho}+W_{l}=0
\end{empheq}
再令
\begin{empheq}{equation}\label{eq52.8}
	v_{l}=\frac{W_{l}^{\prime}}{\rho}=\frac{1}{\rho}\frac{dW_{l}}{d\rho}
\end{empheq}
则
\begin{empheq}{equation*}
	W_{l}^{\prime}=\rho v_{l},\quad W_{l}^{\prime\prime}=\rho v_{l}^{\prime}+v_{l}
\end{empheq}
代入\eqref{eq52.7}式,得到
\begin{empheq}{equation*}
	\rho v_{l}^{\prime}+(2l+3)v_{l}+W_{l}=0
\end{empheq}
再微分一次,并用$\rho$除,得到
\begin{empheq}{equation}\label{eq52.9}
	v_{l}^{\prime\prime}+2(l+2)\frac{v_{l}^{\prime}}{\rho}+v_{l}=0
\end{empheq}
比较\eqref{eq52.7}、\eqref{eq52.9}式,可见
\begin{empheq}{equation}\label{eq52.10}
	W_{l+1}=v_{l}=\frac{1}{\rho}\frac{dW_{l}}{d\rho}
\end{empheq}
这是各$W_{l}$之间的递推公式.$l=0$时,\eqref{eq52.4'}式成为
\eqshort
\begin{empheq}{equation}\label{eq52.11}
	u_{0}^{\prime\prime}+u_{0}=0
\end{empheq}\eqnormal
其独立解为
\begin{empheq}{equation}\label{eq52.12}
	u_{0}=\sin\rho,\cos\rho\text{或}e^{i\varphi},e^{-i\varphi}
\end{empheq}
相应的$R_{0}$及$W_{0}$为
\eqlong
\begin{empheq}{equation}\label{eq52.13}
	R_{0}=W_{0}=\frac{u_{0}}{\rho}=\frac{\sin\rho}{\rho},\frac{\cos\rho}{\rho}\text{或}\frac{e^{i\varphi}}{\rho},\frac{e^{-i\varphi}}{\rho}
\end{empheq}\eqnormal
前两种为驻波,后两种为行波这四种解在$r>O$处均满足能量本征方程$\hat{H}\varPsi=E\varPsi$,亦即
\eqshort
\begin{empheq}{equation}\label{eq52.14}
	\nabla^{2}\varPsi+k^{2}\varPsi=0
\end{empheq}\eqnormal
$\bigg($注意,$l=0$时$Y_{00}=\frac{1}{\sqrt{4\pi}}$,$\varPsi$与$R_{0}$只相差一个常系数.$\bigg)$但是,在全空间($r\rightarrow 0$也包括在内)满足\eqref{eq52.14}式的解,当$l=0$时只有\eqref{eq52.13}式的第一种\footnote{$r\rightarrow0$处另外三种解均与$\frac{1}{r}$成比例,不满足\eqref{eq52.14}式.理由参看本章\ref{F.5-1}注.},即
\begin{empheq}{equation}\label{eq52.15}
	R_{0}=W_{0}=\frac{\sin\rho}{\rho}\equiv j_{0}(\rho)
\end{empheq}
利用\eqref{eq52.10}式及\eqref{eq52.6}式, 即可求出
\begin{empheq}{equation*}
	W_{l}=\bigg(\frac{1}{\rho}\frac{d}{d\rho}\bigg)^{l}\bigg(\frac{\sin\rho}{\rho}\bigg)
\end{empheq}\eqlong
\begin{empheq}{align}\label{eq52.16}
	R_{l}(r)&=\frac{u_{l}}{\rho}=(-1)^{l}\rho^{l}W_{l}	\nonumber\\
	&=(-1)^{l}\rho^{l}\bigg(\frac{1}{\rho}\frac{d}{d\rho}\bigg)^{l}\bigg(\frac{\sin\rho}{\rho}\bigg)\equiv j_{l}(\rho)
\end{empheq}
\eqref{eq52.16}式称为球贝塞耳函数,它和第一类贝塞耳函数$J_{\nu}(\rho)$的关系是
\begin{empheq}{equation}\label{eq52.17}
	j_{l}(\rho)=\sqrt{\frac{\pi}{2\rho}}J_{l+\frac{1}{2}}(\rho),\quad l=0,1,2,\cdots
\end{empheq}
这是唯一能够表示成初等函数的贝塞耳函数.

许多问题中常用到$j_{l}(\rho)$的渐近表示,这可以利用\eqref{eq52.16}式求出.当$\rho\rightarrow\infty(\rho\geqslant l)$,略去$\rho^{-2}$以下小量,并利用公式
\begin{empheq}{equation*}
	\frac{d}{d\rho}\sin\rho=\cos\rho=-\sin\bigg(\rho-\frac{\pi}{2}\bigg)
\end{empheq}\eqnormal
容易得出
\begin{empheq}{equation}\label{eq52.18}
	j_{l}(\rho)\sim\frac{1}{\rho}\sin\bigg(\rho-\frac{l\pi}{2}\bigg),\quad \rho\rightarrow\infty
\end{empheq}
当$\rho\rightarrow 0$,利用公式
\begin{empheq}{align*}
	\frac{\sin\rho}{\rho}=&\sum_{n=0}^{\infty}\frac{(-1)^{n}}{(2n+1)!}\rho^{2n}	\\
	\frac{1}{\rho}\frac{d}{d\rho}&=2\frac{d}{d(\rho^{2})}
\end{empheq}
由\eqref{eq52.16}式易得
\eqlong
\begin{empheq}{equation}\label{eq52.19}
	j_{l}(\rho)\rightarrow\frac{\rho^{l}}{(2l+1)!!}=\frac{\rho^{l}}{1\cdot3\cdot5\cdot\cdots\cdot(2l+1)}
\end{empheq}

作为径向波函数,$j_{l}(kr)$显然是不能归一化的,这是游离态波函数的特点.至于$l$相同,能量为$E=\frac{\hbar^{2}k^{2}}{2\mu}$和$E^{\prime}=\frac{\hbar^{2}k^{\prime2}}{2\mu}$的两个径向波函数$j_{l}(kr)$和$j_{l}(k^{\prime}r)$,则有下列正交归一性
\begin{empheq}{align}\label{eq52.20}
	&\int_{0}^{\infty}j_{l}(kr)j_{l}(k^{\prime}r)r^{2}dr	\nonumber\\
	\approx&\frac{1}{kk^{\prime}}\int_{0}^{\infty}\sin\bigg(kr-\frac{l\pi}{2}\bigg)\sin\bigg(k^{\prime}r-\frac{l\pi}{2}\bigg)dr	\nonumber\\
	=&\frac{1}{2kk^{\prime}}\int_{0}^{\infty}[\cos(k-k^{\prime})r+(-1)^{l+1}\cos(k+k^{\prime})r]dr	\nonumber\\
	=&\frac{\pi}{2kk^{\prime}}\delta(k-k^{\prime})
\end{empheq}
计算中利用了渐近公式\eqref{eq52.18}以及
\begin{empheq}{align}\label{eq52.21}
	\delta(k-k^{\prime})&=\frac{1}{2\pi}\int_{-\infty}^{\infty}e^{i(k-k^{\prime})x}dx	\nonumber\\
	&=\frac{1}{\pi}\int_{0}^{\infty}\cos(k-k^{\prime})xdx
\end{empheq}\eqnormal

\example 将沿$z$方向传播的平面波(动量$p_{z}=\hbar k$)$e^{ikz}$展开成球面波的叠加.

\solution 平面波$e^{ikz}$中是\eqref{eq52.14}式的一个特解,它是$(\hat{p}_{x},\hat{p}_{y},\hat{p}_{z})$的共同本征函数.也是$\hat{L}_{z}$的本征函数,
\begin{empheq}{equation*}
	\hat{L}_{z}e^{ikz}=-i\hbar\frac{\partial}{\partial\varphi}e^{ikr\cos\theta}=0
\end{empheq}
即$L_{z}$的本征值为0.但$e^{ikz}$并非$\hat{\boldsymbol{L}}^{2}$的本征函数.将$e^{ikz}$展开成各$Y_{lm}$的线性叠加,必然具有下列形式
\begin{empheq}{equation}\label{eq52.22}
	e^{ikz}=e^{ikr\cos\theta}=\sum_{l=0}^{\infty}C_{l}j_{l}(kr)Y_{l0}(\theta)
\end{empheq}
由于
\begin{empheq}{equation}\label{eq52.23}
	Y_{l0}(\theta)=\sqrt{\frac{2l+1}{4\pi}}P_{l}(\cos\theta)
\end{empheq}
所以\eqref{eq52.22}式也可写成
\begin{empheq}{equation*}\label{eq52.22'}
	e^{ikz}=e^{ikr\cos\theta}=\sum_{l=0}^{\infty}C_{l}^{\prime}j_{l}(kr)P_{l}(\cos\theta)
	\tag{$5.2.22^{\prime}$}
\end{empheq}
其中
\begin{empheq}{equation*}
	C_{l}^{\prime}=\sqrt{\frac{2l+1}{4\pi}}C_{l}
\end{empheq}
为待定系数.将\eqref{eq52.22'}左端展开成幕级数,
\begin{empheq}{equation}\label{eq52.24}
	e^{ikr\cos\theta}=\sum_{0}^{\infty}\frac{1}{l!}(ikr\cos\theta)^{l}
\end{empheq}
其中$kr$的幕次总是和$\cos\theta$的幕次相同.

根据$P_{l}$的定义
\eqlong
\begin{empheq}{equation*}
	P_{l}(\cos\theta)=\frac{1}{2^{l}l!}\bigg(\frac{d}{d\cos\theta}\bigg)^{l}(\cos^{2}\theta-1)^{l}
\end{empheq}\eqnormal
最高次项$(\cos\theta)^{l}$的系数为
\begin{empheq}{equation*}
	\frac{(2l)(2l-1)\cdots(l+1)}{2^{l}l!}
\end{empheq}
而在$j_{l}(kr)$的级数表示式中,最低幕次为$(kr)^{l}$,系数为[见\eqref{eq52.19}式]$\frac{1}{(2l+1)!!}$.因此\eqref{eq52.22'}式右端$(kr\cos\theta)^{l}$项系数为
\begin{empheq}{equation*}
	\frac{C_{l}^{\prime}(2l)(2l-1)\cdots(l+1)}{2^{l}l!(2l+1)!!}
\end{empheq}
与\eqref{eq52.24}式比较,即得
\eqlong
\begin{empheq}{equation}\label{eq52.25}
	C_{l}^{\prime}=\frac{i^{l}2^{l}(2l+1)!!}{(2l)(2l-1)\cdots(l+1)}=(2l+1)i^{l}
\end{empheq}
代入\eqref{eq52.22'}式,即得
\begin{empheq}{equation}\label{eq52.26}
	e^{ikz}=\sum_{l=0}^{\infty}(2l+1)i^{l}j_{l}(kr)P_{l}(\cos\theta)
\end{empheq}
这就是所求展开式.当$kr\rightarrow\infty$,利用\eqref{eq52.18}式可得
\begin{empheq}{equation}\label{eq52.27}
	e^{ikz}\sim\frac{1}{kr}\sum_{l=0}^{\infty}(2l+1)i^{l}\sin\bigg(kr-\frac{l\pi}{2}\bigg)P_{l}(\cos\theta)
\end{empheq}\eqnormal
这个结果对于散射问题很有用.