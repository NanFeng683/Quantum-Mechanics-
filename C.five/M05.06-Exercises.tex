\begin{exercises}
\exercise 限于$l=1$ $(\boldsymbol{L}^{2}=2\hbar^{2})$的情形,利用$Y_{lm}(m=1,0,-1)$具体函数形式,用坐标轮换$(x\rightarrow y,y\rightarrow z,z\rightarrow x)$法求$\boldsymbol{L}^{2},L_{x}$共同本征函数,表示成$Y_{lm}$的线性叠加.再求$\boldsymbol{L}^{2},L_{y}$共同本征函数.

\exercise 限于$l=1$的情形.利用上题结果验证:$L_{x}=0$的本征函数,$L_{y}=0$的本征函数以及$L_{z}=0$的本征函数$(Y_{10})$三者互相正交.你能否不依赖本征函数的具体函数形式而对此结果给予理论的证明?

[提示:利用升降算符方法,求$Y_{10}$态$L_{x}^{2}$及平均值,从而论证$Y_{10}$态下$L_{z}=0$的概率为0.]

\exercise 求$Y_{11}$态下$L_{x}$的可能测值及相应概率,用下述两种方法.

(a) 将$Y_{11}$按$\boldsymbol{L}^{2},L_{x}$,共同本征函数(5-1题)展开.

(b) 对$Y_{11}$态计算$L_{x}$及$L_{x}^{2}$平均值.

\exercise 设粒子状态为$\varPsi=A(x+y+2z)e^{-\lambda r}$,$(\lambda>0)$求

(a) $\boldsymbol{L}^{2}$的取值\quad(b)$\bar{L}_{z}$\quad(c)$L_{z}$等于$\hbar$的概率\quad(d)$L_{x}$的可能取值及相关概率.

\exercise 用不确定关系估算氢原子基态能量.

$\bigg[$提示:取$\Delta r\Delta p\sim\hbar,<r^{-1}>\sim\dfrac{1}{\Delta r}$$\bigg]$

\exercise 质量为$\mu$的粒子在中心力场$V(r)=-\lambda r^{-\frac{3}{2}}(\lambda>0)$中运动,试用不确定度关系估算基态能量.

\exercise 中心力问题的径向方程
\begin{empheq}{equation*}
	\bigg[-\frac{\hbar^{2}}{2\mu}\frac{d^{2}}{dr^{2}}+\frac{l(l+1)\hbar^{2}}{2\mu r^{2}}+V(r)\bigg]u(r)=Eu(r)
\end{empheq}
可以等效地当作一维能量本征方程,$u(r)$相当于波函数,等效能量算符为
\begin{empheq}{equation*}
	H(l,r)=-\frac{\hbar^{2}}{2\mu}\frac{d^{2}}{dr^{2}}+\frac{l(l+1)\hbar^{2}}{2\mu r^{2}}+V(r)
\end{empheq}
能级记为$E_{ln}(E_{l1}<E_{l2}<\cdots)$.试用海尔曼定理证明:$l$越大,$E_{ln}$越高($n$固定).

[提示:视$l$为参数,求$\partial H/\partial l$,再用海尔曼定理.]

\exercise 对于氢原子基态$\varPsi_{100}$,计算$\Delta x,\Delta p$,验证不确定度关系.

\exercise 对于氢原子基态$\varPsi_{100}$,计算电子出现在经典禁区的概率.

\exercise 对于氢原子的$(n,l,m)=(n,n-1,m)$态,计算电子出现在经典禁区的概率,并对$n=1,2,3,4,5$算出这个概率的数值,归纳出它对$n$的依赖关系.

\exercise 对于氢原子的$(n,n-1,m)$态,求最概然半径及$\bar{r}$,$\bar{r^{2}}$,$\Delta r$.

\exercise 原子核发生$\beta^{-}$衰变,电荷突然由$Ze$变成$(Z+1)e$.衰变过程中核外的电子波函数不受影响.对于原来处于原子$(Z)$的K层(1s层,$nlm=100$)的电子.衰变后仍旧处于原子$(Z+1)$的K层的概率等于多少?

\exercise 单价原子的外层电子(价电子)所受原子核及内层电子的平均作用势可以近似表示成
\begin{empheq}{equation*}
	V(r)=-\frac{\e^{2}}{r}-\lambda a_{0}\frac{\e^{2}}{r^{2}},\quad 0<\lambda \leqslant 1
\end{empheq}
试求价电子能级,与氢原子能级比较,列出主量子数$n$的修正数公式.

[提示:将$V(r)$的第二项与离心势能合并.]

\exercise 对于在中心力场$V(r)$中运动的粒子(质量$\mu$)的任何一个$(H,\boldsymbol{L}^{2},L_{z})$共同本征态,证明
\begin{empheq}{equation*}
	\bigg\langle\frac{dV}{dr}\bigg\rangle-\frac{l(l+1)\hbar^{2}}{\mu}\bigg\langle\frac{1}{r^{3}}\bigg\rangle=\frac{2\pi\hbar^{2}}{\mu}|\varPsi(0)|^{2}
\end{empheq}

$\bigg[$提示:将径向方程乘以$du$,再逐项积分.注意$r\rightarrow0,\infty$处的边界条件.也可先求$\bigg[H,\dfrac{\partial}{\partial r}\bigg]$,再对$\varPsi$计算平均值.$\bigg]$

\exercise 对于类氢离子(核电荷$Ze$)的$nlm$态,求$<r^{\nu}>,\nu=-1,-2,-3$.

$\bigg[$ 提示:利用海尔曼定理求出$\bigg\langle\dfrac{1}{r}\bigg\rangle$,$\bigg\langle\dfrac{1}{r^{2}}\bigg\rangle$,再利用上题证明的公式求$\bigg\langle\dfrac{1}{r^{3}}\bigg\rangle$.注意$l=0$时$\bigg\langle\dfrac{1}{r^{3}}\bigg\rangle\rightarrow\infty$ $\bigg]$.

\exercise 对于类氢离子(核电荷$Ze$)的$nlm$态,证明Kramers公式:
\eqindent{1}
\begin{empheq}{equation*}
	\frac{\lambda+1}{n^{2}}<r^{\lambda}>-(2\lambda+1)\frac{a_{0}}{Z}<r^{\lambda-1}>+\frac{\lambda}{4}[(2l+1)^{2}-\lambda^{2}]\frac{a_{0}^{2}}{Z^{2}}<r^{\lambda-2}>=0
\end{empheq}\eqnormal

求出公式成立的条件

[提示:以$r^{\lambda}u$乘径向方程,逐项积分.再以$2r^{\lambda+1}u^{\prime}$乘径向方程,逐项积分将两种结果合并,即可达到目的.]

\exercise 利用上题证明的公式,计算$nlm$态的$<r>,<r^{2}>$.

\exercise 对于氢原子的各s态$(nlm=nOO)$,计算$\Delta x,\Delta p_{x}$,并讨论经典极限$(n\geqslant1)$情况.

[提示:s态各向同性,$\overline{x^{2}}=\dfrac{1}{3}\overline{r^{2}},\overline{p_{z}^{2}}=\dfrac{1}{3}<\boldsymbol{p}^{2}>$.]

\exercise 类氢离子(核电荷$Ze$)$l\geqslant1$的情形下,按下列步骤对径向方程作近似处理:(a)求等效势能$V_{l}=V(r)+\dfrac{l(l+1)\hbar^{2}}{2\mu r^{2}}$极小值及相应距离$r_{l}$.(b)将$V_{l}$在$r_{l}$附近作泰勒展开,保留$(r-r_{l})^{2}$项.(c)视电子的径向运动为$r\sim r_{l}$附近的小振动,给定$l$,求最低能级与低激发能级,并和精确值比较.

\exercise 二维各向同性谐振子是指在势场
\begin{empheq}{equation*}
	V=\frac{1}{2}k(x^{2}+y^{2})=\frac{1}{2}\mu\omega^{2}\rho^{2},\quad \omega=\sqrt{\frac{k}{\mu}}
\end{empheq}
中运动的粒子.$\hat{H},\hat{L}_{z}$共同本征函数表示成$\varPsi=W(\rho)e^{im\varphi}$.写出$W(\rho)$满足的径向方程,并按下列步骤求解.

(a) 令$W(\rho)=u(\rho)\rho^{|m|}e^{-\alpha^{2}\rho^{2}/2}$,$\alpha=\sqrt{\dfrac{\mu\omega}{\hbar}}$

(b) 令$\dfrac{E}{\hbar\omega}=\varepsilon$,$\alpha^{2}\rho^{2}=\xi,u(\rho)=F(\xi)$,证明$F(\xi)$满足合流超几何方程.

(c) 仿照$\S$\ref{sec:05.04},论证$F(\xi)$必须是$n_{p}$次多项式.
 
(d) 确定能级.

\exercise 三维各向同性谐振子
\begin{empheq}{equation*}
	V=\frac{1}{2}k(x^{2}+y^{2}+z^{2})=\frac{1}{2}\mu\omega^{2}r^{2},\omega=\sqrt{\frac{k}{\mu}}
\end{empheq}

$H,\boldsymbol{L}^{2},L_{z}$共同本征函数表示成
\begin{empheq}{equation*}
	\varPsi=R(r)Y_{lm}(\theta,\varphi)=\frac{u(r)}{r}Y_{lm}(\theta,\varphi)
\end{empheq}

写出径向方程,并仿照上题方法求解.

[提示:令$u(r)=F(\xi)\xi^{(l+1)/2}e^{-\xi/2},\xi=\alpha^{2}r^{2}$.]

\exercise 讨论5-20,5-21题中能级的简并度,与2-15,2-16题的结果对照.

\exercise 对于三维各向同性谐振子的基态波函数,计算最概然半径,$\bar{r}$及$\Delta r$.

\exercise 仿照$\S$\ref{sec:05.04}例题所用方法,对于三维各向同性谐振子的$(n,lm)$态,计算离心势能$\dfrac{\boldsymbol{L}^{2}}{2\mu r^{2}}$的平均值.


\end{exercises}