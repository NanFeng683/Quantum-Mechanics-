\ctexset{chapter={number={\quad},name={}}}  % 应放于cls文件中,chapter计数始
\chapter{致读者}\label{chp:0}
\ctexset{chapter={number={\chinese{chapter}},name={第,章}}}

高等教育自学考试物理专业本科阶段设有理论力学、热力学与统计物理学、电动力学、量子力学以及数学物理方法等课程。这些课程理论要求较高,全日制高校的学生学习起来,也是不轻松的。对这些课程,国内已先后出版了许多很好的教科书,但这些教科书都是与系统讲授并辅之以其他教学环节这种教学方式相适应的,对自学不尽合用。自学高考的学生及有志于提高自己物理素养的各方面读者,切望有一套与现有教材相比有不同特点的、比较适合于自学的理论物理自学教材供他们使用。值得高兴的是、许多高校有经验的教师、专家和出版社都热情支持理论物理自学教材的出版工作。课程的自学考试大纲只规定了每门课程的自学和考试的要求,不同的作者根据大纲编写的教材,还能反映作者对课程内容的理解和体会,还有自己的讲述方式和自己的特点。我们认为,发动社会力量编写和出版符合大纲要求的,不同风格的理论物理自学教材供读者选用,无疑是有益的,电子工业出版社组织的这套《理论物理自学丛书》将是最早出版的一套,《丛书》的内容符合自学考试大纲的要求,并力求适应自学的特点。

物理专业委员会将这套《理论物理自学丛书》作为自学考试“建议试用”教材之一,愿这套自学丛书对自学考试、成人教育,对工程技术人员和全日制高校的教师和学生都有裨益。


%\hspace{6.8cm}\zihao{-5}\kaishu{全国高等教育自学考试指导委员会	\\ 物理专业委员会}
%\hspace{5cm}\normalfont{} \zihao{-5}一九八八年四月\normalsize

\mbox{} 

\leftskip=45mm 全国高等教育自学考试指导委员会 

\leftskip=70mm 物理专业委员会

\pskip
\leftskip=70mm 一九八八年四月

\leftskip=0mm
\clearpage