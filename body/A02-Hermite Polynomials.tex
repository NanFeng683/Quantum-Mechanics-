\clearpage
\chapter{厄密多项式}	\label{A02}

令$\xi,\eta$为独立的复变数,考虑下列在整个复平面解析的二元函数
\eqshort
\begin{empheq}{equation}\label{eqA2.1}
	W(\xi,\eta)=e^{2\xi\eta-\eta^{2}}
\end{empheq}
将它展开成$\eta$的幕级数,表示成
\begin{empheq}{equation}\label{eqA2.2}
	W(\xi,\eta)=\sum_{n=0}^{\infty}H_{n}(\xi)\frac{\eta^{n}}{n!}
\end{empheq}
则 
\begin{empheq}{equation*}
	H_{n}(\xi)=\frac{\partial^{n}W}{\partial\eta^{n}}\bigg|_{\eta=0}
\end{empheq}\eqlong
由于
\begin{empheq}{align*}
	W&=e^{2\xi\eta-\eta^{2}}=e^{\eta^{2}-t^{2}},\quad t=\xi-\eta	\\
	\frac{\partial^{n}W}{\partial\eta^{n}}=&(-1)^{n}\frac{\partial^{n}W}{\partial t^{n}}=(-1)^{n}e^{\xi^{2}}\left(\frac{d}{dt}\right)^{n}e^{-t^{2}}	\\
	&\left(\frac{d}{dt}\right)^{n}e^{-t^{2}}|_{\eta=0}=\left(\frac{d}{d\xi}\right)^{n}e^{-\xi^{2}}
\end{empheq}\eqnormal
所以
\begin{empheq}{equation}\label{eqA2.3}
	H_{n}(\xi)=(-1)^{n}e^{\xi^{2}}\left(\frac{d}{d\xi}\right)^{n}e^{-\xi^{2}}
\end{empheq}
显然$H_{n}(\xi)$是$n$次多项式,称为厄密(Hermite)多项式.上式就是它的微分表达式.$W(\xi,\eta)$称为$H_{n}(\xi)$的生成函数或母函数.

\eqref{eqA2.2}式对$\xi$求导,得到
\begin{empheq}{align*}
	\sum_{n}H_{n}^{\prime}(\xi)\frac{\eta^{n}}{n!}&=2\eta W(\xi,\eta)	\\
	&=\sum_{n}2H_{n}(\xi)\frac{\eta^{n+1}}{n!}
\end{empheq}\eqshort
比较上式两端$\eta^{n}$系数,即得
\begin{empheq}{equation}\label{eqA2.4}
	H_{n}^{\prime}(\xi)=2nH_{n-1}(\xi)
\end{empheq}\eqnormal
\eqref{eqA2.2}式对$\eta$求导,经过类似的步骤,得到
\begin{empheq}{equation}\label{eqA2.5}
	2\xi H_{n}(\xi)=H_{n+1}(\xi)+2nH_{n-1}(\xi)
\end{empheq}
以上二式是厄密多项式的基本递推关系.由\eqref{eqA2.4}、\eqref{eqA2.5}式消去$H_{n-1}$,再对$\xi$求导,再利用\eqref{eqA2.4}式消去$H_{n+1}^{\prime}$,就得$H_{n}$满足的微分方程
\begin{empheq}{equation}\label{eqA2.6}
	H_{n}^{\prime\prime}(\xi)-2\xi H_{n}^{\prime}(\xi)+2nH_{n}(\xi)=0
\end{empheq}
称为厄密方程.\eqref{eqA2.3}式是这个方程的唯一多项式解.

将\eqref{eqA2.2}式中$\eta$换成另一个变数$s$,
\begin{empheq}{equation*}
	e^{2\xi s-s^{2}}=\sum_{m=0}^{\infty}H_{m}(\xi)\frac{s^{m}}{m!}
\end{empheq}
与\eqref{eqA2.2}式相乘,得到
\begin{empheq}{align*}
	&\sum_{n}\sum_{m}H_{n}(\xi)H_{m}(\xi)\frac{\eta^{n}s^{m}}{n!m!}	\\
	=&\exp(2\xi\eta+2\xi s-\eta^{2}-s^{2})	\\
	=&\exp[\eta^{2}+2\eta s-(\xi-\eta-s)^{2}]
\end{empheq}
以$e^{-\xi^{2}}$乘上式两端,并积分$\int_{-\infty}^{\infty}\cdots d\xi$,得到
\begin{empheq}{align*}
	&\sum_{n}\sum_{m}\frac{\eta^{n}s^{m}}{n!m!}\int_{-\infty}^{\infty}e^{-\xi^{2}}H_{n}(\xi)H_{m}(\xi)d\xi	\\
=& e^{2\eta s}\int_{-\infty}^{\infty}e^{-(\xi-\eta-s)^{2}}d\xi	\\
=& e^{2\eta s}\int_{-\infty}^{\infty}e^{-x^{2}}dx=\sqrt{\pi}e^{2\eta s}	\\
=& \sqrt{\pi}\sum_{n=0}^{\infty}2^{n}\frac{(\eta s)^{n}}{n!}
\end{empheq}
比较上式两端,即得厄密多项式的正交归一化公式
\begin{empheq}{equation}\label{eqA2.7}
	\int_{-\infty}^{\infty}e^{-\xi^{2}}H_{n}(\xi)H_{m}(\xi)d\xi=2^{n}n!\sqrt{\pi}\delta_{nm}
\end{empheq}
其中$e^{-\xi^{2}}$为权函数.换言之,下列函数系$\{\phi_{n}(\xi)\}$是$-\infty<\xi<\infty$区域的正交归一化完备函数系:
\begin{empheq}{equation}\label{eqA2.8}
	\phi_{n}(\xi)=(2^{n}n!\sqrt{\pi})^{-\frac{1}{2}}e^{-\xi^{2}/2}H_{n}(\xi)
\end{empheq}
由\eqref{eqA2.3}式可知,$H_{n}(\xi)$及$\phi_{n}(\xi)$的宇称为$(-1)^{n}$,即
\begin{empheq}{align}
	H_{n}(-\xi)=(-1)^{n}H_{n}(\xi)		\label{eqA2.9}\\
	\phi_{n}(-\xi)=(-1)^{n}\phi_{n}(\xi)		\label{eqA2.10}
\end{empheq}\eqlong
前几种厄密多项式是
\begin{empheq}{equation}\label{eqA2.11}
	\begin{aligned}
		H_{0}&(\xi)=1,\quad H_{1}(\xi)=2\xi	\\
		H_{2}(\xi)=&4\xi^{2}-2,\quad H_{3}(\xi)=8\xi^{3}-12\xi	\\
		H_{4}&(\xi)=16\xi^{4}-48\xi^{2}+12
	\end{aligned}
\end{empheq}\eqnormal
注意,$H_{n}(\xi)$中最高次项是$(2\xi)^{n}$,这一点可从\eqref{eqA2.3}式直接看出.

$H_{n}(\xi)$还有另一种微分表达式,即
\begin{empheq}{equation}\label{eqA2.12}
	H_{n}(\xi)=e^{\eta^{2}/2}\left(\xi-\frac{d}{d\xi}\right)^{n}e^{-\xi^{2}/2}
\end{empheq}
用数学归纳法容易证明\eqref{eqA2.12}式与\eqref{eqA2.3}式是等价的,读者可自行证明之.