\achapter
% \stepcounter{answer}


% 1
\answer (a) $ H=H_{0}(x_{1})+H_{0}(x_{2}) $

\aindent 	$ =\bigg(-\dfrac{\hbar^{2}}{2m}\dfrac{\partial^{2}}{\partial x_{1}^{2}}+\dfrac{1}{2}kx_{1}^{2}\bigg)+\bigg(-\dfrac{\hbar^{2}}{2m}\dfrac{\partial^{2}}{\partial x_{2}^{2}}+\dfrac{1}{2}kx_{2}^{2}\bigg) $

\aindent 	$ =\bigg(-\dfrac{\hbar^{2}}{2m_{\text{总}}}\dfrac{\partial^{2}}{\partial X^{2}}+kX^{2}\bigg)+\bigg(-\dfrac{\hbar^{2}}{2\mu}\dfrac{\partial^{2}}{\partial x^{2}}+\dfrac{1}{4}kx^{2}\bigg) $

\aindent 	$ m_{\text{总}}=2m(\text{总质量}),\mu=\dfrac{m_{1}m_{2}}{m_{1}+m_{2}}(\text{折合质量}) $

\aindent 	$ \dfrac{k}{2}=\dfrac{m\omega^{2}}{2},\dfrac{\omega}{2\pi}$为单粒子振动频率

\aindent 	$ k=\dfrac{m_{\text{总}}\omega^{2}}{2},\dfrac{k}{4}=\dfrac{\mu\omega^{2}}{2},$质心运动与相对运动都是谐振动,

\aindent	频率都是$\dfrac{\omega}{2\pi}$

		(b) $ \varPsi_{0}(x_{i})=\exp\bigg(-\dfrac{x_{i}^{2}}{2x_{0}^{2}}\bigg),\quad x_{0}^{2}=\dfrac{\hbar}{m\omega}=\dfrac{\hbar}{\sqrt{mk}} $

\aindent 	$ \varPsi_{1}(x_{i})=x_{i}\varPsi_{0}(x_{i}),i=1,2. $

\aindent 	$ \varPsi^{S}(x_{1},x_{2})=\varPsi_{1}(x_{1})\varPsi_{0}(x_{2})+\varPsi_{0}(x_{1})\varPsi_{1}(x_{2}) $

\aindent	$ \quad\qquad\qquad=\varPsi_{1}(X)\varPsi_{0}(x) $

\aindent 	$ \varPsi^{A}(x_{1},x_{2})=\varPsi_{1}(x_{1})\varPsi_{0}(x_{2})-\varPsi_{0}(x_{1})\varPsi_{1}(x_{2}) $

\aindent	$ \quad\qquad\qquad=\varPsi_{0}(X)\varPsi_{1}(x) $

% 2
\answer $ E_{Nn}=\bigg(N+\dfrac{1}{2}\bigg)\hbar\omega_{1}+\bigg(n+\dfrac{1}{2}\bigg)\hbar\omega,\quad N,n=0,1,2,\cdots $

		$ \omega_{1}=\sqrt{\dfrac{\alpha}{m}},\quad \omega_{2}=\sqrt{\dfrac{\alpha+2\beta}{m}} $

% 3
\answer $ \Psi_{\alpha\alpha\alpha}=\varPsi_{\alpha}(1)\varPsi_{\alpha}(2)\varPsi_{\alpha}(3),\Psi_{\beta\beta\beta},\Psi_{\gamma\gamma\gamma}$类似

		$ \Psi_{\alpha\alpha\beta}=\dfrac{1}{\sqrt{3}}[\varPsi_{\alpha}(1)\varPsi_{\alpha}(2)\varPsi_{\beta}(3)+\varPsi_{\alpha}(2)\varPsi_{\alpha}(3)\varPsi_{\beta}(1)+\varPsi_{\alpha}(3)\varPsi_{\alpha}(1)\varPsi_{\beta}(2)] $
		
		$ \Psi_{\alpha\alpha\gamma},\Psi_{\beta\beta\alpha},\Psi_{\beta\beta\gamma},\Psi_{\gamma\gamma\alpha},\Psi_{\gamma\gamma\beta}$类似
		
		$ \Psi_{\alpha\beta\gamma}=\dfrac{1}{\sqrt{6}}[\varPsi_{\alpha}(1)\varPsi_{\beta}(2)\varPsi_{\gamma}(3)+\varPsi_{\alpha}(2)\varPsi_{\beta}(1)\varPsi_{\gamma}(3)+\cdots\text{(共6项)}]$

% 4
\answer $ \Psi_{0}(x_{1},x_{2})=\dfrac{2}{a}\sin\bigg(\dfrac{\pi x_{1}}{a}\bigg)\sin\bigg(\dfrac{\pi x_{2}}{a}\bigg) $

		$ \bar{x}=0,\quad \Delta x=(\num{0.2556})a,\quad \bar{X}=\dfrac{a}{2},\quad \Delta X=(\num{0.1278})a $
		
		$ \Psi_{1}(x_{1},x_{2}=\dfrac{\sqrt{2}}{a}\bigg[ \sin\bigg(\dfrac{\pi x_{1}}{a}\bigg)\sin\bigg(\dfrac{2\pi x_{2}}{a}\bigg)+\sin\bigg(\dfrac{2\pi x_{1}}{a}\bigg)\sin\bigg(\dfrac{\pi x_{2}}{a}\bigg) \bigg] $

% 5
\answer $ E_{1}=\dfrac{\pi^{2}\hbar^{2}}{ma^{2}}+\dfrac{3V_{0}}{2},\quad E_{2}=\dfrac{5\pi^{2}\hbar^{2}}{2ma^{2}},\quad E_{3}=\dfrac{5\pi^{2}\hbar^{2}}{2ma^{2}}+2V_{0} $

% 6
\answer (a) $ E=-Z\bigg(Z-\dfrac{5}{8}\bigg)\dfrac{\e^{2}}{a_{0}},\quad a_{0}=\dfrac{\hbar^{2}}{m_{e}\e^{2}} $
	
		(b) $ E=-Z\bigg(Z-\dfrac{5}{16}\bigg)^{2}\dfrac{\e^{2}}{a_{0}} $

% 7
\answer (a) $ E=-Z\bigg(Z-\dfrac{1}{2}\bigg)\dfrac{\e^{2}}{a_{0}} $

		(b) $ E=-Z\bigg(Z-\dfrac{1}{4}\bigg)^{2}\dfrac{\e^{2}}{a_{0}} $


% 8
\answer $ C(a,b)+C(b,c)+C(c,a)-J(a,b)+J(b,c)+J(c,a) $

		$C,J$表示式类似于\eqref{eqx3.16}、\eqref{eqx3.17}式

% 9
\answer 对称态$(S+1)(2S+1)$种,反对称态$S(2S+1)$种

% 10
\answer (a) $ Z=4,20,36 \qquad$	(b) $ Z=6,12,18 $

\stepcounter{answer}	\stepcounter{answer}

% 13
\answer (a) $ \chi(t)\left[\alpha(1)\beta(2)\cos\bigg(\dfrac{\omega t}{2}\bigg)-\beta(1)\alpha(2)\sin\bigg(\dfrac{\omega t}{2}\bigg) \right]e^{\frac{i\omega t}{4}} $

		(b) $ \cos^{2}\bigg(\dfrac{\omega t}{2}\bigg)\quad $ (c)$ 0\quad $ (d) $ \dfrac{1}{2},\dfrac{1}{2} $

\stepcounter{answer}

% 15
\answer $\pi^{-}$介子宇称为奇
