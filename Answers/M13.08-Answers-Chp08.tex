\newpage
\achapter
% \stepcounter{answer }

% 1
\answer $ \delta_{l}=-\dfrac{\pi}{2}(\nu-l),\quad \nu(\nu+1)=l(l+1)+\dfrac{2\mu\alpha}{\hbar^{2}} $

		$ \delta_{l}\approx0\dfrac{\pi\mu\alpha}{(2l+1)\hbar^{2}},\quad f(\theta)\approx-\dfrac{\pi\mu\alpha}{2\hbar^{2}k\sin\bigg(\dfrac{\theta}{2}\bigg)} $

% 2
\answer $ f(\theta)=--\dfrac{\pi\mu\alpha}{2\hbar^{2}k\sin\bigg(\dfrac{\theta}{2}\bigg)} $

% 3
\answer (a) $ \delta_{0}\approx-\dfrac{2\mu V_{0}ka^{3}}{3\hbar^{2}},\quad|\delta_{0}|\ll 1 $

	\aindent$ f(\theta)\approx\dfrac{\delta_{0}}{k}\approx-\dfrac{2\mu V_{0}a^{3}}{3\hbar^{2}} $
	
	\aindent$ \sigma_{\text{总}}=4\pi f^{2}\approx\bigg(\dfrac{16\pi}{9}\bigg)\bigg(\dfrac{\mu V_{0}a^{3}}{\hbar^{2}}\bigg)^{2} $
	
		(b) $ f,\sigma_{\text{总}}$同上.适用条件$\dfrac{\mu V_{0}a^{3}}{\hbar^{2}}\ll 1$

% 4
\answer $ f(\theta)=\dfrac{1}{k}(e^{i\delta_{0}}\sin\delta_{0}+3e^{i\delta_{1}}\sin\delta_{1}\cos\theta) $

		$ \sigma(\theta)=\dfrac{1}{k^{2}}[\sin^{2}\delta_{0}+9\sin^{2}\delta_{1}\cos^{2}\theta+6\sin\delta_{0}\sin\delta_{1}\cos(\delta_{0}-\delta_{1})\cos\theta] $
		
		当$ \delta_{0}=\dfrac{\pi}{9},\quad \delta_{1}=\dfrac{\pi}{36},\quad \dfrac{\sigma(\pi)}{\sigma(0)}=\num{0.0353}, $
		
		$ \dfrac{\sigma(\pi)}{\sigma\bigg(\frac{\pi}{2}\bigg)}=\num{0.1080},\quad \dfrac{\sigma\bigg(\frac{\pi}{2}\bigg)}{\sigma(0)}=\num{0.3266} $

% 5
\answer (a) $ f(\theta)=-\dfrac{\mu A}{2\pi\hbar^{2}},\quad \sigma(\theta)=\dfrac{\mu^{2}A^{2}}{4\pi^{2}\hbar^{4}} $

		(b) $ f(\theta)=-\dfrac{4\mu\alpha V_{0}}{\hbar^{2}(\alpha^{2}+q^{2})^{2}},\quad \sigma(\theta)=\dfrac{16\mu^{2}\alpha^{2}V_{0}^{2}}{\hbar^{4}(\alpha^{2}+q^{2})^{4}},\quad q=2k\sin\bigg(\dfrac{\theta}{2}\bigg) $
		
		(c) $ f(\theta)=-\bigg(\dfrac{\sqrt{\pi}\mu V_{0}}{2\hbar^{2}\alpha^{3}}\bigg)\exp\bigg(\dfrac{-q^{2}}{4\alpha^{2}}\bigg),\quad \sigma(\theta)=\bigg(\dfrac{\pi\mu^{2}V_{0}^{2}}{4\hbar^{4}\alpha^{6}}\bigg)\exp\bigg(\dfrac{-q^{2}}{2\alpha^{2}}\bigg)  $

% 6
\answer $ f(\theta)=-\bigg(\dfrac{2\mu a^{2}B}{\hbar^{2}}\bigg)\dfrac{\sin qa}{qa},\quad q=2k\sin\bigg(\dfrac{\theta}{2}\bigg), $

		$ \sigma(\theta)=\bigg(\dfrac{2\mu a^{2}B}{\hbar^{2}}\bigg)^{2}\dfrac{(\sin qa)^{2}}{(qa)^{2}} $
		
		当$ \theta>\dfrac{1}{ka},\quad \sigma(\theta)\approx\dfrac{1}{2}\left[\dfrac{\mu aB}{\hbar^{2}k\sin(\theta/2)}\right]^{2} $

% 7
\answer $ a_{0}=\dfrac{\sqrt{2\mu\alpha}}{\hbar},\quad \delta_{0}=-\dfrac{k\sqrt{2\mu\alpha}}{\hbar} $

		$ f(\theta)=-a_{0},\quad \sigma_{\text{总}}\approx 4\pi a_{0}^{2}=\dfrac{8\pi\mu\alpha}{\hbar^{2}} $

% 8
\answer $ f_{0}=\dfrac{i}{k},\quad \sigma_{0}=4\pi|f_{0}|^{2}=\dfrac{2\pi\hbar^{2}}{\mu E} $

% 9
\answer $ \sigma(0)=\bigg(\dfrac{m_{e}\e A}{3\hbar}\bigg)^{2} $

		基态氢原子,$ A=-3\e a_{0}^{2},\sigma(0)=a_{0}^{2},\quad a_{0}=\dfrac{\hbar^{2}}{m_{e}\e} $(玻尔半径)

% 10
\answer $ \rho(r)=\dfrac{1}{\pi a_{0}^{3}}e^{-2r/a_{0}},\quad F(q)=\dfrac{16}{(4+q^{2}a_{0^{2}})^{2}} $

		$ \theta\ll\dfrac{1}{a_{0}k}$即$ qa_{0}\ll 1,\left[q=2k\sin\bigg(\dfrac{\theta}{2}\bigg)\right]$对于这种小散射角,
		
		$ F(q)\approx \dfrac{1-q^{2}a_{0}^{2}}{2},\quad f(\theta)\approx a_{0}\approx\dfrac{\hbar^{2}}{\mu \e^{2}},\sigma(\theta)\approx a_{0}^{2} $